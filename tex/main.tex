\documentclass{book}
\usepackage{siunitx}
\usepackage{xcolor}
\usepackage[utf8]{inputenc}
\usepackage{booktabs}
\usepackage{graphicx}
\usepackage{geometry}
\usepackage{microtype}
\usepackage{hyperref}

\title{Tables of Psychrometric Properties at Sea Level Pressure}
\author{Mitchell T. Paulus}


\newcommand{\tdb}{\(T_{db}\)}
\newcommand{\twb}{\(T_{wb}\)}
\newcommand{\tdp}{\(T_{dp}\)}
\newcommand{\rh}{\(\phi\)}
\newcommand{\spechum}{\(\omega\)}
\newcommand{\enthalpy}{\(h\)}

\begin{document}

\maketitle{}

\tableofcontents{}

\chapter{Preface}

These charts grew out of the need to be able to quickly find
psychrometric properties for the PE exam. The tables are only in the
I.P. unit system as the PM portion of the PE exam only consists of
problems in these units.

The table entries were completed using the formulations given in
\textit{ASHRAE Fundamentals 2017}. Temperatures between 32°F and 100°F
were considered. Entries with enthalpy values greater than 60
BTU/lb\textsubscript{da}
were not printed.

For the \tdb{}, \rh{} table, entries were skipped if the difference
between the previous values for both \spechum{} and \enthalpy{} were not
greater than 0.0025 and 0.5 BTU/lb\textsubscript{da} respectively.

\begin{table}
\centering
\caption{Symbols and units.}
\label{tab:}
\begin{tabular}{lll}
\toprule
    \tdb & Dry bulb temperature & °F \\
    \twb & Wet bulb temperature & °F \\
    \tdp & Dew point temperature & °F \\
    \rh & Relative humidity & \% (0-100) \\
    \spechum & Specific humidity & 0-1 \\
    \enthalpy & Enthalpy & BTU/(lb\textsubscript{da}) \\
\end{tabular}

\end{table}


\chapter{\(T_{db}\) and \(T_{wb}\) }

The following charts are based on integer combinations of \(T_{db}\) and
\(T_{wb}\).

\newpage
{
\small

\begin{tabular}{llll|llll|llll}
 \toprule 
\(T_{db}\) & \(T_{wb}\) & \(\omega\) & \(h\) & \(T_{db}\) & \(T_{wb}\) & \(\omega\) & \(h\) & \(T_{db}\) & \(T_{wb}\) & \(\omega\) & \(h\)  \\ \midrule 
100 & 93 & 0.0325 & 59.96 & 99 & 80 & 0.0177 & 43.36 & 98 & 67 & 0.0070 & 31.29\\
100 & 92 & 0.0312 & 58.48 & 99 & 79 & 0.0168 & 42.29 & 98 & 66 & 0.0063 & 30.51\\
100 & 91 & 0.0299 & 57.04 & 99 & 78 & 0.0158 & 41.24 & 98 & 65 & 0.0056 & 29.74\\
100 & 90 & 0.0286 & 55.63 & 99 & 77 & 0.0149 & 40.23 & 98 & 64 & 0.0050 & 28.99\\
100 & 89 & 0.0274 & 54.26 & 99 & 76 & 0.0140 & 39.23 & 98 & 63 & 0.0043 & 28.26\\
100 & 88 & 0.0262 & 52.92 & 99 & 75 & 0.0131 & 38.26 & 98 & 62 & 0.0036 & 27.54\\
100 & 87 & 0.0250 & 51.61 & 99 & 74 & 0.0123 & 37.32 & 98 & 61 & 0.0030 & 26.84\\
100 & 86 & 0.0238 & 50.34 & 99 & 73 & 0.0114 & 36.39 & 98 & 60 & 0.0024 & 26.16\\
100 & 85 & 0.0227 & 49.10 & 99 & 72 & 0.0106 & 35.49 & 98 & 59 & 0.0018 & 25.49\\
100 & 84 & 0.0216 & 47.89 & 99 & 71 & 0.0098 & 34.61 & 98 & 58 & 0.0012 & 24.83\\
100 & 83 & 0.0205 & 46.71 & 99 & 70 & 0.0090 & 33.75 & 98 & 57 & 0.0006 & 24.19\\
100 & 82 & 0.0195 & 45.56 & 99 & 69 & 0.0083 & 32.91 & 98 & 56 & 0.0000 & 23.57\\
100 & 81 & 0.0185 & 44.44 & 99 & 68 & 0.0075 & 32.09 & 97 & 92 & 0.0319 & 58.53\\
100 & 80 & 0.0175 & 43.34 & 99 & 67 & 0.0068 & 31.29 & 97 & 91 & 0.0306 & 57.08\\
100 & 79 & 0.0165 & 42.28 & 99 & 66 & 0.0061 & 30.50 & 97 & 90 & 0.0293 & 55.67\\
100 & 78 & 0.0156 & 41.23 & 99 & 65 & 0.0054 & 29.74 & 97 & 89 & 0.0281 & 54.30\\
100 & 77 & 0.0147 & 40.22 & 99 & 64 & 0.0047 & 28.99 & 97 & 88 & 0.0269 & 52.96\\
100 & 76 & 0.0138 & 39.22 & 99 & 63 & 0.0041 & 28.25 & 97 & 87 & 0.0257 & 51.65\\
100 & 75 & 0.0129 & 38.25 & 99 & 62 & 0.0034 & 27.54 & 97 & 86 & 0.0245 & 50.38\\
100 & 74 & 0.0120 & 37.31 & 99 & 61 & 0.0028 & 26.84 & 97 & 85 & 0.0234 & 49.14\\
100 & 73 & 0.0112 & 36.38 & 99 & 60 & 0.0022 & 26.15 & 97 & 84 & 0.0223 & 47.93\\
100 & 72 & 0.0104 & 35.48 & 99 & 59 & 0.0016 & 25.48 & 97 & 83 & 0.0213 & 46.75\\
100 & 71 & 0.0096 & 34.60 & 99 & 58 & 0.0010 & 24.83 & 97 & 82 & 0.0202 & 45.60\\
100 & 70 & 0.0088 & 33.74 & 99 & 57 & 0.0004 & 24.19 & 97 & 81 & 0.0192 & 44.47\\
100 & 69 & 0.0081 & 32.90 & 98 & 93 & 0.0330 & 59.99 & 97 & 80 & 0.0182 & 43.38\\
100 & 68 & 0.0073 & 32.08 & 98 & 92 & 0.0317 & 58.51 & 97 & 79 & 0.0172 & 42.31\\
100 & 67 & 0.0066 & 31.28 & 98 & 91 & 0.0304 & 57.07 & 97 & 78 & 0.0163 & 41.27\\
100 & 66 & 0.0059 & 30.49 & 98 & 90 & 0.0291 & 55.66 & 97 & 77 & 0.0154 & 40.25\\
100 & 65 & 0.0052 & 29.73 & 98 & 89 & 0.0278 & 54.28 & 97 & 76 & 0.0145 & 39.25\\
100 & 64 & 0.0045 & 28.98 & 98 & 88 & 0.0266 & 52.94 & 97 & 75 & 0.0136 & 38.28\\
100 & 63 & 0.0038 & 28.25 & 98 & 87 & 0.0255 & 51.64 & 97 & 74 & 0.0127 & 37.34\\
100 & 62 & 0.0032 & 27.53 & 98 & 86 & 0.0243 & 50.37 & 97 & 73 & 0.0119 & 36.41\\
100 & 61 & 0.0026 & 26.83 & 98 & 85 & 0.0232 & 49.13 & 97 & 72 & 0.0111 & 35.51\\
100 & 60 & 0.0019 & 26.15 & 98 & 84 & 0.0221 & 47.92 & 97 & 71 & 0.0103 & 34.63\\
100 & 59 & 0.0013 & 25.48 & 98 & 83 & 0.0210 & 46.74 & 97 & 70 & 0.0095 & 33.77\\
100 & 58 & 0.0007 & 24.82 & 98 & 82 & 0.0200 & 45.59 & 97 & 69 & 0.0087 & 32.93\\
100 & 57 & 0.0002 & 24.18 & 98 & 81 & 0.0190 & 44.46 & 97 & 68 & 0.0080 & 32.10\\
99 & 93 & 0.0328 & 59.98 & 98 & 80 & 0.0180 & 43.37 & 97 & 67 & 0.0073 & 31.30\\
99 & 92 & 0.0314 & 58.50 & 98 & 79 & 0.0170 & 42.30 & 97 & 66 & 0.0066 & 30.52\\
99 & 91 & 0.0301 & 57.05 & 98 & 78 & 0.0161 & 41.26 & 97 & 65 & 0.0059 & 29.75\\
99 & 90 & 0.0289 & 55.64 & 98 & 77 & 0.0151 & 40.24 & 97 & 64 & 0.0052 & 29.00\\
99 & 89 & 0.0276 & 54.27 & 98 & 76 & 0.0142 & 39.24 & 97 & 63 & 0.0045 & 28.27\\
99 & 88 & 0.0264 & 52.93 & 98 & 75 & 0.0134 & 38.27 & 97 & 62 & 0.0039 & 27.55\\
99 & 87 & 0.0252 & 51.63 & 98 & 74 & 0.0125 & 37.33 & 97 & 61 & 0.0032 & 26.85\\
99 & 86 & 0.0241 & 50.35 & 98 & 73 & 0.0117 & 36.40 & 97 & 60 & 0.0026 & 26.17\\
99 & 85 & 0.0229 & 49.11 & 98 & 72 & 0.0108 & 35.50 & 97 & 59 & 0.0020 & 25.50\\
99 & 84 & 0.0218 & 47.90 & 98 & 71 & 0.0100 & 34.62 & 97 & 58 & 0.0014 & 24.84\\
99 & 83 & 0.0208 & 46.72 & 98 & 70 & 0.0093 & 33.76 & 97 & 57 & 0.0008 & 24.20\\
99 & 82 & 0.0197 & 45.57 & 98 & 69 & 0.0085 & 32.92 & 97 & 56 & 0.0003 & 23.57\\
99 & 81 & 0.0187 & 44.45 & 98 & 68 & 0.0078 & 32.10 & 96 & 92 & 0.0322 & 58.54\\
\bottomrule
\end{tabular}
\newpage
\begin{tabular}{llll|llll|llll}
 \toprule 
\(T_{db}\) & \(T_{wb}\) & \(\omega\) & \(h\) & \(T_{db}\) & \(T_{wb}\) & \(\omega\) & \(h\) & \(T_{db}\) & \(T_{wb}\) & \(\omega\) & \(h\)  \\ \midrule 
96 & 91 & 0.0309 & 57.09 & 95 & 78 & 0.0168 & 41.29 & 94 & 66 & 0.0072 & 30.54\\
96 & 90 & 0.0296 & 55.68 & 95 & 77 & 0.0158 & 40.27 & 94 & 65 & 0.0065 & 29.77\\
96 & 89 & 0.0283 & 54.31 & 95 & 76 & 0.0149 & 39.27 & 94 & 64 & 0.0059 & 29.02\\
96 & 88 & 0.0271 & 52.97 & 95 & 75 & 0.0141 & 38.30 & 94 & 63 & 0.0052 & 28.29\\
96 & 87 & 0.0259 & 51.67 & 95 & 74 & 0.0132 & 37.36 & 94 & 62 & 0.0045 & 27.57\\
96 & 86 & 0.0248 & 50.39 & 95 & 73 & 0.0124 & 36.43 & 94 & 61 & 0.0039 & 26.87\\
96 & 85 & 0.0237 & 49.15 & 95 & 72 & 0.0115 & 35.53 & 94 & 60 & 0.0033 & 26.18\\
96 & 84 & 0.0226 & 47.94 & 95 & 71 & 0.0107 & 34.65 & 94 & 59 & 0.0027 & 25.51\\
96 & 83 & 0.0215 & 46.76 & 95 & 70 & 0.0100 & 33.78 & 94 & 58 & 0.0021 & 24.86\\
96 & 82 & 0.0204 & 45.61 & 95 & 69 & 0.0092 & 32.94 & 94 & 57 & 0.0015 & 24.22\\
96 & 81 & 0.0194 & 44.49 & 95 & 68 & 0.0084 & 32.12 & 94 & 56 & 0.0009 & 23.59\\
96 & 80 & 0.0184 & 43.39 & 95 & 67 & 0.0077 & 31.32 & 94 & 55 & 0.0004 & 22.97\\
96 & 79 & 0.0175 & 42.32 & 95 & 66 & 0.0070 & 30.53 & 93 & 92 & 0.0329 & 58.58\\
96 & 78 & 0.0165 & 41.28 & 95 & 65 & 0.0063 & 29.77 & 93 & 91 & 0.0316 & 57.14\\
96 & 77 & 0.0156 & 40.26 & 95 & 64 & 0.0056 & 29.02 & 93 & 90 & 0.0303 & 55.73\\
96 & 76 & 0.0147 & 39.26 & 95 & 63 & 0.0050 & 28.28 & 93 & 89 & 0.0291 & 54.35\\
96 & 75 & 0.0138 & 38.29 & 95 & 62 & 0.0043 & 27.57 & 93 & 88 & 0.0278 & 53.01\\
96 & 74 & 0.0130 & 37.35 & 95 & 61 & 0.0037 & 26.86 & 93 & 87 & 0.0267 & 51.71\\
96 & 73 & 0.0121 & 36.42 & 95 & 60 & 0.0031 & 26.18 & 93 & 86 & 0.0255 & 50.43\\
96 & 72 & 0.0113 & 35.52 & 95 & 59 & 0.0025 & 25.51 & 93 & 85 & 0.0244 & 49.19\\
96 & 71 & 0.0105 & 34.64 & 95 & 58 & 0.0019 & 24.85 & 93 & 84 & 0.0233 & 47.98\\
96 & 70 & 0.0097 & 33.78 & 95 & 57 & 0.0013 & 24.21 & 93 & 83 & 0.0222 & 46.80\\
96 & 69 & 0.0090 & 32.93 & 95 & 56 & 0.0007 & 23.58 & 93 & 82 & 0.0212 & 45.64\\
96 & 68 & 0.0082 & 32.11 & 95 & 55 & 0.0002 & 22.97 & 93 & 81 & 0.0201 & 44.52\\
96 & 67 & 0.0075 & 31.31 & 94 & 92 & 0.0327 & 58.57 & 93 & 80 & 0.0191 & 43.42\\
96 & 66 & 0.0068 & 30.52 & 94 & 91 & 0.0313 & 57.12 & 93 & 79 & 0.0182 & 42.35\\
96 & 65 & 0.0061 & 29.76 & 94 & 90 & 0.0301 & 55.71 & 93 & 78 & 0.0172 & 41.31\\
96 & 64 & 0.0054 & 29.01 & 94 & 89 & 0.0288 & 54.34 & 93 & 77 & 0.0163 & 40.29\\
96 & 63 & 0.0047 & 28.27 & 94 & 88 & 0.0276 & 53.00 & 93 & 76 & 0.0154 & 39.30\\
96 & 62 & 0.0041 & 27.56 & 94 & 87 & 0.0264 & 51.69 & 93 & 75 & 0.0145 & 38.32\\
96 & 61 & 0.0035 & 26.86 & 94 & 86 & 0.0253 & 50.42 & 93 & 74 & 0.0137 & 37.38\\
96 & 60 & 0.0028 & 26.17 & 94 & 85 & 0.0241 & 49.18 & 93 & 73 & 0.0128 & 36.45\\
96 & 59 & 0.0022 & 25.50 & 94 & 84 & 0.0230 & 47.97 & 93 & 72 & 0.0120 & 35.55\\
96 & 58 & 0.0016 & 24.85 & 94 & 83 & 0.0220 & 46.78 & 93 & 71 & 0.0112 & 34.66\\
96 & 57 & 0.0011 & 24.20 & 94 & 82 & 0.0209 & 45.63 & 93 & 70 & 0.0104 & 33.80\\
96 & 56 & 0.0005 & 23.58 & 94 & 81 & 0.0199 & 44.51 & 93 & 69 & 0.0097 & 32.96\\
95 & 92 & 0.0324 & 58.55 & 94 & 80 & 0.0189 & 43.41 & 93 & 68 & 0.0089 & 32.14\\
95 & 91 & 0.0311 & 57.11 & 94 & 79 & 0.0179 & 42.34 & 93 & 67 & 0.0082 & 31.33\\
95 & 90 & 0.0298 & 55.70 & 94 & 78 & 0.0170 & 41.30 & 93 & 66 & 0.0075 & 30.55\\
95 & 89 & 0.0286 & 54.32 & 94 & 77 & 0.0161 & 40.28 & 93 & 65 & 0.0068 & 29.78\\
95 & 88 & 0.0274 & 52.98 & 94 & 76 & 0.0152 & 39.28 & 93 & 64 & 0.0061 & 29.03\\
95 & 87 & 0.0262 & 51.68 & 94 & 75 & 0.0143 & 38.31 & 93 & 63 & 0.0054 & 28.30\\
95 & 86 & 0.0250 & 50.41 & 94 & 74 & 0.0134 & 37.37 & 93 & 62 & 0.0048 & 27.58\\
95 & 85 & 0.0239 & 49.16 & 94 & 73 & 0.0126 & 36.44 & 93 & 61 & 0.0041 & 26.88\\
95 & 84 & 0.0228 & 47.95 & 94 & 72 & 0.0118 & 35.54 & 93 & 60 & 0.0035 & 26.19\\
95 & 83 & 0.0217 & 46.77 & 94 & 71 & 0.0110 & 34.66 & 93 & 59 & 0.0029 & 25.52\\
95 & 82 & 0.0207 & 45.62 & 94 & 70 & 0.0102 & 33.79 & 93 & 58 & 0.0023 & 24.86\\
95 & 81 & 0.0197 & 44.50 & 94 & 69 & 0.0094 & 32.95 & 93 & 57 & 0.0017 & 24.22\\
95 & 80 & 0.0187 & 43.40 & 94 & 68 & 0.0087 & 32.13 & 93 & 56 & 0.0012 & 23.59\\
95 & 79 & 0.0177 & 42.33 & 94 & 67 & 0.0079 & 31.33 & 93 & 55 & 0.0006 & 22.98\\
\bottomrule
\end{tabular}
\newpage
\begin{tabular}{llll|llll|llll}
 \toprule 
\(T_{db}\) & \(T_{wb}\) & \(\omega\) & \(h\) & \(T_{db}\) & \(T_{wb}\) & \(\omega\) & \(h\) & \(T_{db}\) & \(T_{wb}\) & \(\omega\) & \(h\)  \\ \midrule 
93 & 54 & 0.0000 & 22.37 & 91 & 81 & 0.0206 & 44.54 & 90 & 68 & 0.0096 & 32.16\\
92 & 92 & 0.0331 & 58.60 & 91 & 80 & 0.0196 & 43.45 & 90 & 67 & 0.0089 & 31.36\\
92 & 91 & 0.0318 & 57.15 & 91 & 79 & 0.0186 & 42.38 & 90 & 66 & 0.0081 & 30.57\\
92 & 90 & 0.0305 & 55.74 & 91 & 78 & 0.0177 & 41.33 & 90 & 65 & 0.0074 & 29.80\\
92 & 89 & 0.0293 & 54.37 & 91 & 77 & 0.0168 & 40.31 & 90 & 64 & 0.0068 & 29.05\\
92 & 88 & 0.0281 & 53.03 & 91 & 76 & 0.0159 & 39.32 & 90 & 63 & 0.0061 & 28.32\\
92 & 87 & 0.0269 & 51.72 & 91 & 75 & 0.0150 & 38.34 & 90 & 62 & 0.0054 & 27.60\\
92 & 86 & 0.0257 & 50.44 & 91 & 74 & 0.0141 & 37.40 & 90 & 61 & 0.0048 & 26.90\\
92 & 85 & 0.0246 & 49.20 & 91 & 73 & 0.0133 & 36.47 & 90 & 60 & 0.0042 & 26.21\\
92 & 84 & 0.0235 & 47.99 & 91 & 72 & 0.0125 & 35.57 & 90 & 59 & 0.0036 & 25.54\\
92 & 83 & 0.0224 & 46.81 & 91 & 71 & 0.0117 & 34.68 & 90 & 58 & 0.0030 & 24.88\\
92 & 82 & 0.0214 & 45.66 & 91 & 70 & 0.0109 & 33.82 & 90 & 57 & 0.0024 & 24.24\\
92 & 81 & 0.0204 & 44.53 & 91 & 69 & 0.0101 & 32.98 & 90 & 56 & 0.0018 & 23.61\\
92 & 80 & 0.0194 & 43.43 & 91 & 68 & 0.0094 & 32.15 & 90 & 55 & 0.0013 & 22.99\\
92 & 79 & 0.0184 & 42.36 & 91 & 67 & 0.0086 & 31.35 & 90 & 54 & 0.0007 & 22.39\\
92 & 78 & 0.0175 & 41.32 & 91 & 66 & 0.0079 & 30.56 & 90 & 53 & 0.0002 & 21.80\\
92 & 77 & 0.0165 & 40.30 & 91 & 65 & 0.0072 & 29.80 & 89 & 89 & 0.0300 & 54.41\\
92 & 76 & 0.0156 & 39.31 & 91 & 64 & 0.0065 & 29.04 & 89 & 88 & 0.0288 & 53.07\\
92 & 75 & 0.0148 & 38.33 & 91 & 63 & 0.0059 & 28.31 & 89 & 87 & 0.0276 & 51.76\\
92 & 74 & 0.0139 & 37.39 & 91 & 62 & 0.0052 & 27.59 & 89 & 86 & 0.0265 & 50.48\\
92 & 73 & 0.0130 & 36.46 & 91 & 61 & 0.0046 & 26.89 & 89 & 85 & 0.0253 & 49.24\\
92 & 72 & 0.0122 & 35.56 & 91 & 60 & 0.0040 & 26.20 & 89 & 84 & 0.0242 & 48.03\\
92 & 71 & 0.0114 & 34.67 & 91 & 59 & 0.0034 & 25.53 & 89 & 83 & 0.0232 & 46.85\\
92 & 70 & 0.0106 & 33.81 & 91 & 58 & 0.0028 & 24.87 & 89 & 82 & 0.0221 & 45.69\\
92 & 69 & 0.0099 & 32.97 & 91 & 57 & 0.0022 & 24.23 & 89 & 81 & 0.0211 & 44.57\\
92 & 68 & 0.0091 & 32.15 & 91 & 56 & 0.0016 & 23.60 & 89 & 80 & 0.0201 & 43.47\\
92 & 67 & 0.0084 & 31.34 & 91 & 55 & 0.0010 & 22.99 & 89 & 79 & 0.0191 & 42.40\\
92 & 66 & 0.0077 & 30.56 & 91 & 54 & 0.0005 & 22.38 & 89 & 78 & 0.0182 & 41.35\\
92 & 65 & 0.0070 & 29.79 & 90 & 90 & 0.0310 & 55.77 & 89 & 77 & 0.0172 & 40.33\\
92 & 64 & 0.0063 & 29.04 & 90 & 89 & 0.0298 & 54.39 & 89 & 76 & 0.0163 & 39.34\\
92 & 63 & 0.0056 & 28.30 & 90 & 88 & 0.0286 & 53.05 & 89 & 75 & 0.0155 & 38.36\\
92 & 62 & 0.0050 & 27.59 & 90 & 87 & 0.0274 & 51.75 & 89 & 74 & 0.0146 & 37.42\\
92 & 61 & 0.0044 & 26.88 & 90 & 86 & 0.0262 & 50.47 & 89 & 73 & 0.0137 & 36.49\\
92 & 60 & 0.0037 & 26.20 & 90 & 85 & 0.0251 & 49.23 & 89 & 72 & 0.0129 & 35.58\\
92 & 59 & 0.0031 & 25.53 & 90 & 84 & 0.0240 & 48.02 & 89 & 71 & 0.0121 & 34.70\\
92 & 58 & 0.0025 & 24.87 & 90 & 83 & 0.0229 & 46.83 & 89 & 70 & 0.0113 & 33.84\\
92 & 57 & 0.0019 & 24.23 & 90 & 82 & 0.0219 & 45.68 & 89 & 69 & 0.0106 & 32.99\\
92 & 56 & 0.0014 & 23.60 & 90 & 81 & 0.0208 & 44.56 & 89 & 68 & 0.0098 & 32.17\\
92 & 55 & 0.0008 & 22.98 & 90 & 80 & 0.0199 & 43.46 & 89 & 67 & 0.0091 & 31.37\\
92 & 54 & 0.0003 & 22.38 & 90 & 79 & 0.0189 & 42.39 & 89 & 66 & 0.0084 & 30.58\\
91 & 91 & 0.0321 & 57.17 & 90 & 78 & 0.0179 & 41.34 & 89 & 65 & 0.0077 & 29.81\\
91 & 90 & 0.0308 & 55.75 & 90 & 77 & 0.0170 & 40.32 & 89 & 64 & 0.0070 & 29.06\\
91 & 89 & 0.0295 & 54.38 & 90 & 76 & 0.0161 & 39.33 & 89 & 63 & 0.0063 & 28.32\\
91 & 88 & 0.0283 & 53.04 & 90 & 75 & 0.0152 & 38.35 & 89 & 62 & 0.0057 & 27.61\\
91 & 87 & 0.0271 & 51.73 & 90 & 74 & 0.0144 & 37.41 & 89 & 61 & 0.0050 & 26.90\\
91 & 86 & 0.0260 & 50.46 & 90 & 73 & 0.0135 & 36.48 & 89 & 60 & 0.0044 & 26.22\\
91 & 85 & 0.0249 & 49.21 & 90 & 72 & 0.0127 & 35.57 & 89 & 59 & 0.0038 & 25.54\\
91 & 84 & 0.0238 & 48.00 & 90 & 71 & 0.0119 & 34.69 & 89 & 58 & 0.0032 & 24.89\\
91 & 83 & 0.0227 & 46.82 & 90 & 70 & 0.0111 & 33.83 & 89 & 57 & 0.0026 & 24.24\\
91 & 82 & 0.0216 & 45.67 & 90 & 69 & 0.0103 & 32.99 & 89 & 56 & 0.0020 & 23.61\\
\bottomrule
\end{tabular}
\newpage
\begin{tabular}{llll|llll|llll}
 \toprule 
\(T_{db}\) & \(T_{wb}\) & \(\omega\) & \(h\) & \(T_{db}\) & \(T_{wb}\) & \(\omega\) & \(h\) & \(T_{db}\) & \(T_{wb}\) & \(\omega\) & \(h\)  \\ \midrule 
89 & 55 & 0.0015 & 23.00 & 87 & 77 & 0.0177 & 40.35 & 86 & 62 & 0.0064 & 27.63\\
89 & 54 & 0.0009 & 22.39 & 87 & 76 & 0.0168 & 39.36 & 86 & 61 & 0.0057 & 26.92\\
89 & 53 & 0.0004 & 21.80 & 87 & 75 & 0.0159 & 38.38 & 86 & 60 & 0.0051 & 26.24\\
88 & 88 & 0.0290 & 53.08 & 87 & 74 & 0.0151 & 37.44 & 86 & 59 & 0.0045 & 25.56\\
88 & 87 & 0.0279 & 51.77 & 87 & 73 & 0.0142 & 36.51 & 86 & 58 & 0.0039 & 24.90\\
88 & 86 & 0.0267 & 50.50 & 87 & 72 & 0.0134 & 35.60 & 86 & 57 & 0.0033 & 24.26\\
88 & 85 & 0.0256 & 49.25 & 87 & 71 & 0.0126 & 34.72 & 86 & 56 & 0.0027 & 23.63\\
88 & 84 & 0.0245 & 48.04 & 87 & 70 & 0.0118 & 33.85 & 86 & 55 & 0.0022 & 23.01\\
88 & 83 & 0.0234 & 46.86 & 87 & 69 & 0.0110 & 33.01 & 86 & 54 & 0.0016 & 22.41\\
88 & 82 & 0.0223 & 45.70 & 87 & 68 & 0.0103 & 32.19 & 86 & 53 & 0.0011 & 21.82\\
88 & 81 & 0.0213 & 44.58 & 87 & 67 & 0.0095 & 31.38 & 86 & 52 & 0.0005 & 21.24\\
88 & 80 & 0.0203 & 43.48 & 87 & 66 & 0.0088 & 30.59 & 86 & 51 & 0.0000 & 20.67\\
88 & 79 & 0.0194 & 42.41 & 87 & 65 & 0.0081 & 29.83 & 85 & 85 & 0.0263 & 49.29\\
88 & 78 & 0.0184 & 41.36 & 87 & 64 & 0.0075 & 29.07 & 85 & 84 & 0.0252 & 48.08\\
88 & 77 & 0.0175 & 40.34 & 87 & 63 & 0.0068 & 28.34 & 85 & 83 & 0.0241 & 46.89\\
88 & 76 & 0.0166 & 39.35 & 87 & 62 & 0.0061 & 27.62 & 85 & 82 & 0.0231 & 45.74\\
88 & 75 & 0.0157 & 38.37 & 87 & 61 & 0.0055 & 26.92 & 85 & 81 & 0.0220 & 44.61\\
88 & 74 & 0.0148 & 37.43 & 87 & 60 & 0.0049 & 26.23 & 85 & 80 & 0.0210 & 43.51\\
88 & 73 & 0.0140 & 36.50 & 87 & 59 & 0.0043 & 25.56 & 85 & 79 & 0.0201 & 42.44\\
88 & 72 & 0.0132 & 35.59 & 87 & 58 & 0.0037 & 24.90 & 85 & 78 & 0.0191 & 41.40\\
88 & 71 & 0.0124 & 34.71 & 87 & 57 & 0.0031 & 24.25 & 85 & 77 & 0.0182 & 40.37\\
88 & 70 & 0.0116 & 33.85 & 87 & 56 & 0.0025 & 23.62 & 85 & 76 & 0.0173 & 39.38\\
88 & 69 & 0.0108 & 33.00 & 87 & 55 & 0.0019 & 23.01 & 85 & 75 & 0.0164 & 38.40\\
88 & 68 & 0.0101 & 32.18 & 87 & 54 & 0.0014 & 22.40 & 85 & 74 & 0.0155 & 37.45\\
88 & 67 & 0.0093 & 31.37 & 87 & 53 & 0.0008 & 21.81 & 85 & 73 & 0.0147 & 36.53\\
88 & 66 & 0.0086 & 30.59 & 87 & 52 & 0.0003 & 21.23 & 85 & 72 & 0.0139 & 35.62\\
88 & 65 & 0.0079 & 29.82 & 86 & 86 & 0.0272 & 50.52 & 85 & 71 & 0.0130 & 34.74\\
88 & 64 & 0.0072 & 29.07 & 86 & 85 & 0.0261 & 49.28 & 85 & 70 & 0.0123 & 33.87\\
88 & 63 & 0.0066 & 28.33 & 86 & 84 & 0.0249 & 48.07 & 85 & 69 & 0.0115 & 33.03\\
88 & 62 & 0.0059 & 27.61 & 86 & 83 & 0.0239 & 46.88 & 85 & 68 & 0.0107 & 32.20\\
88 & 61 & 0.0053 & 26.91 & 86 & 82 & 0.0228 & 45.73 & 85 & 67 & 0.0100 & 31.40\\
88 & 60 & 0.0046 & 26.22 & 86 & 81 & 0.0218 & 44.60 & 85 & 66 & 0.0093 & 30.61\\
88 & 59 & 0.0040 & 25.55 & 86 & 80 & 0.0208 & 43.50 & 85 & 65 & 0.0086 & 29.84\\
88 & 58 & 0.0034 & 24.89 & 86 & 79 & 0.0198 & 42.43 & 85 & 64 & 0.0079 & 29.09\\
88 & 57 & 0.0028 & 24.25 & 86 & 78 & 0.0189 & 41.38 & 85 & 63 & 0.0072 & 28.35\\
88 & 56 & 0.0023 & 23.62 & 86 & 77 & 0.0179 & 40.36 & 85 & 62 & 0.0066 & 27.63\\
88 & 55 & 0.0017 & 23.00 & 86 & 76 & 0.0170 & 39.37 & 85 & 61 & 0.0059 & 26.93\\
88 & 54 & 0.0012 & 22.40 & 86 & 75 & 0.0162 & 38.39 & 85 & 60 & 0.0053 & 26.24\\
88 & 53 & 0.0006 & 21.81 & 86 & 74 & 0.0153 & 37.44 & 85 & 59 & 0.0047 & 25.57\\
88 & 52 & 0.0001 & 21.23 & 86 & 73 & 0.0144 & 36.52 & 85 & 58 & 0.0041 & 24.91\\
87 & 87 & 0.0281 & 51.78 & 86 & 72 & 0.0136 & 35.61 & 85 & 57 & 0.0035 & 24.27\\
87 & 86 & 0.0269 & 50.51 & 86 & 71 & 0.0128 & 34.73 & 85 & 56 & 0.0029 & 23.64\\
87 & 85 & 0.0258 & 49.27 & 86 & 70 & 0.0120 & 33.86 & 85 & 55 & 0.0024 & 23.02\\
87 & 84 & 0.0247 & 48.05 & 86 & 69 & 0.0113 & 33.02 & 85 & 54 & 0.0018 & 22.41\\
87 & 83 & 0.0236 & 46.87 & 86 & 68 & 0.0105 & 32.20 & 85 & 53 & 0.0013 & 21.82\\
87 & 82 & 0.0226 & 45.72 & 86 & 67 & 0.0098 & 31.39 & 85 & 52 & 0.0008 & 21.24\\
87 & 81 & 0.0216 & 44.59 & 86 & 66 & 0.0091 & 30.60 & 85 & 51 & 0.0002 & 20.67\\
87 & 80 & 0.0206 & 43.49 & 86 & 65 & 0.0084 & 29.83 & 84 & 84 & 0.0254 & 48.09\\
87 & 79 & 0.0196 & 42.42 & 86 & 64 & 0.0077 & 29.08 & 84 & 83 & 0.0244 & 46.91\\
87 & 78 & 0.0186 & 41.37 & 86 & 63 & 0.0070 & 28.35 & 84 & 82 & 0.0233 & 45.75\\
\bottomrule
\end{tabular}
\newpage
\begin{tabular}{llll|llll|llll}
 \toprule 
\(T_{db}\) & \(T_{wb}\) & \(\omega\) & \(h\) & \(T_{db}\) & \(T_{wb}\) & \(\omega\) & \(h\) & \(T_{db}\) & \(T_{wb}\) & \(\omega\) & \(h\)  \\ \midrule 
84 & 81 & 0.0223 & 44.62 & 83 & 64 & 0.0084 & 29.10 & 81 & 79 & 0.0210 & 42.49\\
84 & 80 & 0.0213 & 43.53 & 83 & 63 & 0.0077 & 28.37 & 81 & 78 & 0.0201 & 41.44\\
84 & 79 & 0.0203 & 42.45 & 83 & 62 & 0.0070 & 27.65 & 81 & 77 & 0.0191 & 40.42\\
84 & 78 & 0.0193 & 41.41 & 83 & 61 & 0.0064 & 26.94 & 81 & 76 & 0.0182 & 39.42\\
84 & 77 & 0.0184 & 40.39 & 83 & 60 & 0.0058 & 26.25 & 81 & 75 & 0.0173 & 38.44\\
84 & 76 & 0.0175 & 39.39 & 83 & 59 & 0.0052 & 25.58 & 81 & 74 & 0.0165 & 37.49\\
84 & 75 & 0.0166 & 38.41 & 83 & 58 & 0.0046 & 24.92 & 81 & 73 & 0.0156 & 36.57\\
84 & 74 & 0.0158 & 37.46 & 83 & 57 & 0.0040 & 24.28 & 81 & 72 & 0.0148 & 35.66\\
84 & 73 & 0.0149 & 36.54 & 83 & 56 & 0.0034 & 23.65 & 81 & 71 & 0.0140 & 34.77\\
84 & 72 & 0.0141 & 35.63 & 83 & 55 & 0.0028 & 23.03 & 81 & 70 & 0.0132 & 33.91\\
84 & 71 & 0.0133 & 34.75 & 83 & 54 & 0.0023 & 22.42 & 81 & 69 & 0.0124 & 33.06\\
84 & 70 & 0.0125 & 33.88 & 83 & 53 & 0.0017 & 21.83 & 81 & 68 & 0.0117 & 32.24\\
84 & 69 & 0.0117 & 33.04 & 83 & 52 & 0.0012 & 21.25 & 81 & 67 & 0.0109 & 31.43\\
84 & 68 & 0.0110 & 32.21 & 83 & 51 & 0.0007 & 20.68 & 81 & 66 & 0.0102 & 30.64\\
84 & 67 & 0.0102 & 31.41 & 83 & 50 & 0.0002 & 20.13 & 81 & 65 & 0.0095 & 29.87\\
84 & 66 & 0.0095 & 30.62 & 82 & 82 & 0.0238 & 45.78 & 81 & 64 & 0.0088 & 29.12\\
84 & 65 & 0.0088 & 29.85 & 82 & 81 & 0.0228 & 44.65 & 81 & 63 & 0.0081 & 28.38\\
84 & 64 & 0.0081 & 29.10 & 82 & 80 & 0.0217 & 43.55 & 81 & 62 & 0.0075 & 27.66\\
84 & 63 & 0.0075 & 28.36 & 82 & 79 & 0.0208 & 42.48 & 81 & 61 & 0.0069 & 26.96\\
84 & 62 & 0.0068 & 27.64 & 82 & 78 & 0.0198 & 41.43 & 81 & 60 & 0.0062 & 26.27\\
84 & 61 & 0.0062 & 26.94 & 82 & 77 & 0.0189 & 40.41 & 81 & 59 & 0.0056 & 25.59\\
84 & 60 & 0.0055 & 26.25 & 82 & 76 & 0.0180 & 39.41 & 81 & 58 & 0.0050 & 24.93\\
84 & 59 & 0.0049 & 25.57 & 82 & 75 & 0.0171 & 38.43 & 81 & 57 & 0.0044 & 24.29\\
84 & 58 & 0.0043 & 24.92 & 82 & 74 & 0.0162 & 37.48 & 81 & 56 & 0.0038 & 23.66\\
84 & 57 & 0.0037 & 24.27 & 82 & 73 & 0.0154 & 36.56 & 81 & 55 & 0.0033 & 23.04\\
84 & 56 & 0.0032 & 23.64 & 82 & 72 & 0.0146 & 35.65 & 81 & 54 & 0.0027 & 22.43\\
84 & 55 & 0.0026 & 23.02 & 82 & 71 & 0.0137 & 34.76 & 81 & 53 & 0.0022 & 21.84\\
84 & 54 & 0.0021 & 22.42 & 82 & 70 & 0.0130 & 33.90 & 81 & 52 & 0.0017 & 21.26\\
84 & 53 & 0.0015 & 21.83 & 82 & 69 & 0.0122 & 33.05 & 81 & 51 & 0.0011 & 20.69\\
84 & 52 & 0.0010 & 21.25 & 82 & 68 & 0.0114 & 32.23 & 81 & 50 & 0.0006 & 20.13\\
84 & 51 & 0.0005 & 20.68 & 82 & 67 & 0.0107 & 31.42 & 81 & 49 & 0.0001 & 19.59\\
83 & 83 & 0.0246 & 46.92 & 82 & 66 & 0.0100 & 30.63 & 80 & 80 & 0.0222 & 43.57\\
83 & 82 & 0.0235 & 45.76 & 82 & 65 & 0.0093 & 29.86 & 80 & 79 & 0.0212 & 42.50\\
83 & 81 & 0.0225 & 44.64 & 82 & 64 & 0.0086 & 29.11 & 80 & 78 & 0.0203 & 41.45\\
83 & 80 & 0.0215 & 43.54 & 82 & 63 & 0.0079 & 28.37 & 80 & 77 & 0.0194 & 40.43\\
83 & 79 & 0.0205 & 42.46 & 82 & 62 & 0.0073 & 27.65 & 80 & 76 & 0.0184 & 39.43\\
83 & 78 & 0.0196 & 41.42 & 82 & 61 & 0.0066 & 26.95 & 80 & 75 & 0.0176 & 38.46\\
83 & 77 & 0.0186 & 40.40 & 82 & 60 & 0.0060 & 26.26 & 80 & 74 & 0.0167 & 37.50\\
83 & 76 & 0.0177 & 39.40 & 82 & 59 & 0.0054 & 25.59 & 80 & 73 & 0.0158 & 36.57\\
83 & 75 & 0.0169 & 38.42 & 82 & 58 & 0.0048 & 24.93 & 80 & 72 & 0.0150 & 35.67\\
83 & 74 & 0.0160 & 37.47 & 82 & 57 & 0.0042 & 24.28 & 80 & 71 & 0.0142 & 34.78\\
83 & 73 & 0.0151 & 36.55 & 82 & 56 & 0.0036 & 23.65 & 80 & 70 & 0.0134 & 33.92\\
83 & 72 & 0.0143 & 35.64 & 82 & 55 & 0.0031 & 23.03 & 80 & 69 & 0.0126 & 33.07\\
83 & 71 & 0.0135 & 34.75 & 82 & 54 & 0.0025 & 22.43 & 80 & 68 & 0.0119 & 32.25\\
83 & 70 & 0.0127 & 33.89 & 82 & 53 & 0.0020 & 21.84 & 80 & 67 & 0.0112 & 31.44\\
83 & 69 & 0.0120 & 33.05 & 82 & 52 & 0.0014 & 21.26 & 80 & 66 & 0.0104 & 30.65\\
83 & 68 & 0.0112 & 32.22 & 82 & 51 & 0.0009 & 20.69 & 80 & 65 & 0.0097 & 29.88\\
83 & 67 & 0.0105 & 31.41 & 82 & 50 & 0.0004 & 20.13 & 80 & 64 & 0.0091 & 29.12\\
83 & 66 & 0.0098 & 30.63 & 81 & 81 & 0.0230 & 44.66 & 80 & 63 & 0.0084 & 28.39\\
83 & 65 & 0.0090 & 29.86 & 81 & 80 & 0.0220 & 43.56 & 80 & 62 & 0.0077 & 27.67\\
\bottomrule
\end{tabular}
\newpage
\begin{tabular}{llll|llll|llll}
 \toprule 
\(T_{db}\) & \(T_{wb}\) & \(\omega\) & \(h\) & \(T_{db}\) & \(T_{wb}\) & \(\omega\) & \(h\) & \(T_{db}\) & \(T_{wb}\) & \(\omega\) & \(h\)  \\ \midrule 
80 & 61 & 0.0071 & 26.96 & 78 & 73 & 0.0163 & 36.59 & 77 & 53 & 0.0031 & 21.86\\
80 & 60 & 0.0065 & 26.27 & 78 & 72 & 0.0155 & 35.69 & 77 & 52 & 0.0026 & 21.28\\
80 & 59 & 0.0058 & 25.60 & 78 & 71 & 0.0147 & 34.80 & 77 & 51 & 0.0020 & 20.71\\
80 & 58 & 0.0052 & 24.94 & 78 & 70 & 0.0139 & 33.93 & 77 & 50 & 0.0015 & 20.15\\
80 & 57 & 0.0046 & 24.29 & 78 & 69 & 0.0131 & 33.09 & 77 & 49 & 0.0010 & 19.60\\
80 & 56 & 0.0041 & 23.66 & 78 & 68 & 0.0124 & 32.26 & 77 & 48 & 0.0005 & 19.07\\
80 & 55 & 0.0035 & 23.04 & 78 & 67 & 0.0116 & 31.45 & 77 & 47 & 0.0001 & 18.54\\
80 & 54 & 0.0030 & 22.44 & 78 & 66 & 0.0109 & 30.67 & 76 & 76 & 0.0194 & 39.47\\
80 & 53 & 0.0024 & 21.85 & 78 & 65 & 0.0102 & 29.89 & 76 & 75 & 0.0185 & 38.50\\
80 & 52 & 0.0019 & 21.26 & 78 & 64 & 0.0095 & 29.14 & 76 & 74 & 0.0176 & 37.54\\
80 & 51 & 0.0014 & 20.70 & 78 & 63 & 0.0088 & 28.40 & 76 & 73 & 0.0168 & 36.61\\
80 & 50 & 0.0009 & 20.14 & 78 & 62 & 0.0082 & 27.68 & 76 & 72 & 0.0160 & 35.71\\
80 & 49 & 0.0004 & 19.59 & 78 & 61 & 0.0075 & 26.98 & 76 & 71 & 0.0151 & 34.82\\
79 & 79 & 0.0215 & 42.51 & 78 & 60 & 0.0069 & 26.29 & 76 & 70 & 0.0144 & 33.95\\
79 & 78 & 0.0205 & 41.46 & 78 & 59 & 0.0063 & 25.61 & 76 & 69 & 0.0136 & 33.11\\
79 & 77 & 0.0196 & 40.44 & 78 & 58 & 0.0057 & 24.95 & 76 & 68 & 0.0128 & 32.28\\
79 & 76 & 0.0187 & 39.44 & 78 & 57 & 0.0051 & 24.31 & 76 & 67 & 0.0121 & 31.47\\
79 & 75 & 0.0178 & 38.47 & 78 & 56 & 0.0045 & 23.67 & 76 & 66 & 0.0114 & 30.68\\
79 & 74 & 0.0169 & 37.51 & 78 & 55 & 0.0040 & 23.05 & 76 & 65 & 0.0107 & 29.91\\
79 & 73 & 0.0161 & 36.58 & 78 & 54 & 0.0034 & 22.45 & 76 & 64 & 0.0100 & 29.15\\
79 & 72 & 0.0153 & 35.68 & 78 & 53 & 0.0029 & 21.85 & 76 & 63 & 0.0093 & 28.42\\
79 & 71 & 0.0144 & 34.79 & 78 & 52 & 0.0023 & 21.27 & 76 & 62 & 0.0086 & 27.69\\
79 & 70 & 0.0137 & 33.93 & 78 & 51 & 0.0018 & 20.70 & 76 & 61 & 0.0080 & 26.99\\
79 & 69 & 0.0129 & 33.08 & 78 & 50 & 0.0013 & 20.15 & 76 & 60 & 0.0074 & 26.30\\
79 & 68 & 0.0121 & 32.25 & 78 & 49 & 0.0008 & 19.60 & 76 & 59 & 0.0067 & 25.62\\
79 & 67 & 0.0114 & 31.45 & 78 & 48 & 0.0003 & 19.06 & 76 & 58 & 0.0061 & 24.96\\
79 & 66 & 0.0107 & 30.66 & 77 & 77 & 0.0201 & 40.46 & 76 & 57 & 0.0056 & 24.32\\
79 & 65 & 0.0100 & 29.89 & 77 & 76 & 0.0192 & 39.46 & 76 & 56 & 0.0050 & 23.68\\
79 & 64 & 0.0093 & 29.13 & 77 & 75 & 0.0183 & 38.49 & 76 & 55 & 0.0044 & 23.06\\
79 & 63 & 0.0086 & 28.40 & 77 & 74 & 0.0174 & 37.53 & 76 & 54 & 0.0039 & 22.46\\
79 & 62 & 0.0079 & 27.67 & 77 & 73 & 0.0165 & 36.60 & 76 & 53 & 0.0033 & 21.86\\
79 & 61 & 0.0073 & 26.97 & 77 & 72 & 0.0157 & 35.70 & 76 & 52 & 0.0028 & 21.28\\
79 & 60 & 0.0067 & 26.28 & 77 & 71 & 0.0149 & 34.81 & 76 & 51 & 0.0023 & 20.71\\
79 & 59 & 0.0061 & 25.61 & 77 & 70 & 0.0141 & 33.94 & 76 & 50 & 0.0017 & 20.15\\
79 & 58 & 0.0055 & 24.95 & 77 & 69 & 0.0133 & 33.10 & 76 & 49 & 0.0012 & 19.61\\
79 & 57 & 0.0049 & 24.30 & 77 & 68 & 0.0126 & 32.27 & 76 & 48 & 0.0008 & 19.07\\
79 & 56 & 0.0043 & 23.67 & 77 & 67 & 0.0119 & 31.46 & 76 & 47 & 0.0003 & 18.55\\
79 & 55 & 0.0037 & 23.05 & 77 & 66 & 0.0111 & 30.67 & 75 & 75 & 0.0187 & 38.51\\
79 & 54 & 0.0032 & 22.44 & 77 & 65 & 0.0104 & 29.90 & 75 & 74 & 0.0179 & 37.55\\
79 & 53 & 0.0026 & 21.85 & 77 & 64 & 0.0097 & 29.15 & 75 & 73 & 0.0170 & 36.62\\
79 & 52 & 0.0021 & 21.27 & 77 & 63 & 0.0091 & 28.41 & 75 & 72 & 0.0162 & 35.71\\
79 & 51 & 0.0016 & 20.70 & 77 & 62 & 0.0084 & 27.69 & 75 & 71 & 0.0154 & 34.83\\
79 & 50 & 0.0011 & 20.14 & 77 & 61 & 0.0078 & 26.98 & 75 & 70 & 0.0146 & 33.96\\
79 & 49 & 0.0006 & 19.60 & 77 & 60 & 0.0071 & 26.29 & 75 & 69 & 0.0138 & 33.11\\
79 & 48 & 0.0001 & 19.06 & 77 & 59 & 0.0065 & 25.62 & 75 & 68 & 0.0131 & 32.29\\
78 & 78 & 0.0208 & 41.47 & 77 & 58 & 0.0059 & 24.96 & 75 & 67 & 0.0123 & 31.48\\
78 & 77 & 0.0198 & 40.45 & 77 & 57 & 0.0053 & 24.31 & 75 & 66 & 0.0116 & 30.69\\
78 & 76 & 0.0189 & 39.45 & 77 & 56 & 0.0047 & 23.68 & 75 & 65 & 0.0109 & 29.92\\
78 & 75 & 0.0180 & 38.48 & 77 & 55 & 0.0042 & 23.06 & 75 & 64 & 0.0102 & 29.16\\
78 & 74 & 0.0172 & 37.52 & 77 & 54 & 0.0036 & 22.45 & 75 & 63 & 0.0095 & 28.42\\
\bottomrule
\end{tabular}
\newpage
\begin{tabular}{llll|llll|llll}
 \toprule 
\(T_{db}\) & \(T_{wb}\) & \(\omega\) & \(h\) & \(T_{db}\) & \(T_{wb}\) & \(\omega\) & \(h\) & \(T_{db}\) & \(T_{wb}\) & \(\omega\) & \(h\)  \\ \midrule 
75 & 62 & 0.0089 & 27.70 & 73 & 69 & 0.0143 & 33.13 & 72 & 47 & 0.0012 & 18.56\\
75 & 61 & 0.0082 & 27.00 & 73 & 68 & 0.0135 & 32.30 & 72 & 46 & 0.0007 & 18.04\\
75 & 60 & 0.0076 & 26.31 & 73 & 67 & 0.0128 & 31.49 & 72 & 45 & 0.0002 & 17.54\\
75 & 59 & 0.0070 & 25.63 & 73 & 66 & 0.0121 & 30.70 & 71 & 71 & 0.0163 & 34.86\\
75 & 58 & 0.0064 & 24.97 & 73 & 65 & 0.0114 & 29.93 & 71 & 70 & 0.0155 & 34.00\\
75 & 57 & 0.0058 & 24.32 & 73 & 64 & 0.0107 & 29.18 & 71 & 69 & 0.0147 & 33.15\\
75 & 56 & 0.0052 & 23.69 & 73 & 63 & 0.0100 & 28.44 & 71 & 68 & 0.0140 & 32.32\\
75 & 55 & 0.0046 & 23.07 & 73 & 62 & 0.0093 & 27.72 & 71 & 67 & 0.0132 & 31.51\\
75 & 54 & 0.0041 & 22.46 & 73 & 61 & 0.0087 & 27.01 & 71 & 66 & 0.0125 & 30.72\\
75 & 53 & 0.0035 & 21.87 & 73 & 60 & 0.0080 & 26.32 & 71 & 65 & 0.0118 & 29.95\\
75 & 52 & 0.0030 & 21.29 & 73 & 59 & 0.0074 & 25.64 & 71 & 64 & 0.0111 & 29.19\\
75 & 51 & 0.0025 & 20.72 & 73 & 58 & 0.0068 & 24.98 & 71 & 63 & 0.0104 & 28.45\\
75 & 50 & 0.0020 & 20.16 & 73 & 57 & 0.0062 & 24.33 & 71 & 62 & 0.0098 & 27.73\\
75 & 49 & 0.0015 & 19.61 & 73 & 56 & 0.0057 & 23.70 & 71 & 61 & 0.0091 & 27.02\\
75 & 48 & 0.0010 & 19.07 & 73 & 55 & 0.0051 & 23.08 & 71 & 60 & 0.0085 & 26.33\\
75 & 47 & 0.0005 & 18.55 & 73 & 54 & 0.0045 & 22.47 & 71 & 59 & 0.0079 & 25.65\\
75 & 46 & 0.0000 & 18.03 & 73 & 53 & 0.0040 & 21.88 & 71 & 58 & 0.0073 & 24.99\\
74 & 74 & 0.0181 & 37.56 & 73 & 52 & 0.0035 & 21.30 & 71 & 57 & 0.0067 & 24.35\\
74 & 73 & 0.0173 & 36.63 & 73 & 51 & 0.0029 & 20.73 & 71 & 56 & 0.0061 & 23.71\\
74 & 72 & 0.0164 & 35.72 & 73 & 50 & 0.0024 & 20.17 & 71 & 55 & 0.0055 & 23.09\\
74 & 71 & 0.0156 & 34.84 & 73 & 49 & 0.0019 & 19.62 & 71 & 54 & 0.0050 & 22.48\\
74 & 70 & 0.0148 & 33.97 & 73 & 48 & 0.0014 & 19.08 & 71 & 53 & 0.0044 & 21.89\\
74 & 69 & 0.0140 & 33.12 & 73 & 47 & 0.0009 & 18.56 & 71 & 52 & 0.0039 & 21.30\\
74 & 68 & 0.0133 & 32.30 & 73 & 46 & 0.0005 & 18.04 & 71 & 51 & 0.0034 & 20.73\\
74 & 67 & 0.0125 & 31.49 & 73 & 45 & 0.0000 & 17.53 & 71 & 50 & 0.0029 & 20.17\\
74 & 66 & 0.0118 & 30.70 & 72 & 72 & 0.0169 & 35.74 & 71 & 49 & 0.0024 & 19.63\\
74 & 65 & 0.0111 & 29.92 & 72 & 71 & 0.0161 & 34.85 & 71 & 48 & 0.0019 & 19.09\\
74 & 64 & 0.0104 & 29.17 & 72 & 70 & 0.0153 & 33.99 & 71 & 47 & 0.0014 & 18.56\\
74 & 63 & 0.0098 & 28.43 & 72 & 69 & 0.0145 & 33.14 & 71 & 46 & 0.0009 & 18.05\\
74 & 62 & 0.0091 & 27.71 & 72 & 68 & 0.0138 & 32.31 & 71 & 45 & 0.0005 & 17.54\\
74 & 61 & 0.0084 & 27.00 & 72 & 67 & 0.0130 & 31.50 & 71 & 44 & 0.0000 & 17.04\\
74 & 60 & 0.0078 & 26.31 & 72 & 66 & 0.0123 & 30.71 & 70 & 70 & 0.0158 & 34.01\\
74 & 59 & 0.0072 & 25.64 & 72 & 65 & 0.0116 & 29.94 & 70 & 69 & 0.0150 & 33.16\\
74 & 58 & 0.0066 & 24.98 & 72 & 64 & 0.0109 & 29.18 & 70 & 68 & 0.0142 & 32.33\\
74 & 57 & 0.0060 & 24.33 & 72 & 63 & 0.0102 & 28.44 & 70 & 67 & 0.0135 & 31.52\\
74 & 56 & 0.0054 & 23.69 & 72 & 62 & 0.0096 & 27.72 & 70 & 66 & 0.0128 & 30.73\\
74 & 55 & 0.0049 & 23.07 & 72 & 61 & 0.0089 & 27.02 & 70 & 65 & 0.0120 & 29.95\\
74 & 54 & 0.0043 & 22.47 & 72 & 60 & 0.0083 & 26.32 & 70 & 64 & 0.0114 & 29.20\\
74 & 53 & 0.0038 & 21.87 & 72 & 59 & 0.0077 & 25.65 & 70 & 63 & 0.0107 & 28.46\\
74 & 52 & 0.0032 & 21.29 & 72 & 58 & 0.0071 & 24.99 & 70 & 62 & 0.0100 & 27.74\\
74 & 51 & 0.0027 & 20.72 & 72 & 57 & 0.0065 & 24.34 & 70 & 61 & 0.0094 & 27.03\\
74 & 50 & 0.0022 & 20.16 & 72 & 56 & 0.0059 & 23.71 & 70 & 60 & 0.0087 & 26.34\\
74 & 49 & 0.0017 & 19.61 & 72 & 55 & 0.0053 & 23.09 & 70 & 59 & 0.0081 & 25.66\\
74 & 48 & 0.0012 & 19.08 & 72 & 54 & 0.0048 & 22.48 & 70 & 58 & 0.0075 & 25.00\\
74 & 47 & 0.0007 & 18.55 & 72 & 53 & 0.0042 & 21.88 & 70 & 57 & 0.0069 & 24.35\\
74 & 46 & 0.0003 & 18.04 & 72 & 52 & 0.0037 & 21.30 & 70 & 56 & 0.0063 & 23.72\\
73 & 73 & 0.0175 & 36.64 & 72 & 51 & 0.0032 & 20.73 & 70 & 55 & 0.0058 & 23.10\\
73 & 72 & 0.0167 & 35.73 & 72 & 50 & 0.0026 & 20.17 & 70 & 54 & 0.0052 & 22.49\\
73 & 71 & 0.0158 & 34.85 & 72 & 49 & 0.0021 & 19.62 & 70 & 53 & 0.0047 & 21.89\\
73 & 70 & 0.0151 & 33.98 & 72 & 48 & 0.0017 & 19.09 & 70 & 52 & 0.0041 & 21.31\\
\bottomrule
\end{tabular}
\newpage
\begin{tabular}{llll|llll|llll}
 \toprule 
\(T_{db}\) & \(T_{wb}\) & \(\omega\) & \(h\) & \(T_{db}\) & \(T_{wb}\) & \(\omega\) & \(h\) & \(T_{db}\) & \(T_{wb}\) & \(\omega\) & \(h\)  \\ \midrule 
70 & 51 & 0.0036 & 20.74 & 68 & 52 & 0.0046 & 21.32 & 66 & 52 & 0.0050 & 21.33\\
70 & 50 & 0.0031 & 20.18 & 68 & 51 & 0.0041 & 20.75 & 66 & 51 & 0.0045 & 20.76\\
70 & 49 & 0.0026 & 19.63 & 68 & 50 & 0.0035 & 20.19 & 66 & 50 & 0.0040 & 20.19\\
70 & 48 & 0.0021 & 19.09 & 68 & 49 & 0.0030 & 19.64 & 66 & 49 & 0.0035 & 19.65\\
70 & 47 & 0.0016 & 18.57 & 68 & 48 & 0.0025 & 19.10 & 66 & 48 & 0.0030 & 19.11\\
70 & 46 & 0.0011 & 18.05 & 68 & 47 & 0.0021 & 18.57 & 66 & 47 & 0.0025 & 18.58\\
70 & 45 & 0.0007 & 17.54 & 68 & 46 & 0.0016 & 18.05 & 66 & 46 & 0.0020 & 18.06\\
70 & 44 & 0.0002 & 17.04 & 68 & 45 & 0.0011 & 17.55 & 66 & 45 & 0.0016 & 17.55\\
69 & 69 & 0.0152 & 33.17 & 68 & 44 & 0.0007 & 17.05 & 66 & 44 & 0.0011 & 17.05\\
69 & 68 & 0.0145 & 32.34 & 68 & 43 & 0.0002 & 16.56 & 66 & 43 & 0.0007 & 16.57\\
69 & 67 & 0.0137 & 31.53 & 67 & 67 & 0.0142 & 31.54 & 66 & 42 & 0.0002 & 16.09\\
69 & 66 & 0.0130 & 30.74 & 67 & 66 & 0.0135 & 30.75 & 65 & 65 & 0.0132 & 29.99\\
69 & 65 & 0.0123 & 29.96 & 67 & 65 & 0.0127 & 29.98 & 65 & 64 & 0.0125 & 29.24\\
69 & 64 & 0.0116 & 29.21 & 67 & 64 & 0.0120 & 29.22 & 65 & 63 & 0.0118 & 28.50\\
69 & 63 & 0.0109 & 28.47 & 67 & 63 & 0.0114 & 28.48 & 65 & 62 & 0.0112 & 27.77\\
69 & 62 & 0.0102 & 27.74 & 67 & 62 & 0.0107 & 27.76 & 65 & 61 & 0.0105 & 27.06\\
69 & 61 & 0.0096 & 27.04 & 67 & 61 & 0.0101 & 27.05 & 65 & 60 & 0.0099 & 26.37\\
69 & 60 & 0.0090 & 26.34 & 67 & 60 & 0.0094 & 26.36 & 65 & 59 & 0.0093 & 25.69\\
69 & 59 & 0.0083 & 25.67 & 67 & 59 & 0.0088 & 25.68 & 65 & 58 & 0.0087 & 25.03\\
69 & 58 & 0.0077 & 25.00 & 67 & 58 & 0.0082 & 25.02 & 65 & 57 & 0.0081 & 24.38\\
69 & 57 & 0.0071 & 24.36 & 67 & 57 & 0.0076 & 24.37 & 65 & 56 & 0.0075 & 23.74\\
69 & 56 & 0.0066 & 23.72 & 67 & 56 & 0.0070 & 23.73 & 65 & 55 & 0.0069 & 23.12\\
69 & 55 & 0.0060 & 23.10 & 67 & 55 & 0.0064 & 23.11 & 65 & 54 & 0.0063 & 22.51\\
69 & 54 & 0.0054 & 22.49 & 67 & 54 & 0.0059 & 22.50 & 65 & 53 & 0.0058 & 21.92\\
69 & 53 & 0.0049 & 21.90 & 67 & 53 & 0.0053 & 21.91 & 65 & 52 & 0.0053 & 21.33\\
69 & 52 & 0.0044 & 21.31 & 67 & 52 & 0.0048 & 21.32 & 65 & 51 & 0.0047 & 20.76\\
69 & 51 & 0.0038 & 20.74 & 67 & 51 & 0.0043 & 20.75 & 65 & 50 & 0.0042 & 20.20\\
69 & 50 & 0.0033 & 20.18 & 67 & 50 & 0.0038 & 20.19 & 65 & 49 & 0.0037 & 19.65\\
69 & 49 & 0.0028 & 19.63 & 67 & 49 & 0.0033 & 19.64 & 65 & 48 & 0.0032 & 19.11\\
69 & 48 & 0.0023 & 19.10 & 67 & 48 & 0.0028 & 19.10 & 65 & 47 & 0.0027 & 18.58\\
69 & 47 & 0.0018 & 18.57 & 67 & 47 & 0.0023 & 18.58 & 65 & 46 & 0.0023 & 18.06\\
69 & 46 & 0.0014 & 18.05 & 67 & 46 & 0.0018 & 18.06 & 65 & 45 & 0.0018 & 17.56\\
69 & 45 & 0.0009 & 17.54 & 67 & 45 & 0.0013 & 17.55 & 65 & 44 & 0.0013 & 17.06\\
69 & 44 & 0.0004 & 17.05 & 67 & 44 & 0.0009 & 17.05 & 65 & 43 & 0.0009 & 16.57\\
68 & 68 & 0.0147 & 32.35 & 67 & 43 & 0.0004 & 16.56 & 65 & 42 & 0.0004 & 16.09\\
68 & 67 & 0.0139 & 31.54 & 67 & 42 & 0.0000 & 16.08 & 65 & 41 & 0.0000 & 15.62\\
68 & 66 & 0.0132 & 30.74 & 66 & 66 & 0.0137 & 30.76 & 64 & 64 & 0.0127 & 29.24\\
68 & 65 & 0.0125 & 29.97 & 66 & 65 & 0.0130 & 29.99 & 64 & 63 & 0.0121 & 28.50\\
68 & 64 & 0.0118 & 29.21 & 66 & 64 & 0.0123 & 29.23 & 64 & 62 & 0.0114 & 27.78\\
68 & 63 & 0.0111 & 28.47 & 66 & 63 & 0.0116 & 28.49 & 64 & 61 & 0.0107 & 27.07\\
68 & 62 & 0.0105 & 27.75 & 66 & 62 & 0.0109 & 27.76 & 64 & 60 & 0.0101 & 26.38\\
68 & 61 & 0.0098 & 27.04 & 66 & 61 & 0.0103 & 27.06 & 64 & 59 & 0.0095 & 25.70\\
68 & 60 & 0.0092 & 26.35 & 66 & 60 & 0.0097 & 26.36 & 64 & 58 & 0.0089 & 25.03\\
68 & 59 & 0.0086 & 25.67 & 66 & 59 & 0.0090 & 25.69 & 64 & 57 & 0.0083 & 24.39\\
68 & 58 & 0.0080 & 25.01 & 66 & 58 & 0.0084 & 25.02 & 64 & 56 & 0.0077 & 23.75\\
68 & 57 & 0.0074 & 24.36 & 66 & 57 & 0.0078 & 24.37 & 64 & 55 & 0.0071 & 23.13\\
68 & 56 & 0.0068 & 23.73 & 66 & 56 & 0.0072 & 23.74 & 64 & 54 & 0.0066 & 22.52\\
68 & 55 & 0.0062 & 23.11 & 66 & 55 & 0.0067 & 23.12 & 64 & 53 & 0.0060 & 21.92\\
68 & 54 & 0.0057 & 22.50 & 66 & 54 & 0.0061 & 22.51 & 64 & 52 & 0.0055 & 21.34\\
68 & 53 & 0.0051 & 21.90 & 66 & 53 & 0.0056 & 21.91 & 64 & 51 & 0.0050 & 20.76\\
\bottomrule
\end{tabular}
\newpage
\begin{tabular}{llll|llll|llll}
 \toprule 
\(T_{db}\) & \(T_{wb}\) & \(\omega\) & \(h\) & \(T_{db}\) & \(T_{wb}\) & \(\omega\) & \(h\) & \(T_{db}\) & \(T_{wb}\) & \(\omega\) & \(h\)  \\ \midrule 
64 & 50 & 0.0044 & 20.20 & 62 & 46 & 0.0029 & 18.07 & 60 & 40 & 0.0007 & 15.16\\
64 & 49 & 0.0039 & 19.65 & 62 & 45 & 0.0025 & 17.56 & 60 & 39 & 0.0003 & 14.71\\
64 & 48 & 0.0034 & 19.11 & 62 & 44 & 0.0020 & 17.07 & 59 & 59 & 0.0106 & 25.73\\
64 & 47 & 0.0030 & 18.59 & 62 & 43 & 0.0016 & 16.58 & 59 & 58 & 0.0100 & 25.06\\
64 & 46 & 0.0025 & 18.07 & 62 & 42 & 0.0011 & 16.10 & 59 & 57 & 0.0094 & 24.41\\
64 & 45 & 0.0020 & 17.56 & 62 & 41 & 0.0007 & 15.62 & 59 & 56 & 0.0088 & 23.78\\
64 & 44 & 0.0016 & 17.06 & 62 & 40 & 0.0003 & 15.16 & 59 & 55 & 0.0083 & 23.15\\
64 & 43 & 0.0011 & 16.57 & 61 & 61 & 0.0114 & 27.09 & 59 & 54 & 0.0077 & 22.54\\
64 & 42 & 0.0007 & 16.09 & 61 & 60 & 0.0108 & 26.40 & 59 & 53 & 0.0072 & 21.94\\
64 & 41 & 0.0002 & 15.62 & 61 & 59 & 0.0102 & 25.72 & 59 & 52 & 0.0066 & 21.36\\
63 & 63 & 0.0123 & 28.51 & 61 & 58 & 0.0096 & 25.05 & 59 & 51 & 0.0061 & 20.79\\
63 & 62 & 0.0116 & 27.78 & 61 & 57 & 0.0090 & 24.40 & 59 & 50 & 0.0056 & 20.22\\
63 & 61 & 0.0110 & 27.08 & 61 & 56 & 0.0084 & 23.77 & 59 & 49 & 0.0051 & 19.67\\
63 & 60 & 0.0103 & 26.38 & 61 & 55 & 0.0078 & 23.14 & 59 & 48 & 0.0046 & 19.13\\
63 & 59 & 0.0097 & 25.70 & 61 & 54 & 0.0073 & 22.53 & 59 & 47 & 0.0041 & 18.60\\
63 & 58 & 0.0091 & 25.04 & 61 & 53 & 0.0067 & 21.94 & 59 & 46 & 0.0036 & 18.08\\
63 & 57 & 0.0085 & 24.39 & 61 & 52 & 0.0062 & 21.35 & 59 & 45 & 0.0031 & 17.57\\
63 & 56 & 0.0079 & 23.75 & 61 & 51 & 0.0056 & 20.78 & 59 & 44 & 0.0027 & 17.07\\
63 & 55 & 0.0074 & 23.13 & 61 & 50 & 0.0051 & 20.22 & 59 & 43 & 0.0022 & 16.58\\
63 & 54 & 0.0068 & 22.52 & 61 & 49 & 0.0046 & 19.66 & 59 & 42 & 0.0018 & 16.10\\
63 & 53 & 0.0062 & 21.93 & 61 & 48 & 0.0041 & 19.12 & 59 & 41 & 0.0014 & 15.63\\
63 & 52 & 0.0057 & 21.34 & 61 & 47 & 0.0036 & 18.60 & 59 & 40 & 0.0009 & 15.17\\
63 & 51 & 0.0052 & 20.77 & 61 & 46 & 0.0032 & 18.08 & 59 & 39 & 0.0005 & 14.71\\
63 & 50 & 0.0047 & 20.21 & 61 & 45 & 0.0027 & 17.57 & 59 & 38 & 0.0001 & 14.26\\
63 & 49 & 0.0042 & 19.66 & 61 & 44 & 0.0022 & 17.07 & 58 & 58 & 0.0103 & 25.07\\
63 & 48 & 0.0037 & 19.12 & 61 & 43 & 0.0018 & 16.58 & 58 & 57 & 0.0097 & 24.42\\
63 & 47 & 0.0032 & 18.59 & 61 & 42 & 0.0013 & 16.10 & 58 & 56 & 0.0091 & 23.78\\
63 & 46 & 0.0027 & 18.07 & 61 & 41 & 0.0009 & 15.63 & 58 & 55 & 0.0085 & 23.16\\
63 & 45 & 0.0022 & 17.56 & 61 & 40 & 0.0005 & 15.16 & 58 & 54 & 0.0079 & 22.55\\
63 & 44 & 0.0018 & 17.06 & 61 & 39 & 0.0001 & 14.71 & 58 & 53 & 0.0074 & 21.95\\
63 & 43 & 0.0013 & 16.57 & 60 & 60 & 0.0110 & 26.40 & 58 & 52 & 0.0068 & 21.36\\
63 & 42 & 0.0009 & 16.09 & 60 & 59 & 0.0104 & 25.72 & 58 & 51 & 0.0063 & 20.79\\
63 & 41 & 0.0005 & 15.62 & 60 & 58 & 0.0098 & 25.06 & 58 & 50 & 0.0058 & 20.23\\
63 & 40 & 0.0000 & 15.16 & 60 & 57 & 0.0092 & 24.41 & 58 & 49 & 0.0053 & 19.68\\
62 & 62 & 0.0119 & 27.79 & 60 & 56 & 0.0086 & 23.77 & 58 & 48 & 0.0048 & 19.14\\
62 & 61 & 0.0112 & 27.08 & 60 & 55 & 0.0080 & 23.15 & 58 & 47 & 0.0043 & 18.61\\
62 & 60 & 0.0106 & 26.39 & 60 & 54 & 0.0075 & 22.54 & 58 & 46 & 0.0038 & 18.09\\
62 & 59 & 0.0099 & 25.71 & 60 & 53 & 0.0069 & 21.94 & 58 & 45 & 0.0034 & 17.58\\
62 & 58 & 0.0093 & 25.05 & 60 & 52 & 0.0064 & 21.35 & 58 & 44 & 0.0029 & 17.08\\
62 & 57 & 0.0087 & 24.40 & 60 & 51 & 0.0059 & 20.78 & 58 & 43 & 0.0025 & 16.59\\
62 & 56 & 0.0082 & 23.76 & 60 & 50 & 0.0053 & 20.22 & 58 & 42 & 0.0020 & 16.10\\
62 & 55 & 0.0076 & 23.14 & 60 & 49 & 0.0048 & 19.67 & 58 & 41 & 0.0016 & 15.63\\
62 & 54 & 0.0070 & 22.53 & 60 & 48 & 0.0043 & 19.13 & 58 & 40 & 0.0011 & 15.17\\
62 & 53 & 0.0065 & 21.93 & 60 & 47 & 0.0039 & 18.60 & 58 & 39 & 0.0007 & 14.71\\
62 & 52 & 0.0059 & 21.35 & 60 & 46 & 0.0034 & 18.08 & 58 & 38 & 0.0003 & 14.26\\
62 & 51 & 0.0054 & 20.77 & 60 & 45 & 0.0029 & 17.57 & 57 & 57 & 0.0099 & 24.43\\
62 & 50 & 0.0049 & 20.21 & 60 & 44 & 0.0025 & 17.07 & 57 & 56 & 0.0093 & 23.79\\
62 & 49 & 0.0044 & 19.66 & 60 & 43 & 0.0020 & 16.58 & 57 & 55 & 0.0087 & 23.16\\
62 & 48 & 0.0039 & 19.12 & 60 & 42 & 0.0016 & 16.10 & 57 & 54 & 0.0082 & 22.55\\
62 & 47 & 0.0034 & 18.59 & 60 & 41 & 0.0011 & 15.63 & 57 & 53 & 0.0076 & 21.95\\
\bottomrule
\end{tabular}
\newpage
\begin{tabular}{llll|llll|llll}
 \toprule 
\(T_{db}\) & \(T_{wb}\) & \(\omega\) & \(h\) & \(T_{db}\) & \(T_{wb}\) & \(\omega\) & \(h\) & \(T_{db}\) & \(T_{wb}\) & \(\omega\) & \(h\)  \\ \midrule 
57 & 52 & 0.0071 & 21.37 & 55 & 41 & 0.0022 & 15.64 & 52 & 47 & 0.0057 & 18.63\\
57 & 51 & 0.0065 & 20.79 & 55 & 40 & 0.0018 & 15.17 & 52 & 46 & 0.0052 & 18.11\\
57 & 50 & 0.0060 & 20.23 & 55 & 39 & 0.0014 & 14.72 & 52 & 45 & 0.0047 & 17.59\\
57 & 49 & 0.0055 & 19.68 & 55 & 38 & 0.0010 & 14.27 & 52 & 44 & 0.0043 & 17.09\\
57 & 48 & 0.0050 & 19.14 & 55 & 37 & 0.0006 & 13.83 & 52 & 43 & 0.0038 & 16.60\\
57 & 47 & 0.0045 & 18.61 & 55 & 36 & 0.0002 & 13.40 & 52 & 42 & 0.0034 & 16.12\\
57 & 46 & 0.0041 & 18.09 & 54 & 54 & 0.0089 & 22.57 & 52 & 41 & 0.0029 & 15.64\\
57 & 45 & 0.0036 & 17.58 & 54 & 53 & 0.0083 & 21.97 & 52 & 40 & 0.0025 & 15.18\\
57 & 44 & 0.0031 & 17.08 & 54 & 52 & 0.0078 & 21.38 & 52 & 39 & 0.0021 & 14.72\\
57 & 43 & 0.0027 & 16.59 & 54 & 51 & 0.0072 & 20.81 & 52 & 38 & 0.0017 & 14.27\\
57 & 42 & 0.0022 & 16.11 & 54 & 50 & 0.0067 & 20.24 & 52 & 37 & 0.0012 & 13.83\\
57 & 41 & 0.0018 & 15.63 & 54 & 49 & 0.0062 & 19.69 & 52 & 36 & 0.0008 & 13.40\\
57 & 40 & 0.0014 & 15.17 & 54 & 48 & 0.0057 & 19.15 & 52 & 35 & 0.0005 & 12.97\\
57 & 39 & 0.0010 & 14.71 & 54 & 47 & 0.0052 & 18.62 & 52 & 34 & 0.0001 & 12.55\\
57 & 38 & 0.0005 & 14.27 & 54 & 46 & 0.0047 & 18.10 & 51 & 51 & 0.0079 & 20.82\\
57 & 37 & 0.0001 & 13.83 & 54 & 45 & 0.0043 & 17.59 & 51 & 50 & 0.0074 & 20.26\\
56 & 56 & 0.0095 & 23.79 & 54 & 44 & 0.0038 & 17.09 & 51 & 49 & 0.0069 & 19.70\\
56 & 55 & 0.0090 & 23.17 & 54 & 43 & 0.0034 & 16.60 & 51 & 48 & 0.0064 & 19.16\\
56 & 54 & 0.0084 & 22.56 & 54 & 42 & 0.0029 & 16.11 & 51 & 47 & 0.0059 & 18.63\\
56 & 53 & 0.0078 & 21.96 & 54 & 41 & 0.0025 & 15.64 & 51 & 46 & 0.0054 & 18.11\\
56 & 52 & 0.0073 & 21.37 & 54 & 40 & 0.0020 & 15.18 & 51 & 45 & 0.0049 & 17.60\\
56 & 51 & 0.0068 & 20.80 & 54 & 39 & 0.0016 & 14.72 & 51 & 44 & 0.0045 & 17.10\\
56 & 50 & 0.0063 & 20.24 & 54 & 38 & 0.0012 & 14.27 & 51 & 43 & 0.0040 & 16.60\\
56 & 49 & 0.0057 & 19.68 & 54 & 37 & 0.0008 & 13.83 & 51 & 42 & 0.0036 & 16.12\\
56 & 48 & 0.0053 & 19.14 & 54 & 36 & 0.0004 & 13.40 & 51 & 41 & 0.0031 & 15.65\\
56 & 47 & 0.0048 & 18.61 & 54 & 35 & 0.0000 & 12.97 & 51 & 40 & 0.0027 & 15.18\\
56 & 46 & 0.0043 & 18.09 & 53 & 53 & 0.0085 & 21.97 & 51 & 39 & 0.0023 & 14.72\\
56 & 45 & 0.0038 & 17.58 & 53 & 52 & 0.0080 & 21.39 & 51 & 38 & 0.0019 & 14.27\\
56 & 44 & 0.0034 & 17.08 & 53 & 51 & 0.0075 & 20.81 & 51 & 37 & 0.0015 & 13.83\\
56 & 43 & 0.0029 & 16.59 & 53 & 50 & 0.0069 & 20.25 & 51 & 36 & 0.0011 & 13.40\\
56 & 42 & 0.0025 & 16.11 & 53 & 49 & 0.0064 & 19.70 & 51 & 35 & 0.0007 & 12.97\\
56 & 41 & 0.0020 & 15.64 & 53 & 48 & 0.0059 & 19.15 & 51 & 34 & 0.0003 & 12.55\\
56 & 40 & 0.0016 & 15.17 & 53 & 47 & 0.0054 & 18.62 & 50 & 50 & 0.0076 & 20.26\\
56 & 39 & 0.0012 & 14.72 & 53 & 46 & 0.0050 & 18.10 & 50 & 49 & 0.0071 & 19.71\\
56 & 38 & 0.0008 & 14.27 & 53 & 45 & 0.0045 & 17.59 & 50 & 48 & 0.0066 & 19.16\\
56 & 37 & 0.0004 & 13.83 & 53 & 44 & 0.0040 & 17.09 & 50 & 47 & 0.0061 & 18.63\\
55 & 55 & 0.0092 & 23.17 & 53 & 43 & 0.0036 & 16.60 & 50 & 46 & 0.0056 & 18.11\\
55 & 54 & 0.0086 & 22.56 & 53 & 42 & 0.0031 & 16.12 & 50 & 45 & 0.0052 & 17.60\\
55 & 53 & 0.0081 & 21.96 & 53 & 41 & 0.0027 & 15.64 & 50 & 44 & 0.0047 & 17.10\\
55 & 52 & 0.0075 & 21.38 & 53 & 40 & 0.0023 & 15.18 & 50 & 43 & 0.0043 & 16.61\\
55 & 51 & 0.0070 & 20.80 & 53 & 39 & 0.0018 & 14.72 & 50 & 42 & 0.0038 & 16.12\\
55 & 50 & 0.0065 & 20.24 & 53 & 38 & 0.0014 & 14.27 & 50 & 41 & 0.0034 & 15.65\\
55 & 49 & 0.0060 & 19.69 & 53 & 37 & 0.0010 & 13.83 & 50 & 40 & 0.0029 & 15.18\\
55 & 48 & 0.0055 & 19.15 & 53 & 36 & 0.0006 & 13.40 & 50 & 39 & 0.0025 & 14.72\\
55 & 47 & 0.0050 & 18.62 & 53 & 35 & 0.0002 & 12.97 & 50 & 38 & 0.0021 & 14.28\\
55 & 46 & 0.0045 & 18.10 & 52 & 52 & 0.0082 & 21.39 & 50 & 37 & 0.0017 & 13.83\\
55 & 45 & 0.0040 & 17.59 & 52 & 51 & 0.0077 & 20.82 & 50 & 36 & 0.0013 & 13.40\\
55 & 44 & 0.0036 & 17.08 & 52 & 50 & 0.0072 & 20.25 & 50 & 35 & 0.0009 & 12.97\\
55 & 43 & 0.0031 & 16.59 & 52 & 49 & 0.0067 & 19.70 & 50 & 34 & 0.0005 & 12.55\\
55 & 42 & 0.0027 & 16.11 & 52 & 48 & 0.0062 & 19.16 & 50 & 33 & 0.0001 & 12.14\\
\bottomrule
\end{tabular}
\newpage
\begin{tabular}{llll|llll|llll}
 \toprule 
\(T_{db}\) & \(T_{wb}\) & \(\omega\) & \(h\) & \(T_{db}\) & \(T_{wb}\) & \(\omega\) & \(h\) & \(T_{db}\) & \(T_{wb}\) & \(\omega\) & \(h\)  \\ \midrule 
49 & 49 & 0.0073 & 19.71 & 46 & 46 & 0.0066 & 18.12 & 43 & 35 & 0.0025 & 12.98\\
49 & 48 & 0.0068 & 19.17 & 46 & 45 & 0.0061 & 17.61 & 43 & 34 & 0.0021 & 12.56\\
49 & 47 & 0.0064 & 18.64 & 46 & 44 & 0.0056 & 17.11 & 43 & 33 & 0.0017 & 12.14\\
49 & 46 & 0.0059 & 18.11 & 46 & 43 & 0.0052 & 16.62 & 43 & 32 & 0.0013 & 11.74\\
49 & 45 & 0.0054 & 17.60 & 46 & 42 & 0.0047 & 16.13 & 42 & 42 & 0.0056 & 16.14\\
49 & 44 & 0.0049 & 17.10 & 46 & 41 & 0.0043 & 15.66 & 42 & 41 & 0.0052 & 15.66\\
49 & 43 & 0.0045 & 16.61 & 46 & 40 & 0.0038 & 15.19 & 42 & 40 & 0.0047 & 15.20\\
49 & 42 & 0.0040 & 16.12 & 46 & 39 & 0.0034 & 14.73 & 42 & 39 & 0.0043 & 14.74\\
49 & 41 & 0.0036 & 15.65 & 46 & 38 & 0.0030 & 14.28 & 42 & 38 & 0.0039 & 14.29\\
49 & 40 & 0.0032 & 15.18 & 46 & 37 & 0.0026 & 13.84 & 42 & 37 & 0.0035 & 13.84\\
49 & 39 & 0.0027 & 14.73 & 46 & 36 & 0.0022 & 13.40 & 42 & 36 & 0.0031 & 13.41\\
49 & 38 & 0.0023 & 14.28 & 46 & 35 & 0.0018 & 12.98 & 42 & 35 & 0.0027 & 12.98\\
49 & 37 & 0.0019 & 13.83 & 46 & 34 & 0.0014 & 12.56 & 42 & 34 & 0.0023 & 12.56\\
49 & 36 & 0.0015 & 13.40 & 46 & 33 & 0.0010 & 12.14 & 42 & 33 & 0.0019 & 12.14\\
49 & 35 & 0.0011 & 12.97 & 46 & 32 & 0.0006 & 11.74 & 42 & 32 & 0.0015 & 11.74\\
49 & 34 & 0.0007 & 12.55 & 45 & 45 & 0.0063 & 17.61 & 41 & 41 & 0.0054 & 15.67\\
49 & 33 & 0.0004 & 12.14 & 45 & 44 & 0.0058 & 17.11 & 41 & 40 & 0.0050 & 15.20\\
48 & 48 & 0.0071 & 19.17 & 45 & 43 & 0.0054 & 16.62 & 41 & 39 & 0.0045 & 14.74\\
48 & 47 & 0.0066 & 18.64 & 45 & 42 & 0.0049 & 16.13 & 41 & 38 & 0.0041 & 14.29\\
48 & 46 & 0.0061 & 18.12 & 45 & 41 & 0.0045 & 15.66 & 41 & 37 & 0.0037 & 13.84\\
48 & 45 & 0.0056 & 17.61 & 45 & 40 & 0.0041 & 15.19 & 41 & 36 & 0.0033 & 13.41\\
48 & 44 & 0.0052 & 17.10 & 45 & 39 & 0.0036 & 14.73 & 41 & 35 & 0.0029 & 12.98\\
48 & 43 & 0.0047 & 16.61 & 45 & 38 & 0.0032 & 14.28 & 41 & 34 & 0.0025 & 12.56\\
48 & 42 & 0.0043 & 16.13 & 45 & 37 & 0.0028 & 13.84 & 41 & 33 & 0.0021 & 12.14\\
48 & 41 & 0.0038 & 15.65 & 45 & 36 & 0.0024 & 13.40 & 41 & 32 & 0.0018 & 11.74\\
48 & 40 & 0.0034 & 15.19 & 45 & 35 & 0.0020 & 12.98 & 40 & 40 & 0.0052 & 15.20\\
48 & 39 & 0.0030 & 14.73 & 45 & 34 & 0.0016 & 12.56 & 40 & 39 & 0.0048 & 14.74\\
48 & 38 & 0.0025 & 14.28 & 45 & 33 & 0.0012 & 12.14 & 40 & 38 & 0.0043 & 14.29\\
48 & 37 & 0.0021 & 13.84 & 45 & 32 & 0.0009 & 11.74 & 40 & 37 & 0.0039 & 13.85\\
48 & 36 & 0.0017 & 13.40 & 44 & 44 & 0.0061 & 17.11 & 40 & 36 & 0.0035 & 13.41\\
48 & 35 & 0.0013 & 12.97 & 44 & 43 & 0.0056 & 16.62 & 40 & 35 & 0.0031 & 12.98\\
48 & 34 & 0.0010 & 12.55 & 44 & 42 & 0.0052 & 16.14 & 40 & 34 & 0.0027 & 12.56\\
48 & 33 & 0.0006 & 12.14 & 44 & 41 & 0.0047 & 15.66 & 40 & 33 & 0.0024 & 12.14\\
48 & 32 & 0.0002 & 11.74 & 44 & 40 & 0.0043 & 15.19 & 40 & 32 & 0.0020 & 11.74\\
47 & 47 & 0.0068 & 18.64 & 44 & 39 & 0.0039 & 14.73 & 39 & 39 & 0.0050 & 14.74\\
47 & 46 & 0.0063 & 18.12 & 44 & 38 & 0.0034 & 14.28 & 39 & 38 & 0.0046 & 14.29\\
47 & 45 & 0.0058 & 17.61 & 44 & 37 & 0.0030 & 13.84 & 39 & 37 & 0.0042 & 13.85\\
47 & 44 & 0.0054 & 17.11 & 44 & 36 & 0.0026 & 13.41 & 39 & 36 & 0.0038 & 13.41\\
47 & 43 & 0.0049 & 16.61 & 44 & 35 & 0.0022 & 12.98 & 39 & 35 & 0.0034 & 12.98\\
47 & 42 & 0.0045 & 16.13 & 44 & 34 & 0.0018 & 12.56 & 39 & 34 & 0.0030 & 12.56\\
47 & 41 & 0.0040 & 15.65 & 44 & 33 & 0.0015 & 12.14 & 39 & 33 & 0.0026 & 12.14\\
47 & 40 & 0.0036 & 15.19 & 44 & 32 & 0.0011 & 11.74 & 39 & 32 & 0.0022 & 11.74\\
47 & 39 & 0.0032 & 14.73 & 43 & 43 & 0.0058 & 16.62 & 38 & 38 & 0.0048 & 14.29\\
47 & 38 & 0.0028 & 14.28 & 43 & 42 & 0.0054 & 16.14 & 38 & 37 & 0.0044 & 13.85\\
47 & 37 & 0.0024 & 13.84 & 43 & 41 & 0.0049 & 15.66 & 38 & 36 & 0.0040 & 13.41\\
47 & 36 & 0.0020 & 13.40 & 43 & 40 & 0.0045 & 15.19 & 38 & 35 & 0.0036 & 12.98\\
47 & 35 & 0.0016 & 12.98 & 43 & 39 & 0.0041 & 14.74 & 38 & 34 & 0.0032 & 12.56\\
47 & 34 & 0.0012 & 12.56 & 43 & 38 & 0.0037 & 14.28 & 38 & 33 & 0.0028 & 12.14\\
47 & 33 & 0.0008 & 12.14 & 43 & 37 & 0.0033 & 13.84 & 38 & 32 & 0.0024 & 11.74\\
47 & 32 & 0.0004 & 11.74 & 43 & 36 & 0.0029 & 13.41 & 37 & 37 & 0.0046 & 13.85\\
\bottomrule
\end{tabular}
\newpage
\begin{tabular}{llll|llll|llll}
 \toprule 
\(T_{db}\) & \(T_{wb}\) & \(\omega\) & \(h\) & \(T_{db}\) & \(T_{wb}\) & \(\omega\) & \(h\) & \(T_{db}\) & \(T_{wb}\) & \(\omega\) & \(h\)  \\ \midrule 
37 & 36 & 0.0042 & 13.41 &  &  &  &  &  &  &  & \\
37 & 35 & 0.0038 & 12.98 &  &  &  &  &  &  &  & \\
37 & 34 & 0.0034 & 12.56 &  &  &  &  &  &  &  & \\
37 & 33 & 0.0030 & 12.14 &  &  &  &  &  &  &  & \\
37 & 32 & 0.0027 & 11.74 &  &  &  &  &  &  &  & \\
36 & 36 & 0.0044 & 13.41 &  &  &  &  &  &  &  & \\
36 & 35 & 0.0040 & 12.98 &  &  &  &  &  &  &  & \\
36 & 34 & 0.0036 & 12.56 &  &  &  &  &  &  &  & \\
36 & 33 & 0.0033 & 12.14 &  &  &  &  &  &  &  & \\
36 & 32 & 0.0029 & 11.74 &  &  &  &  &  &  &  & \\
35 & 35 & 0.0043 & 12.98 &  &  &  &  &  &  &  & \\
35 & 34 & 0.0039 & 12.56 &  &  &  &  &  &  &  & \\
35 & 33 & 0.0035 & 12.15 &  &  &  &  &  &  &  & \\
35 & 32 & 0.0031 & 11.74 &  &  &  &  &  &  &  & \\
34 & 34 & 0.0041 & 12.56 &  &  &  &  &  &  &  & \\
34 & 33 & 0.0037 & 12.15 &  &  &  &  &  &  &  & \\
34 & 32 & 0.0033 & 11.74 &  &  &  &  &  &  &  & \\
33 & 33 & 0.0039 & 12.15 &  &  &  &  &  &  &  & \\
33 & 32 & 0.0035 & 11.74 &  &  &  &  &  &  &  & \\
32 & 32 & 0.0038 & 11.74 &  &  &  &  &  &  &  & \\
 &  &  &  &  &  &  &  &  &  &  & \\
 &  &  &  &  &  &  &  &  &  &  & \\
 &  &  &  &  &  &  &  &  &  &  & \\
 &  &  &  &  &  &  &  &  &  &  & \\
 &  &  &  &  &  &  &  &  &  &  & \\
 &  &  &  &  &  &  &  &  &  &  & \\
 &  &  &  &  &  &  &  &  &  &  & \\
 &  &  &  &  &  &  &  &  &  &  & \\
 &  &  &  &  &  &  &  &  &  &  & \\
 &  &  &  &  &  &  &  &  &  &  & \\
 &  &  &  &  &  &  &  &  &  &  & \\
 &  &  &  &  &  &  &  &  &  &  & \\
 &  &  &  &  &  &  &  &  &  &  & \\
 &  &  &  &  &  &  &  &  &  &  & \\
 &  &  &  &  &  &  &  &  &  &  & \\
 &  &  &  &  &  &  &  &  &  &  & \\
 &  &  &  &  &  &  &  &  &  &  & \\
 &  &  &  &  &  &  &  &  &  &  & \\
 &  &  &  &  &  &  &  &  &  &  & \\
 &  &  &  &  &  &  &  &  &  &  & \\
 &  &  &  &  &  &  &  &  &  &  & \\
 &  &  &  &  &  &  &  &  &  &  & \\
 &  &  &  &  &  &  &  &  &  &  & \\
 &  &  &  &  &  &  &  &  &  &  & \\
 &  &  &  &  &  &  &  &  &  &  & \\
 &  &  &  &  &  &  &  &  &  &  & \\
 &  &  &  &  &  &  &  &  &  &  & \\
 &  &  &  &  &  &  &  &  &  &  & \\
 &  &  &  &  &  &  &  &  &  &  & \\
 &  &  &  &  &  &  &  &  &  &  & \\
\bottomrule
\end{tabular}
\newpage


}

\chapter{\(T_{db}\) and \(\phi\) }


\newpage
{
\small

\begin{tabular}{llll|llll|llll}
 \toprule 
\(T_{db}\) & \(\phi\) & \(\omega\) & \(h\) & \(T_{db}\) & \(\phi\) & \(\omega\) & \(h\) & \(T_{db}\) & \(\phi\) & \(\omega\) & \(h\)  \\ \midrule 
100 & 76 & 0.0321 & 59.51 & 100 & 26 & 0.0106 & 35.75 & 99 & 45 & 0.0181 & 43.72\\
100 & 75 & 0.0317 & 59.02 & 100 & 25 & 0.0102 & 35.29 & 99 & 44 & 0.0176 & 43.26\\
100 & 74 & 0.0312 & 58.53 & 100 & 24 & 0.0098 & 34.83 & 99 & 43 & 0.0172 & 42.80\\
100 & 73 & 0.0308 & 58.04 & 100 & 23 & 0.0094 & 34.37 & 99 & 42 & 0.0168 & 42.35\\
100 & 72 & 0.0303 & 57.55 & 100 & 22 & 0.0090 & 33.92 & 99 & 41 & 0.0164 & 41.89\\
100 & 71 & 0.0299 & 57.06 & 100 & 21 & 0.0086 & 33.46 & 99 & 40 & 0.0160 & 41.44\\
100 & 70 & 0.0295 & 56.58 & 100 & 20 & 0.0081 & 33.00 & 99 & 39 & 0.0156 & 40.99\\
100 & 69 & 0.0290 & 56.09 & 100 & 19 & 0.0077 & 32.55 & 99 & 38 & 0.0152 & 40.54\\
100 & 68 & 0.0286 & 55.60 & 100 & 18 & 0.0073 & 32.09 & 99 & 37 & 0.0148 & 40.08\\
100 & 67 & 0.0281 & 55.12 & 100 & 17 & 0.0069 & 31.64 & 99 & 36 & 0.0144 & 39.63\\
100 & 66 & 0.0277 & 54.63 & 100 & 16 & 0.0065 & 31.18 & 99 & 35 & 0.0140 & 39.18\\
100 & 65 & 0.0273 & 54.15 & 100 & 15 & 0.0061 & 30.73 & 99 & 34 & 0.0135 & 38.73\\
100 & 64 & 0.0268 & 53.66 & 100 & 14 & 0.0057 & 30.28 & 99 & 33 & 0.0131 & 38.28\\
100 & 63 & 0.0264 & 53.18 & 100 & 13 & 0.0053 & 29.82 & 99 & 32 & 0.0127 & 37.83\\
100 & 62 & 0.0260 & 52.70 & 100 & 12 & 0.0049 & 29.37 & 99 & 31 & 0.0123 & 37.38\\
100 & 61 & 0.0255 & 52.22 & 100 & 11 & 0.0045 & 28.92 & 99 & 30 & 0.0119 & 36.94\\
100 & 60 & 0.0251 & 51.73 & 99 & 79 & 0.0324 & 59.58 & 99 & 29 & 0.0115 & 36.49\\
100 & 59 & 0.0247 & 51.25 & 99 & 78 & 0.0320 & 59.10 & 99 & 28 & 0.0111 & 36.04\\
100 & 58 & 0.0242 & 50.77 & 99 & 77 & 0.0316 & 58.63 & 99 & 27 & 0.0107 & 35.60\\
100 & 57 & 0.0238 & 50.29 & 99 & 76 & 0.0311 & 58.15 & 99 & 26 & 0.0103 & 35.15\\
100 & 56 & 0.0234 & 49.82 & 99 & 75 & 0.0307 & 57.67 & 99 & 25 & 0.0099 & 34.70\\
100 & 55 & 0.0229 & 49.34 & 99 & 74 & 0.0303 & 57.20 & 99 & 24 & 0.0095 & 34.26\\
100 & 54 & 0.0225 & 48.86 & 99 & 73 & 0.0298 & 56.73 & 99 & 23 & 0.0091 & 33.82\\
100 & 53 & 0.0221 & 48.38 & 99 & 72 & 0.0294 & 56.25 & 99 & 22 & 0.0087 & 33.37\\
100 & 52 & 0.0216 & 47.91 & 99 & 71 & 0.0290 & 55.78 & 99 & 21 & 0.0083 & 32.93\\
100 & 51 & 0.0212 & 47.43 & 99 & 70 & 0.0286 & 55.31 & 99 & 20 & 0.0079 & 32.49\\
100 & 50 & 0.0208 & 46.96 & 99 & 69 & 0.0281 & 54.84 & 99 & 19 & 0.0075 & 32.05\\
100 & 49 & 0.0203 & 46.48 & 99 & 68 & 0.0277 & 54.37 & 99 & 18 & 0.0071 & 31.61\\
100 & 48 & 0.0199 & 46.01 & 99 & 67 & 0.0273 & 53.90 & 99 & 17 & 0.0067 & 31.16\\
100 & 47 & 0.0195 & 45.54 & 99 & 66 & 0.0268 & 53.43 & 99 & 16 & 0.0063 & 30.72\\
100 & 46 & 0.0191 & 45.06 & 99 & 65 & 0.0264 & 52.96 & 99 & 15 & 0.0059 & 30.29\\
100 & 45 & 0.0186 & 44.59 & 99 & 64 & 0.0260 & 52.49 & 99 & 14 & 0.0055 & 29.85\\
100 & 44 & 0.0182 & 44.12 & 99 & 63 & 0.0256 & 52.03 & 99 & 13 & 0.0051 & 29.41\\
100 & 43 & 0.0178 & 43.65 & 99 & 62 & 0.0252 & 51.56 & 99 & 12 & 0.0047 & 28.97\\
100 & 42 & 0.0174 & 43.18 & 99 & 61 & 0.0247 & 51.09 & 99 & 11 & 0.0043 & 28.53\\
100 & 41 & 0.0169 & 42.71 & 99 & 60 & 0.0243 & 50.63 & 98 & 82 & 0.0327 & 59.59\\
100 & 40 & 0.0165 & 42.24 & 99 & 59 & 0.0239 & 50.16 & 98 & 81 & 0.0322 & 59.12\\
100 & 39 & 0.0161 & 41.78 & 99 & 58 & 0.0235 & 49.70 & 98 & 80 & 0.0318 & 58.66\\
100 & 38 & 0.0157 & 41.31 & 99 & 57 & 0.0231 & 49.23 & 98 & 79 & 0.0314 & 58.20\\
100 & 37 & 0.0152 & 40.84 & 99 & 56 & 0.0226 & 48.77 & 98 & 78 & 0.0310 & 57.74\\
100 & 36 & 0.0148 & 40.38 & 99 & 55 & 0.0222 & 48.31 & 98 & 77 & 0.0306 & 57.28\\
100 & 35 & 0.0144 & 39.91 & 99 & 54 & 0.0218 & 47.85 & 98 & 76 & 0.0301 & 56.82\\
100 & 34 & 0.0140 & 39.45 & 99 & 53 & 0.0214 & 47.38 & 98 & 75 & 0.0297 & 56.36\\
100 & 33 & 0.0136 & 38.98 & 99 & 52 & 0.0210 & 46.92 & 98 & 74 & 0.0293 & 55.90\\
100 & 32 & 0.0131 & 38.52 & 99 & 51 & 0.0205 & 46.46 & 98 & 73 & 0.0289 & 55.45\\
100 & 31 & 0.0127 & 38.06 & 99 & 50 & 0.0201 & 46.00 & 98 & 72 & 0.0285 & 54.99\\
100 & 30 & 0.0123 & 37.59 & 99 & 49 & 0.0197 & 45.55 & 98 & 71 & 0.0281 & 54.53\\
100 & 29 & 0.0119 & 37.13 & 99 & 48 & 0.0193 & 45.09 & 98 & 70 & 0.0277 & 54.07\\
100 & 28 & 0.0115 & 36.67 & 99 & 47 & 0.0189 & 44.63 & 98 & 69 & 0.0272 & 53.62\\
100 & 27 & 0.0110 & 36.21 & 99 & 46 & 0.0185 & 44.17 & 98 & 68 & 0.0268 & 53.16\\
\bottomrule
\end{tabular}
\newpage
\begin{tabular}{llll|llll|llll}
 \toprule 
\(T_{db}\) & \(\phi\) & \(\omega\) & \(h\) & \(T_{db}\) & \(\phi\) & \(\omega\) & \(h\) & \(T_{db}\) & \(\phi\) & \(\omega\) & \(h\)  \\ \midrule 
98 & 67 & 0.0264 & 52.71 & 98 & 17 & 0.0065 & 30.70 & 97 & 43 & 0.0162 & 41.16\\
98 & 66 & 0.0260 & 52.26 & 98 & 16 & 0.0061 & 30.27 & 97 & 42 & 0.0158 & 40.73\\
98 & 65 & 0.0256 & 51.80 & 98 & 15 & 0.0057 & 29.85 & 97 & 41 & 0.0154 & 40.31\\
98 & 64 & 0.0252 & 51.35 & 98 & 14 & 0.0053 & 29.42 & 97 & 40 & 0.0150 & 39.88\\
98 & 63 & 0.0248 & 50.90 & 98 & 13 & 0.0050 & 29.00 & 97 & 39 & 0.0147 & 39.46\\
98 & 62 & 0.0244 & 50.45 & 98 & 12 & 0.0046 & 28.57 & 97 & 38 & 0.0143 & 39.03\\
98 & 61 & 0.0240 & 49.99 & 98 & 11 & 0.0042 & 28.15 & 97 & 37 & 0.0139 & 38.61\\
98 & 60 & 0.0236 & 49.54 & 97 & 86 & 0.0332 & 59.99 & 97 & 36 & 0.0135 & 38.18\\
98 & 59 & 0.0232 & 49.09 & 97 & 85 & 0.0328 & 59.54 & 97 & 35 & 0.0131 & 37.76\\
98 & 58 & 0.0227 & 48.65 & 97 & 84 & 0.0324 & 59.09 & 97 & 34 & 0.0127 & 37.34\\
98 & 57 & 0.0223 & 48.20 & 97 & 83 & 0.0320 & 58.65 & 97 & 33 & 0.0124 & 36.92\\
98 & 56 & 0.0219 & 47.75 & 97 & 82 & 0.0316 & 58.20 & 97 & 32 & 0.0120 & 36.50\\
98 & 55 & 0.0215 & 47.30 & 97 & 81 & 0.0312 & 57.75 & 97 & 31 & 0.0116 & 36.08\\
98 & 54 & 0.0211 & 46.85 & 97 & 80 & 0.0308 & 57.30 & 97 & 30 & 0.0112 & 35.66\\
98 & 53 & 0.0207 & 46.41 & 97 & 79 & 0.0304 & 56.86 & 97 & 29 & 0.0108 & 35.24\\
98 & 52 & 0.0203 & 45.96 & 97 & 78 & 0.0300 & 56.41 & 97 & 28 & 0.0104 & 34.82\\
98 & 51 & 0.0199 & 45.52 & 97 & 77 & 0.0296 & 55.97 & 97 & 27 & 0.0101 & 34.40\\
98 & 50 & 0.0195 & 45.07 & 97 & 76 & 0.0292 & 55.52 & 97 & 26 & 0.0097 & 33.98\\
98 & 49 & 0.0191 & 44.63 & 97 & 75 & 0.0288 & 55.08 & 97 & 25 & 0.0093 & 33.56\\
98 & 48 & 0.0187 & 44.18 & 97 & 74 & 0.0284 & 54.64 & 97 & 24 & 0.0089 & 33.15\\
98 & 47 & 0.0183 & 43.74 & 97 & 73 & 0.0280 & 54.19 & 97 & 23 & 0.0086 & 32.73\\
98 & 46 & 0.0179 & 43.30 & 97 & 72 & 0.0276 & 53.75 & 97 & 22 & 0.0082 & 32.31\\
98 & 45 & 0.0175 & 42.86 & 97 & 71 & 0.0272 & 53.31 & 97 & 21 & 0.0078 & 31.90\\
98 & 44 & 0.0171 & 42.41 & 97 & 70 & 0.0268 & 52.87 & 97 & 20 & 0.0074 & 31.48\\
98 & 43 & 0.0167 & 41.97 & 97 & 69 & 0.0264 & 52.43 & 97 & 19 & 0.0071 & 31.07\\
98 & 42 & 0.0163 & 41.53 & 97 & 68 & 0.0260 & 51.99 & 97 & 18 & 0.0067 & 30.65\\
98 & 41 & 0.0159 & 41.09 & 97 & 67 & 0.0256 & 51.55 & 97 & 17 & 0.0063 & 30.24\\
98 & 40 & 0.0155 & 40.65 & 97 & 66 & 0.0252 & 51.11 & 97 & 16 & 0.0059 & 29.83\\
98 & 39 & 0.0151 & 40.21 & 97 & 65 & 0.0248 & 50.67 & 97 & 15 & 0.0056 & 29.41\\
98 & 38 & 0.0147 & 39.78 & 97 & 64 & 0.0244 & 50.23 & 97 & 14 & 0.0052 & 29.00\\
98 & 37 & 0.0143 & 39.34 & 97 & 63 & 0.0240 & 49.80 & 97 & 13 & 0.0048 & 28.59\\
98 & 36 & 0.0139 & 38.90 & 97 & 62 & 0.0236 & 49.36 & 97 & 12 & 0.0044 & 28.18\\
98 & 35 & 0.0135 & 38.47 & 97 & 61 & 0.0232 & 48.92 & 97 & 11 & 0.0041 & 27.77\\
98 & 34 & 0.0131 & 38.03 & 97 & 60 & 0.0228 & 48.49 & 96 & 89 & 0.0334 & 59.88\\
98 & 33 & 0.0127 & 37.59 & 97 & 59 & 0.0224 & 48.05 & 96 & 88 & 0.0330 & 59.45\\
98 & 32 & 0.0123 & 37.16 & 97 & 58 & 0.0220 & 47.62 & 96 & 87 & 0.0326 & 59.01\\
98 & 31 & 0.0120 & 36.72 & 97 & 57 & 0.0216 & 47.18 & 96 & 86 & 0.0322 & 58.58\\
98 & 30 & 0.0116 & 36.29 & 97 & 56 & 0.0213 & 46.75 & 96 & 85 & 0.0318 & 58.14\\
98 & 29 & 0.0112 & 35.86 & 97 & 55 & 0.0209 & 46.32 & 96 & 84 & 0.0314 & 57.71\\
98 & 28 & 0.0108 & 35.42 & 97 & 54 & 0.0205 & 45.88 & 96 & 83 & 0.0310 & 57.28\\
98 & 27 & 0.0104 & 34.99 & 97 & 53 & 0.0201 & 45.45 & 96 & 82 & 0.0306 & 56.84\\
98 & 26 & 0.0100 & 34.56 & 97 & 52 & 0.0197 & 45.02 & 96 & 81 & 0.0302 & 56.41\\
98 & 25 & 0.0096 & 34.13 & 97 & 51 & 0.0193 & 44.59 & 96 & 80 & 0.0298 & 55.98\\
98 & 24 & 0.0092 & 33.70 & 97 & 50 & 0.0189 & 44.16 & 96 & 79 & 0.0295 & 55.55\\
98 & 23 & 0.0088 & 33.27 & 97 & 49 & 0.0185 & 43.73 & 96 & 78 & 0.0291 & 55.12\\
98 & 22 & 0.0084 & 32.84 & 97 & 48 & 0.0181 & 43.30 & 96 & 77 & 0.0287 & 54.69\\
98 & 21 & 0.0080 & 32.41 & 97 & 47 & 0.0177 & 42.87 & 96 & 76 & 0.0283 & 54.26\\
98 & 20 & 0.0077 & 31.98 & 97 & 46 & 0.0174 & 42.44 & 96 & 75 & 0.0279 & 53.83\\
98 & 19 & 0.0073 & 31.55 & 97 & 45 & 0.0170 & 42.01 & 96 & 74 & 0.0275 & 53.40\\
98 & 18 & 0.0069 & 31.13 & 97 & 44 & 0.0166 & 41.59 & 96 & 73 & 0.0271 & 52.97\\
\bottomrule
\end{tabular}
\newpage
\begin{tabular}{llll|llll|llll}
 \toprule 
\(T_{db}\) & \(\phi\) & \(\omega\) & \(h\) & \(T_{db}\) & \(\phi\) & \(\omega\) & \(h\) & \(T_{db}\) & \(\phi\) & \(\omega\) & \(h\)  \\ \midrule 
96 & 72 & 0.0267 & 52.54 & 96 & 22 & 0.0079 & 31.79 & 95 & 54 & 0.0192 & 44.00\\
96 & 71 & 0.0263 & 52.12 & 96 & 21 & 0.0076 & 31.39 & 95 & 53 & 0.0189 & 43.60\\
96 & 70 & 0.0260 & 51.69 & 96 & 20 & 0.0072 & 30.99 & 95 & 52 & 0.0185 & 43.19\\
96 & 69 & 0.0256 & 51.26 & 96 & 19 & 0.0068 & 30.59 & 95 & 51 & 0.0181 & 42.79\\
96 & 68 & 0.0252 & 50.84 & 96 & 18 & 0.0065 & 30.19 & 95 & 50 & 0.0178 & 42.39\\
96 & 67 & 0.0248 & 50.41 & 96 & 17 & 0.0061 & 29.78 & 95 & 49 & 0.0174 & 41.98\\
96 & 66 & 0.0244 & 49.99 & 96 & 16 & 0.0057 & 29.38 & 95 & 48 & 0.0170 & 41.58\\
96 & 65 & 0.0240 & 49.56 & 96 & 15 & 0.0054 & 28.98 & 95 & 47 & 0.0167 & 41.18\\
96 & 64 & 0.0236 & 49.14 & 96 & 14 & 0.0050 & 28.58 & 95 & 46 & 0.0163 & 40.78\\
96 & 63 & 0.0233 & 48.72 & 96 & 13 & 0.0047 & 28.19 & 95 & 45 & 0.0159 & 40.38\\
96 & 62 & 0.0229 & 48.29 & 96 & 12 & 0.0043 & 27.79 & 95 & 44 & 0.0156 & 39.98\\
96 & 61 & 0.0225 & 47.87 & 96 & 11 & 0.0039 & 27.39 & 95 & 43 & 0.0152 & 39.58\\
96 & 60 & 0.0221 & 47.45 & 95 & 92 & 0.0335 & 59.73 & 95 & 42 & 0.0148 & 39.18\\
96 & 59 & 0.0217 & 47.03 & 95 & 91 & 0.0331 & 59.30 & 95 & 41 & 0.0145 & 38.78\\
96 & 58 & 0.0214 & 46.61 & 95 & 90 & 0.0327 & 58.88 & 95 & 40 & 0.0141 & 38.38\\
96 & 57 & 0.0210 & 46.19 & 95 & 89 & 0.0323 & 58.46 & 95 & 39 & 0.0138 & 37.98\\
96 & 56 & 0.0206 & 45.77 & 95 & 88 & 0.0319 & 58.04 & 95 & 38 & 0.0134 & 37.58\\
96 & 55 & 0.0202 & 45.35 & 95 & 87 & 0.0316 & 57.62 & 95 & 37 & 0.0130 & 37.19\\
96 & 54 & 0.0198 & 44.93 & 95 & 86 & 0.0312 & 57.20 & 95 & 36 & 0.0127 & 36.79\\
96 & 53 & 0.0195 & 44.51 & 95 & 85 & 0.0308 & 56.78 & 95 & 35 & 0.0123 & 36.39\\
96 & 52 & 0.0191 & 44.10 & 95 & 84 & 0.0304 & 56.36 & 95 & 34 & 0.0120 & 36.00\\
96 & 51 & 0.0187 & 43.68 & 95 & 83 & 0.0300 & 55.94 & 95 & 33 & 0.0116 & 35.60\\
96 & 50 & 0.0183 & 43.26 & 95 & 82 & 0.0297 & 55.52 & 95 & 32 & 0.0112 & 35.21\\
96 & 49 & 0.0179 & 42.85 & 95 & 81 & 0.0293 & 55.10 & 95 & 31 & 0.0109 & 34.81\\
96 & 48 & 0.0176 & 42.43 & 95 & 80 & 0.0289 & 54.68 & 95 & 30 & 0.0105 & 34.42\\
96 & 47 & 0.0172 & 42.02 & 95 & 79 & 0.0285 & 54.27 & 95 & 29 & 0.0102 & 34.03\\
96 & 46 & 0.0168 & 41.60 & 95 & 78 & 0.0281 & 53.85 & 95 & 28 & 0.0098 & 33.63\\
96 & 45 & 0.0164 & 41.19 & 95 & 77 & 0.0278 & 53.44 & 95 & 27 & 0.0095 & 33.24\\
96 & 44 & 0.0161 & 40.77 & 95 & 76 & 0.0274 & 53.02 & 95 & 26 & 0.0091 & 32.85\\
96 & 43 & 0.0157 & 40.36 & 95 & 75 & 0.0270 & 52.61 & 95 & 25 & 0.0088 & 32.46\\
96 & 42 & 0.0153 & 39.95 & 95 & 74 & 0.0266 & 52.19 & 95 & 24 & 0.0084 & 32.06\\
96 & 41 & 0.0149 & 39.54 & 95 & 73 & 0.0263 & 51.78 & 95 & 23 & 0.0080 & 31.67\\
96 & 40 & 0.0146 & 39.12 & 95 & 72 & 0.0259 & 51.36 & 95 & 22 & 0.0077 & 31.28\\
96 & 39 & 0.0142 & 38.71 & 95 & 71 & 0.0255 & 50.95 & 95 & 21 & 0.0073 & 30.89\\
96 & 38 & 0.0138 & 38.30 & 95 & 70 & 0.0251 & 50.54 & 95 & 20 & 0.0070 & 30.50\\
96 & 37 & 0.0135 & 37.89 & 95 & 69 & 0.0248 & 50.13 & 95 & 19 & 0.0066 & 30.11\\
96 & 36 & 0.0131 & 37.48 & 95 & 68 & 0.0244 & 49.71 & 95 & 18 & 0.0063 & 29.72\\
96 & 35 & 0.0127 & 37.07 & 95 & 67 & 0.0240 & 49.30 & 95 & 17 & 0.0059 & 29.34\\
96 & 34 & 0.0123 & 36.66 & 95 & 66 & 0.0237 & 48.89 & 95 & 16 & 0.0056 & 28.95\\
96 & 33 & 0.0120 & 36.25 & 95 & 65 & 0.0233 & 48.48 & 95 & 15 & 0.0052 & 28.56\\
96 & 32 & 0.0116 & 35.85 & 95 & 64 & 0.0229 & 48.07 & 95 & 14 & 0.0049 & 28.17\\
96 & 31 & 0.0112 & 35.44 & 95 & 63 & 0.0225 & 47.66 & 95 & 13 & 0.0045 & 27.79\\
96 & 30 & 0.0109 & 35.03 & 95 & 62 & 0.0222 & 47.26 & 95 & 12 & 0.0042 & 27.40\\
96 & 29 & 0.0105 & 34.63 & 95 & 61 & 0.0218 & 46.85 & 95 & 11 & 0.0038 & 27.02\\
96 & 28 & 0.0101 & 34.22 & 95 & 60 & 0.0214 & 46.44 & 94 & 96 & 0.0339 & 59.93\\
96 & 27 & 0.0098 & 33.81 & 95 & 59 & 0.0211 & 46.03 & 94 & 95 & 0.0335 & 59.52\\
96 & 26 & 0.0094 & 33.41 & 95 & 58 & 0.0207 & 45.62 & 94 & 94 & 0.0331 & 59.11\\
96 & 25 & 0.0090 & 33.00 & 95 & 57 & 0.0203 & 45.22 & 94 & 93 & 0.0328 & 58.70\\
96 & 24 & 0.0087 & 32.60 & 95 & 56 & 0.0200 & 44.81 & 94 & 92 & 0.0324 & 58.29\\
96 & 23 & 0.0083 & 32.20 & 95 & 55 & 0.0196 & 44.41 & 94 & 91 & 0.0320 & 57.88\\
\bottomrule
\end{tabular}
\newpage
\begin{tabular}{llll|llll|llll}
 \toprule 
\(T_{db}\) & \(\phi\) & \(\omega\) & \(h\) & \(T_{db}\) & \(\phi\) & \(\omega\) & \(h\) & \(T_{db}\) & \(\phi\) & \(\omega\) & \(h\)  \\ \midrule 
94 & 90 & 0.0317 & 57.48 & 94 & 40 & 0.0137 & 37.65 & 93 & 79 & 0.0267 & 51.80\\
94 & 89 & 0.0313 & 57.07 & 94 & 39 & 0.0133 & 37.27 & 93 & 78 & 0.0264 & 51.41\\
94 & 88 & 0.0309 & 56.66 & 94 & 38 & 0.0130 & 36.88 & 93 & 77 & 0.0260 & 51.02\\
94 & 87 & 0.0306 & 56.26 & 94 & 37 & 0.0126 & 36.50 & 93 & 76 & 0.0257 & 50.63\\
94 & 86 & 0.0302 & 55.85 & 94 & 36 & 0.0123 & 36.11 & 93 & 75 & 0.0253 & 50.25\\
94 & 85 & 0.0298 & 55.44 & 94 & 35 & 0.0119 & 35.73 & 93 & 74 & 0.0250 & 49.86\\
94 & 84 & 0.0295 & 55.04 & 94 & 34 & 0.0116 & 35.35 & 93 & 73 & 0.0246 & 49.47\\
94 & 83 & 0.0291 & 54.63 & 94 & 33 & 0.0112 & 34.96 & 93 & 72 & 0.0243 & 49.09\\
94 & 82 & 0.0287 & 54.23 & 94 & 32 & 0.0109 & 34.58 & 93 & 71 & 0.0239 & 48.70\\
94 & 81 & 0.0284 & 53.83 & 94 & 31 & 0.0106 & 34.20 & 93 & 70 & 0.0236 & 48.31\\
94 & 80 & 0.0280 & 53.42 & 94 & 30 & 0.0102 & 33.82 & 93 & 69 & 0.0232 & 47.93\\
94 & 79 & 0.0276 & 53.02 & 94 & 29 & 0.0099 & 33.44 & 93 & 68 & 0.0229 & 47.54\\
94 & 78 & 0.0273 & 52.62 & 94 & 28 & 0.0095 & 33.05 & 93 & 67 & 0.0225 & 47.16\\
94 & 77 & 0.0269 & 52.21 & 94 & 27 & 0.0092 & 32.67 & 93 & 66 & 0.0222 & 46.78\\
94 & 76 & 0.0265 & 51.81 & 94 & 26 & 0.0088 & 32.29 & 93 & 65 & 0.0218 & 46.39\\
94 & 75 & 0.0262 & 51.41 & 94 & 25 & 0.0085 & 31.92 & 93 & 64 & 0.0215 & 46.01\\
94 & 74 & 0.0258 & 51.01 & 94 & 24 & 0.0081 & 31.54 & 93 & 63 & 0.0211 & 45.63\\
94 & 73 & 0.0254 & 50.61 & 94 & 23 & 0.0078 & 31.16 & 93 & 62 & 0.0208 & 45.24\\
94 & 72 & 0.0251 & 50.21 & 94 & 22 & 0.0075 & 30.78 & 93 & 61 & 0.0204 & 44.86\\
94 & 71 & 0.0247 & 49.81 & 94 & 21 & 0.0071 & 30.40 & 93 & 60 & 0.0201 & 44.48\\
94 & 70 & 0.0243 & 49.41 & 94 & 20 & 0.0068 & 30.02 & 93 & 59 & 0.0198 & 44.10\\
94 & 69 & 0.0240 & 49.02 & 94 & 19 & 0.0064 & 29.65 & 93 & 58 & 0.0194 & 43.72\\
94 & 68 & 0.0236 & 48.62 & 94 & 18 & 0.0061 & 29.27 & 93 & 57 & 0.0191 & 43.34\\
94 & 67 & 0.0233 & 48.22 & 94 & 17 & 0.0057 & 28.89 & 93 & 56 & 0.0187 & 42.96\\
94 & 66 & 0.0229 & 47.82 & 94 & 16 & 0.0054 & 28.52 & 93 & 55 & 0.0184 & 42.58\\
94 & 65 & 0.0225 & 47.43 & 94 & 15 & 0.0051 & 28.14 & 93 & 54 & 0.0180 & 42.20\\
94 & 64 & 0.0222 & 47.03 & 94 & 14 & 0.0047 & 27.77 & 93 & 53 & 0.0177 & 41.82\\
94 & 63 & 0.0218 & 46.63 & 94 & 13 & 0.0044 & 27.39 & 93 & 52 & 0.0173 & 41.44\\
94 & 62 & 0.0215 & 46.24 & 94 & 12 & 0.0040 & 27.02 & 93 & 51 & 0.0170 & 41.07\\
94 & 61 & 0.0211 & 45.84 & 94 & 11 & 0.0037 & 26.65 & 93 & 50 & 0.0167 & 40.69\\
94 & 60 & 0.0208 & 45.45 & 93 & 99 & 0.0339 & 59.67 & 93 & 49 & 0.0163 & 40.31\\
94 & 59 & 0.0204 & 45.06 & 93 & 98 & 0.0335 & 59.27 & 93 & 48 & 0.0160 & 39.93\\
94 & 58 & 0.0200 & 44.66 & 93 & 97 & 0.0332 & 58.88 & 93 & 47 & 0.0156 & 39.56\\
94 & 57 & 0.0197 & 44.27 & 93 & 96 & 0.0328 & 58.48 & 93 & 46 & 0.0153 & 39.18\\
94 & 56 & 0.0193 & 43.88 & 93 & 95 & 0.0324 & 58.08 & 93 & 45 & 0.0150 & 38.81\\
94 & 55 & 0.0190 & 43.48 & 93 & 94 & 0.0321 & 57.69 & 93 & 44 & 0.0146 & 38.43\\
94 & 54 & 0.0186 & 43.09 & 93 & 93 & 0.0317 & 57.29 & 93 & 43 & 0.0143 & 38.06\\
94 & 53 & 0.0183 & 42.70 & 93 & 92 & 0.0314 & 56.90 & 93 & 42 & 0.0139 & 37.68\\
94 & 52 & 0.0179 & 42.31 & 93 & 91 & 0.0310 & 56.50 & 93 & 41 & 0.0136 & 37.31\\
94 & 51 & 0.0176 & 41.92 & 93 & 90 & 0.0306 & 56.11 & 93 & 40 & 0.0133 & 36.94\\
94 & 50 & 0.0172 & 41.53 & 93 & 89 & 0.0303 & 55.71 & 93 & 39 & 0.0129 & 36.56\\
94 & 49 & 0.0168 & 41.14 & 93 & 88 & 0.0299 & 55.32 & 93 & 38 & 0.0126 & 36.19\\
94 & 48 & 0.0165 & 40.75 & 93 & 87 & 0.0296 & 54.93 & 93 & 37 & 0.0122 & 35.82\\
94 & 47 & 0.0161 & 40.36 & 93 & 86 & 0.0292 & 54.54 & 93 & 36 & 0.0119 & 35.45\\
94 & 46 & 0.0158 & 39.97 & 93 & 85 & 0.0289 & 54.14 & 93 & 35 & 0.0116 & 35.08\\
94 & 45 & 0.0154 & 39.58 & 93 & 84 & 0.0285 & 53.75 & 93 & 34 & 0.0112 & 34.70\\
94 & 44 & 0.0151 & 39.20 & 93 & 83 & 0.0282 & 53.36 & 93 & 33 & 0.0109 & 34.33\\
94 & 43 & 0.0147 & 38.81 & 93 & 82 & 0.0278 & 52.97 & 93 & 32 & 0.0106 & 33.96\\
94 & 42 & 0.0144 & 38.42 & 93 & 81 & 0.0274 & 52.58 & 93 & 31 & 0.0102 & 33.59\\
94 & 41 & 0.0140 & 38.04 & 93 & 80 & 0.0271 & 52.19 & 93 & 30 & 0.0099 & 33.22\\
\bottomrule
\end{tabular}
\newpage
\begin{tabular}{llll|llll|llll}
 \toprule 
\(T_{db}\) & \(\phi\) & \(\omega\) & \(h\) & \(T_{db}\) & \(\phi\) & \(\omega\) & \(h\) & \(T_{db}\) & \(\phi\) & \(\omega\) & \(h\)  \\ \midrule 
93 & 29 & 0.0096 & 32.86 & 92 & 69 & 0.0225 & 46.87 & 92 & 19 & 0.0060 & 28.73\\
93 & 28 & 0.0092 & 32.49 & 92 & 68 & 0.0222 & 46.50 & 92 & 18 & 0.0057 & 28.38\\
93 & 27 & 0.0089 & 32.12 & 92 & 67 & 0.0218 & 46.12 & 92 & 17 & 0.0054 & 28.03\\
93 & 26 & 0.0086 & 31.75 & 92 & 66 & 0.0215 & 45.75 & 92 & 16 & 0.0051 & 27.67\\
93 & 25 & 0.0082 & 31.38 & 92 & 65 & 0.0211 & 45.38 & 92 & 15 & 0.0048 & 27.32\\
93 & 24 & 0.0079 & 31.02 & 92 & 64 & 0.0208 & 45.01 & 92 & 14 & 0.0044 & 26.97\\
93 & 23 & 0.0076 & 30.65 & 92 & 63 & 0.0205 & 44.64 & 92 & 13 & 0.0041 & 26.62\\
93 & 22 & 0.0072 & 30.28 & 92 & 62 & 0.0201 & 44.27 & 92 & 12 & 0.0038 & 26.27\\
93 & 21 & 0.0069 & 29.92 & 92 & 61 & 0.0198 & 43.90 & 92 & 11 & 0.0035 & 25.92\\
93 & 20 & 0.0066 & 29.55 & 92 & 60 & 0.0195 & 43.53 & 91 & 100 & 0.0321 & 57.17\\
93 & 19 & 0.0062 & 29.19 & 92 & 59 & 0.0191 & 43.17 & 91 & 99 & 0.0317 & 56.79\\
93 & 18 & 0.0059 & 28.82 & 92 & 58 & 0.0188 & 42.80 & 91 & 98 & 0.0314 & 56.42\\
93 & 17 & 0.0056 & 28.46 & 92 & 57 & 0.0185 & 42.43 & 91 & 97 & 0.0311 & 56.05\\
93 & 16 & 0.0052 & 28.09 & 92 & 56 & 0.0181 & 42.06 & 91 & 96 & 0.0307 & 55.68\\
93 & 15 & 0.0049 & 27.73 & 92 & 55 & 0.0178 & 41.69 & 91 & 95 & 0.0304 & 55.31\\
93 & 14 & 0.0046 & 27.37 & 92 & 54 & 0.0175 & 41.33 & 91 & 94 & 0.0301 & 54.94\\
93 & 13 & 0.0042 & 27.00 & 92 & 53 & 0.0171 & 40.96 & 91 & 93 & 0.0297 & 54.57\\
93 & 12 & 0.0039 & 26.64 & 92 & 52 & 0.0168 & 40.60 & 91 & 92 & 0.0294 & 54.21\\
93 & 11 & 0.0036 & 26.28 & 92 & 51 & 0.0165 & 40.23 & 91 & 91 & 0.0290 & 53.84\\
92 & 100 & 0.0331 & 58.60 & 92 & 50 & 0.0161 & 39.87 & 91 & 90 & 0.0287 & 53.47\\
92 & 99 & 0.0328 & 58.21 & 92 & 49 & 0.0158 & 39.50 & 91 & 89 & 0.0284 & 53.10\\
92 & 98 & 0.0324 & 57.83 & 92 & 48 & 0.0155 & 39.14 & 91 & 88 & 0.0280 & 52.74\\
92 & 97 & 0.0321 & 57.45 & 92 & 47 & 0.0151 & 38.77 & 91 & 87 & 0.0277 & 52.37\\
92 & 96 & 0.0317 & 57.06 & 92 & 46 & 0.0148 & 38.41 & 91 & 86 & 0.0274 & 52.00\\
92 & 95 & 0.0314 & 56.68 & 92 & 45 & 0.0145 & 38.05 & 91 & 85 & 0.0271 & 51.64\\
92 & 94 & 0.0311 & 56.30 & 92 & 44 & 0.0142 & 37.68 & 91 & 84 & 0.0267 & 51.27\\
92 & 93 & 0.0307 & 55.92 & 92 & 43 & 0.0138 & 37.32 & 91 & 83 & 0.0264 & 50.91\\
92 & 92 & 0.0304 & 55.53 & 92 & 42 & 0.0135 & 36.96 & 91 & 82 & 0.0261 & 50.54\\
92 & 91 & 0.0300 & 55.15 & 92 & 41 & 0.0132 & 36.60 & 91 & 81 & 0.0257 & 50.18\\
92 & 90 & 0.0297 & 54.77 & 92 & 40 & 0.0128 & 36.23 & 91 & 80 & 0.0254 & 49.81\\
92 & 89 & 0.0293 & 54.39 & 92 & 39 & 0.0125 & 35.87 & 91 & 79 & 0.0251 & 49.45\\
92 & 88 & 0.0290 & 54.01 & 92 & 38 & 0.0122 & 35.51 & 91 & 78 & 0.0247 & 49.08\\
92 & 87 & 0.0286 & 53.63 & 92 & 37 & 0.0119 & 35.15 & 91 & 77 & 0.0244 & 48.72\\
92 & 86 & 0.0283 & 53.25 & 92 & 36 & 0.0115 & 34.79 & 91 & 76 & 0.0241 & 48.36\\
92 & 85 & 0.0279 & 52.87 & 92 & 35 & 0.0112 & 34.43 & 91 & 75 & 0.0237 & 48.00\\
92 & 84 & 0.0276 & 52.50 & 92 & 34 & 0.0109 & 34.07 & 91 & 74 & 0.0234 & 47.63\\
92 & 83 & 0.0273 & 52.12 & 92 & 33 & 0.0106 & 33.72 & 91 & 73 & 0.0231 & 47.27\\
92 & 82 & 0.0269 & 51.74 & 92 & 32 & 0.0102 & 33.36 & 91 & 72 & 0.0228 & 46.91\\
92 & 81 & 0.0266 & 51.36 & 92 & 31 & 0.0099 & 33.00 & 91 & 71 & 0.0224 & 46.55\\
92 & 80 & 0.0262 & 50.99 & 92 & 30 & 0.0096 & 32.64 & 91 & 70 & 0.0221 & 46.19\\
92 & 79 & 0.0259 & 50.61 & 92 & 29 & 0.0093 & 32.28 & 91 & 69 & 0.0218 & 45.83\\
92 & 78 & 0.0255 & 50.23 & 92 & 28 & 0.0089 & 31.93 & 91 & 68 & 0.0215 & 45.47\\
92 & 77 & 0.0252 & 49.86 & 92 & 27 & 0.0086 & 31.57 & 91 & 67 & 0.0211 & 45.11\\
92 & 76 & 0.0249 & 49.48 & 92 & 26 & 0.0083 & 31.21 & 91 & 66 & 0.0208 & 44.75\\
92 & 75 & 0.0245 & 49.11 & 92 & 25 & 0.0080 & 30.86 & 91 & 65 & 0.0205 & 44.39\\
92 & 74 & 0.0242 & 48.73 & 92 & 24 & 0.0076 & 30.50 & 91 & 64 & 0.0202 & 44.04\\
92 & 73 & 0.0238 & 48.36 & 92 & 23 & 0.0073 & 30.15 & 91 & 63 & 0.0198 & 43.68\\
92 & 72 & 0.0235 & 47.99 & 92 & 22 & 0.0070 & 29.79 & 91 & 62 & 0.0195 & 43.32\\
92 & 71 & 0.0232 & 47.61 & 92 & 21 & 0.0067 & 29.44 & 91 & 61 & 0.0192 & 42.96\\
92 & 70 & 0.0228 & 47.24 & 92 & 20 & 0.0064 & 29.09 & 91 & 60 & 0.0189 & 42.61\\
\bottomrule
\end{tabular}
\newpage
\begin{tabular}{llll|llll|llll}
 \toprule 
\(T_{db}\) & \(\phi\) & \(\omega\) & \(h\) & \(T_{db}\) & \(\phi\) & \(\omega\) & \(h\) & \(T_{db}\) & \(\phi\) & \(\omega\) & \(h\)  \\ \midrule 
91 & 59 & 0.0185 & 42.25 & 90 & 99 & 0.0307 & 55.41 & 90 & 49 & 0.0148 & 37.93\\
91 & 58 & 0.0182 & 41.89 & 90 & 98 & 0.0304 & 55.05 & 90 & 48 & 0.0145 & 37.59\\
91 & 57 & 0.0179 & 41.54 & 90 & 97 & 0.0301 & 54.69 & 90 & 47 & 0.0142 & 37.25\\
91 & 56 & 0.0176 & 41.18 & 90 & 96 & 0.0297 & 54.34 & 90 & 46 & 0.0139 & 36.91\\
91 & 55 & 0.0172 & 40.83 & 90 & 95 & 0.0294 & 53.98 & 90 & 45 & 0.0136 & 36.57\\
91 & 54 & 0.0169 & 40.47 & 90 & 94 & 0.0291 & 53.62 & 90 & 44 & 0.0133 & 36.23\\
91 & 53 & 0.0166 & 40.12 & 90 & 93 & 0.0288 & 53.27 & 90 & 43 & 0.0130 & 35.89\\
91 & 52 & 0.0163 & 39.77 & 90 & 92 & 0.0284 & 52.91 & 90 & 42 & 0.0127 & 35.55\\
91 & 51 & 0.0160 & 39.41 & 90 & 91 & 0.0281 & 52.55 & 90 & 41 & 0.0124 & 35.21\\
91 & 50 & 0.0156 & 39.06 & 90 & 90 & 0.0278 & 52.20 & 90 & 40 & 0.0121 & 34.87\\
91 & 49 & 0.0153 & 38.71 & 90 & 89 & 0.0275 & 51.84 & 90 & 39 & 0.0117 & 34.53\\
91 & 48 & 0.0150 & 38.35 & 90 & 88 & 0.0271 & 51.49 & 90 & 38 & 0.0114 & 34.19\\
91 & 47 & 0.0147 & 38.00 & 90 & 87 & 0.0268 & 51.14 & 90 & 37 & 0.0111 & 33.86\\
91 & 46 & 0.0144 & 37.65 & 90 & 86 & 0.0265 & 50.78 & 90 & 36 & 0.0108 & 33.52\\
91 & 45 & 0.0140 & 37.30 & 90 & 85 & 0.0262 & 50.43 & 90 & 35 & 0.0105 & 33.18\\
91 & 44 & 0.0137 & 36.95 & 90 & 84 & 0.0259 & 50.07 & 90 & 34 & 0.0102 & 32.85\\
91 & 43 & 0.0134 & 36.60 & 90 & 83 & 0.0255 & 49.72 & 90 & 33 & 0.0099 & 32.51\\
91 & 42 & 0.0131 & 36.25 & 90 & 82 & 0.0252 & 49.37 & 90 & 32 & 0.0096 & 32.18\\
91 & 41 & 0.0128 & 35.90 & 90 & 81 & 0.0249 & 49.02 & 90 & 31 & 0.0093 & 31.84\\
91 & 40 & 0.0124 & 35.55 & 90 & 80 & 0.0246 & 48.66 & 90 & 30 & 0.0090 & 31.50\\
91 & 39 & 0.0121 & 35.20 & 90 & 79 & 0.0243 & 48.31 & 90 & 29 & 0.0087 & 31.17\\
91 & 38 & 0.0118 & 34.85 & 90 & 78 & 0.0239 & 47.96 & 90 & 28 & 0.0084 & 30.84\\
91 & 37 & 0.0115 & 34.50 & 90 & 77 & 0.0236 & 47.61 & 90 & 27 & 0.0081 & 30.50\\
91 & 36 & 0.0112 & 34.15 & 90 & 76 & 0.0233 & 47.26 & 90 & 26 & 0.0078 & 30.17\\
91 & 35 & 0.0109 & 33.80 & 90 & 75 & 0.0230 & 46.91 & 90 & 25 & 0.0075 & 29.83\\
91 & 34 & 0.0105 & 33.46 & 90 & 74 & 0.0227 & 46.56 & 90 & 24 & 0.0072 & 29.50\\
91 & 33 & 0.0102 & 33.11 & 90 & 73 & 0.0224 & 46.21 & 90 & 23 & 0.0069 & 29.17\\
91 & 32 & 0.0099 & 32.76 & 90 & 72 & 0.0220 & 45.86 & 90 & 22 & 0.0066 & 28.84\\
91 & 31 & 0.0096 & 32.41 & 90 & 71 & 0.0217 & 45.51 & 90 & 21 & 0.0063 & 28.50\\
91 & 30 & 0.0093 & 32.07 & 90 & 70 & 0.0214 & 45.17 & 90 & 20 & 0.0060 & 28.17\\
91 & 29 & 0.0090 & 31.72 & 90 & 69 & 0.0211 & 44.82 & 90 & 19 & 0.0057 & 27.84\\
91 & 28 & 0.0087 & 31.38 & 90 & 68 & 0.0208 & 44.47 & 90 & 18 & 0.0054 & 27.51\\
91 & 27 & 0.0083 & 31.03 & 90 & 67 & 0.0205 & 44.12 & 90 & 17 & 0.0051 & 27.18\\
91 & 26 & 0.0080 & 30.69 & 90 & 66 & 0.0201 & 43.78 & 90 & 16 & 0.0048 & 26.85\\
91 & 25 & 0.0077 & 30.34 & 90 & 65 & 0.0198 & 43.43 & 90 & 15 & 0.0045 & 26.52\\
91 & 24 & 0.0074 & 30.00 & 90 & 64 & 0.0195 & 43.08 & 90 & 14 & 0.0042 & 26.19\\
91 & 23 & 0.0071 & 29.65 & 90 & 63 & 0.0192 & 42.74 & 90 & 13 & 0.0039 & 25.86\\
91 & 22 & 0.0068 & 29.31 & 90 & 62 & 0.0189 & 42.39 & 90 & 12 & 0.0036 & 25.53\\
91 & 21 & 0.0065 & 28.97 & 90 & 61 & 0.0186 & 42.05 & 90 & 11 & 0.0033 & 25.20\\
91 & 20 & 0.0062 & 28.63 & 90 & 60 & 0.0183 & 41.70 & 89 & 100 & 0.0300 & 54.41\\
91 & 19 & 0.0058 & 28.28 & 90 & 59 & 0.0179 & 41.36 & 89 & 99 & 0.0297 & 54.06\\
91 & 18 & 0.0055 & 27.94 & 90 & 58 & 0.0176 & 41.01 & 89 & 98 & 0.0294 & 53.71\\
91 & 17 & 0.0052 & 27.60 & 90 & 57 & 0.0173 & 40.67 & 89 & 97 & 0.0291 & 53.37\\
91 & 16 & 0.0049 & 27.26 & 90 & 56 & 0.0170 & 40.32 & 89 & 96 & 0.0288 & 53.02\\
91 & 15 & 0.0046 & 26.92 & 90 & 55 & 0.0167 & 39.98 & 89 & 95 & 0.0285 & 52.68\\
91 & 14 & 0.0043 & 26.58 & 90 & 54 & 0.0164 & 39.64 & 89 & 94 & 0.0281 & 52.33\\
91 & 13 & 0.0040 & 26.24 & 90 & 53 & 0.0161 & 39.29 & 89 & 93 & 0.0278 & 51.99\\
91 & 12 & 0.0037 & 25.90 & 90 & 52 & 0.0158 & 38.95 & 89 & 92 & 0.0275 & 51.65\\
91 & 11 & 0.0034 & 25.56 & 90 & 51 & 0.0154 & 38.61 & 89 & 91 & 0.0272 & 51.30\\
90 & 100 & 0.0310 & 55.77 & 90 & 50 & 0.0151 & 38.27 & 89 & 90 & 0.0269 & 50.96\\
\bottomrule
\end{tabular}
\newpage
\begin{tabular}{llll|llll|llll}
 \toprule 
\(T_{db}\) & \(\phi\) & \(\omega\) & \(h\) & \(T_{db}\) & \(\phi\) & \(\omega\) & \(h\) & \(T_{db}\) & \(\phi\) & \(\omega\) & \(h\)  \\ \midrule 
89 & 89 & 0.0266 & 50.62 & 89 & 39 & 0.0114 & 33.88 & 88 & 79 & 0.0227 & 46.12\\
89 & 88 & 0.0263 & 50.27 & 89 & 38 & 0.0111 & 33.55 & 88 & 78 & 0.0224 & 45.79\\
89 & 87 & 0.0260 & 49.93 & 89 & 37 & 0.0108 & 33.23 & 88 & 77 & 0.0221 & 45.47\\
89 & 86 & 0.0256 & 49.59 & 89 & 36 & 0.0105 & 32.90 & 88 & 76 & 0.0218 & 45.14\\
89 & 85 & 0.0253 & 49.25 & 89 & 35 & 0.0102 & 32.57 & 88 & 75 & 0.0215 & 44.81\\
89 & 84 & 0.0250 & 48.91 & 89 & 34 & 0.0099 & 32.25 & 88 & 74 & 0.0212 & 44.49\\
89 & 83 & 0.0247 & 48.57 & 89 & 33 & 0.0096 & 31.92 & 88 & 73 & 0.0209 & 44.16\\
89 & 82 & 0.0244 & 48.23 & 89 & 32 & 0.0093 & 31.60 & 88 & 72 & 0.0206 & 43.83\\
89 & 81 & 0.0241 & 47.88 & 89 & 31 & 0.0090 & 31.27 & 88 & 71 & 0.0203 & 43.51\\
89 & 80 & 0.0238 & 47.54 & 89 & 30 & 0.0087 & 30.95 & 88 & 70 & 0.0201 & 43.18\\
89 & 79 & 0.0235 & 47.21 & 89 & 29 & 0.0084 & 30.63 & 88 & 69 & 0.0198 & 42.86\\
89 & 78 & 0.0232 & 46.87 & 89 & 28 & 0.0081 & 30.30 & 88 & 68 & 0.0195 & 42.53\\
89 & 77 & 0.0229 & 46.53 & 89 & 27 & 0.0078 & 29.98 & 88 & 67 & 0.0192 & 42.21\\
89 & 76 & 0.0226 & 46.19 & 89 & 26 & 0.0075 & 29.66 & 88 & 66 & 0.0189 & 41.88\\
89 & 75 & 0.0223 & 45.85 & 89 & 25 & 0.0072 & 29.33 & 88 & 65 & 0.0186 & 41.56\\
89 & 74 & 0.0219 & 45.51 & 89 & 24 & 0.0070 & 29.01 & 88 & 64 & 0.0183 & 41.24\\
89 & 73 & 0.0216 & 45.17 & 89 & 23 & 0.0067 & 28.69 & 88 & 63 & 0.0180 & 40.91\\
89 & 72 & 0.0213 & 44.84 & 89 & 22 & 0.0064 & 28.37 & 88 & 62 & 0.0177 & 40.59\\
89 & 71 & 0.0210 & 44.50 & 89 & 21 & 0.0061 & 28.04 & 88 & 61 & 0.0174 & 40.27\\
89 & 70 & 0.0207 & 44.16 & 89 & 20 & 0.0058 & 27.72 & 88 & 60 & 0.0171 & 39.94\\
89 & 69 & 0.0204 & 43.83 & 89 & 19 & 0.0055 & 27.40 & 88 & 59 & 0.0168 & 39.62\\
89 & 68 & 0.0201 & 43.49 & 89 & 18 & 0.0052 & 27.08 & 88 & 58 & 0.0165 & 39.30\\
89 & 67 & 0.0198 & 43.15 & 89 & 17 & 0.0049 & 26.76 & 88 & 57 & 0.0162 & 38.98\\
89 & 66 & 0.0195 & 42.82 & 89 & 16 & 0.0046 & 26.44 & 88 & 56 & 0.0159 & 38.66\\
89 & 65 & 0.0192 & 42.48 & 89 & 15 & 0.0043 & 26.12 & 88 & 55 & 0.0156 & 38.34\\
89 & 64 & 0.0189 & 42.15 & 89 & 14 & 0.0040 & 25.80 & 88 & 54 & 0.0154 & 38.02\\
89 & 63 & 0.0186 & 41.81 & 89 & 13 & 0.0037 & 25.48 & 88 & 53 & 0.0151 & 37.69\\
89 & 62 & 0.0183 & 41.48 & 89 & 12 & 0.0035 & 25.16 & 88 & 52 & 0.0148 & 37.37\\
89 & 61 & 0.0180 & 41.15 & 89 & 11 & 0.0032 & 24.85 & 88 & 51 & 0.0145 & 37.05\\
89 & 60 & 0.0177 & 40.81 & 88 & 100 & 0.0290 & 53.08 & 88 & 50 & 0.0142 & 36.74\\
89 & 59 & 0.0174 & 40.48 & 88 & 99 & 0.0287 & 52.75 & 88 & 49 & 0.0139 & 36.42\\
89 & 58 & 0.0171 & 40.15 & 88 & 98 & 0.0284 & 52.41 & 88 & 48 & 0.0136 & 36.10\\
89 & 57 & 0.0168 & 39.81 & 88 & 97 & 0.0281 & 52.08 & 88 & 47 & 0.0133 & 35.78\\
89 & 56 & 0.0165 & 39.48 & 88 & 96 & 0.0278 & 51.74 & 88 & 46 & 0.0130 & 35.46\\
89 & 55 & 0.0162 & 39.15 & 88 & 95 & 0.0275 & 51.41 & 88 & 45 & 0.0127 & 35.14\\
89 & 54 & 0.0159 & 38.82 & 88 & 94 & 0.0272 & 51.08 & 88 & 44 & 0.0125 & 34.82\\
89 & 53 & 0.0156 & 38.49 & 88 & 93 & 0.0269 & 50.75 & 88 & 43 & 0.0122 & 34.51\\
89 & 52 & 0.0153 & 38.16 & 88 & 92 & 0.0266 & 50.41 & 88 & 42 & 0.0119 & 34.19\\
89 & 51 & 0.0150 & 37.82 & 88 & 91 & 0.0263 & 50.08 & 88 & 41 & 0.0116 & 33.87\\
89 & 50 & 0.0147 & 37.49 & 88 & 90 & 0.0260 & 49.75 & 88 & 40 & 0.0113 & 33.56\\
89 & 49 & 0.0144 & 37.16 & 88 & 89 & 0.0257 & 49.42 & 88 & 39 & 0.0110 & 33.24\\
89 & 48 & 0.0141 & 36.83 & 88 & 88 & 0.0254 & 49.09 & 88 & 38 & 0.0107 & 32.92\\
89 & 47 & 0.0138 & 36.50 & 88 & 87 & 0.0251 & 48.76 & 88 & 37 & 0.0104 & 32.61\\
89 & 46 & 0.0135 & 36.18 & 88 & 86 & 0.0248 & 48.43 & 88 & 36 & 0.0102 & 32.29\\
89 & 45 & 0.0132 & 35.85 & 88 & 85 & 0.0245 & 48.10 & 88 & 35 & 0.0099 & 31.98\\
89 & 44 & 0.0129 & 35.52 & 88 & 84 & 0.0242 & 47.77 & 88 & 34 & 0.0096 & 31.66\\
89 & 43 & 0.0126 & 35.19 & 88 & 83 & 0.0239 & 47.44 & 88 & 33 & 0.0093 & 31.35\\
89 & 42 & 0.0123 & 34.86 & 88 & 82 & 0.0236 & 47.11 & 88 & 32 & 0.0090 & 31.03\\
89 & 41 & 0.0120 & 34.53 & 88 & 81 & 0.0233 & 46.78 & 88 & 31 & 0.0087 & 30.72\\
89 & 40 & 0.0117 & 34.21 & 88 & 80 & 0.0230 & 46.45 & 88 & 30 & 0.0084 & 30.40\\
\bottomrule
\end{tabular}
\newpage
\begin{tabular}{llll|llll|llll}
 \toprule 
\(T_{db}\) & \(\phi\) & \(\omega\) & \(h\) & \(T_{db}\) & \(\phi\) & \(\omega\) & \(h\) & \(T_{db}\) & \(\phi\) & \(\omega\) & \(h\)  \\ \midrule 
88 & 29 & 0.0082 & 30.09 & 87 & 69 & 0.0191 & 41.91 & 87 & 19 & 0.0052 & 26.54\\
88 & 28 & 0.0079 & 29.78 & 87 & 68 & 0.0188 & 41.60 & 87 & 18 & 0.0049 & 26.24\\
88 & 27 & 0.0076 & 29.46 & 87 & 67 & 0.0186 & 41.28 & 87 & 17 & 0.0046 & 25.94\\
88 & 26 & 0.0073 & 29.15 & 87 & 66 & 0.0183 & 40.97 & 87 & 16 & 0.0043 & 25.64\\
88 & 25 & 0.0070 & 28.84 & 87 & 65 & 0.0180 & 40.66 & 87 & 15 & 0.0041 & 25.34\\
88 & 24 & 0.0067 & 28.53 & 87 & 64 & 0.0177 & 40.34 & 87 & 14 & 0.0038 & 25.04\\
88 & 23 & 0.0064 & 28.22 & 87 & 63 & 0.0174 & 40.03 & 87 & 13 & 0.0035 & 24.75\\
88 & 22 & 0.0062 & 27.90 & 87 & 62 & 0.0171 & 39.72 & 87 & 12 & 0.0032 & 24.45\\
88 & 21 & 0.0059 & 27.59 & 87 & 61 & 0.0168 & 39.41 & 87 & 11 & 0.0030 & 24.15\\
88 & 20 & 0.0056 & 27.28 & 87 & 60 & 0.0166 & 39.09 & 86 & 100 & 0.0272 & 50.52\\
88 & 19 & 0.0053 & 26.97 & 87 & 59 & 0.0163 & 38.78 & 86 & 99 & 0.0269 & 50.21\\
88 & 18 & 0.0050 & 26.66 & 87 & 58 & 0.0160 & 38.47 & 86 & 98 & 0.0266 & 49.90\\
88 & 17 & 0.0048 & 26.35 & 87 & 57 & 0.0157 & 38.16 & 86 & 97 & 0.0263 & 49.59\\
88 & 16 & 0.0045 & 26.04 & 87 & 56 & 0.0154 & 37.85 & 86 & 96 & 0.0261 & 49.28\\
88 & 15 & 0.0042 & 25.73 & 87 & 55 & 0.0151 & 37.54 & 86 & 95 & 0.0258 & 48.97\\
88 & 14 & 0.0039 & 25.42 & 87 & 54 & 0.0149 & 37.23 & 86 & 94 & 0.0255 & 48.66\\
88 & 13 & 0.0036 & 25.11 & 87 & 53 & 0.0146 & 36.92 & 86 & 93 & 0.0252 & 48.35\\
88 & 12 & 0.0033 & 24.80 & 87 & 52 & 0.0143 & 36.61 & 86 & 92 & 0.0249 & 48.04\\
88 & 11 & 0.0031 & 24.50 & 87 & 51 & 0.0140 & 36.30 & 86 & 91 & 0.0246 & 47.73\\
87 & 100 & 0.0281 & 51.78 & 87 & 50 & 0.0137 & 35.99 & 86 & 90 & 0.0244 & 47.42\\
87 & 99 & 0.0278 & 51.46 & 87 & 49 & 0.0135 & 35.68 & 86 & 89 & 0.0241 & 47.11\\
87 & 98 & 0.0275 & 51.14 & 87 & 48 & 0.0132 & 35.37 & 86 & 88 & 0.0238 & 46.80\\
87 & 97 & 0.0272 & 50.82 & 87 & 47 & 0.0129 & 35.07 & 86 & 87 & 0.0235 & 46.49\\
87 & 96 & 0.0269 & 50.50 & 87 & 46 & 0.0126 & 34.76 & 86 & 86 & 0.0232 & 46.18\\
87 & 95 & 0.0266 & 50.17 & 87 & 45 & 0.0123 & 34.45 & 86 & 85 & 0.0230 & 45.87\\
87 & 94 & 0.0263 & 49.85 & 87 & 44 & 0.0121 & 34.14 & 86 & 84 & 0.0227 & 45.57\\
87 & 93 & 0.0261 & 49.53 & 87 & 43 & 0.0118 & 33.84 & 86 & 83 & 0.0224 & 45.26\\
87 & 92 & 0.0258 & 49.21 & 87 & 42 & 0.0115 & 33.53 & 86 & 82 & 0.0221 & 44.95\\
87 & 91 & 0.0255 & 48.89 & 87 & 41 & 0.0112 & 33.22 & 86 & 81 & 0.0218 & 44.65\\
87 & 90 & 0.0252 & 48.57 & 87 & 40 & 0.0109 & 32.92 & 86 & 80 & 0.0216 & 44.34\\
87 & 89 & 0.0249 & 48.25 & 87 & 39 & 0.0107 & 32.61 & 86 & 79 & 0.0213 & 44.03\\
87 & 88 & 0.0246 & 47.93 & 87 & 38 & 0.0104 & 32.30 & 86 & 78 & 0.0210 & 43.73\\
87 & 87 & 0.0243 & 47.61 & 87 & 37 & 0.0101 & 32.00 & 86 & 77 & 0.0207 & 43.42\\
87 & 86 & 0.0240 & 47.29 & 87 & 36 & 0.0098 & 31.69 & 86 & 76 & 0.0204 & 43.11\\
87 & 85 & 0.0237 & 46.97 & 87 & 35 & 0.0096 & 31.39 & 86 & 75 & 0.0202 & 42.81\\
87 & 84 & 0.0234 & 46.65 & 87 & 34 & 0.0093 & 31.08 & 86 & 74 & 0.0199 & 42.50\\
87 & 83 & 0.0231 & 46.34 & 87 & 33 & 0.0090 & 30.78 & 86 & 73 & 0.0196 & 42.20\\
87 & 82 & 0.0229 & 46.02 & 87 & 32 & 0.0087 & 30.47 & 86 & 72 & 0.0193 & 41.90\\
87 & 81 & 0.0226 & 45.70 & 87 & 31 & 0.0084 & 30.17 & 86 & 71 & 0.0191 & 41.59\\
87 & 80 & 0.0223 & 45.38 & 87 & 30 & 0.0082 & 29.87 & 86 & 70 & 0.0188 & 41.29\\
87 & 79 & 0.0220 & 45.07 & 87 & 29 & 0.0079 & 29.56 & 86 & 69 & 0.0185 & 40.98\\
87 & 78 & 0.0217 & 44.75 & 87 & 28 & 0.0076 & 29.26 & 86 & 68 & 0.0182 & 40.68\\
87 & 77 & 0.0214 & 44.43 & 87 & 27 & 0.0073 & 28.96 & 86 & 67 & 0.0180 & 40.38\\
87 & 76 & 0.0211 & 44.12 & 87 & 26 & 0.0071 & 28.66 & 86 & 66 & 0.0177 & 40.07\\
87 & 75 & 0.0208 & 43.80 & 87 & 25 & 0.0068 & 28.35 & 86 & 65 & 0.0174 & 39.77\\
87 & 74 & 0.0206 & 43.48 & 87 & 24 & 0.0065 & 28.05 & 86 & 64 & 0.0171 & 39.47\\
87 & 73 & 0.0203 & 43.17 & 87 & 23 & 0.0062 & 27.75 & 86 & 63 & 0.0169 & 39.17\\
87 & 72 & 0.0200 & 42.85 & 87 & 22 & 0.0060 & 27.45 & 86 & 62 & 0.0166 & 38.86\\
87 & 71 & 0.0197 & 42.54 & 87 & 21 & 0.0057 & 27.15 & 86 & 61 & 0.0163 & 38.56\\
87 & 70 & 0.0194 & 42.22 & 87 & 20 & 0.0054 & 26.85 & 86 & 60 & 0.0160 & 38.26\\
\bottomrule
\end{tabular}
\newpage
\begin{tabular}{llll|llll|llll}
 \toprule 
\(T_{db}\) & \(\phi\) & \(\omega\) & \(h\) & \(T_{db}\) & \(\phi\) & \(\omega\) & \(h\) & \(T_{db}\) & \(\phi\) & \(\omega\) & \(h\)  \\ \midrule 
86 & 59 & 0.0158 & 37.96 & 85 & 99 & 0.0260 & 48.99 & 85 & 49 & 0.0126 & 34.26\\
86 & 58 & 0.0155 & 37.66 & 85 & 98 & 0.0257 & 48.69 & 85 & 48 & 0.0123 & 33.97\\
86 & 57 & 0.0152 & 37.36 & 85 & 97 & 0.0255 & 48.39 & 85 & 47 & 0.0121 & 33.68\\
86 & 56 & 0.0149 & 37.06 & 85 & 96 & 0.0252 & 48.09 & 85 & 46 & 0.0118 & 33.39\\
86 & 55 & 0.0147 & 36.76 & 85 & 95 & 0.0249 & 47.79 & 85 & 45 & 0.0116 & 33.11\\
86 & 54 & 0.0144 & 36.46 & 85 & 94 & 0.0247 & 47.49 & 85 & 44 & 0.0113 & 32.82\\
86 & 53 & 0.0141 & 36.16 & 85 & 93 & 0.0244 & 47.19 & 85 & 43 & 0.0110 & 32.53\\
86 & 52 & 0.0138 & 35.86 & 85 & 92 & 0.0241 & 46.89 & 85 & 42 & 0.0108 & 32.24\\
86 & 51 & 0.0136 & 35.56 & 85 & 91 & 0.0238 & 46.59 & 85 & 41 & 0.0105 & 31.96\\
86 & 50 & 0.0133 & 35.26 & 85 & 90 & 0.0236 & 46.29 & 85 & 40 & 0.0103 & 31.67\\
86 & 49 & 0.0130 & 34.96 & 85 & 89 & 0.0233 & 45.99 & 85 & 39 & 0.0100 & 31.38\\
86 & 48 & 0.0128 & 34.66 & 85 & 88 & 0.0230 & 45.70 & 85 & 38 & 0.0097 & 31.10\\
86 & 47 & 0.0125 & 34.37 & 85 & 87 & 0.0227 & 45.40 & 85 & 37 & 0.0095 & 30.81\\
86 & 46 & 0.0122 & 34.07 & 85 & 86 & 0.0225 & 45.10 & 85 & 36 & 0.0092 & 30.53\\
86 & 45 & 0.0119 & 33.77 & 85 & 85 & 0.0222 & 44.80 & 85 & 35 & 0.0090 & 30.24\\
86 & 44 & 0.0117 & 33.47 & 85 & 84 & 0.0219 & 44.51 & 85 & 34 & 0.0087 & 29.96\\
86 & 43 & 0.0114 & 33.18 & 85 & 83 & 0.0217 & 44.21 & 85 & 33 & 0.0084 & 29.67\\
86 & 42 & 0.0111 & 32.88 & 85 & 82 & 0.0214 & 43.91 & 85 & 32 & 0.0082 & 29.39\\
86 & 41 & 0.0109 & 32.58 & 85 & 81 & 0.0211 & 43.62 & 85 & 31 & 0.0079 & 29.10\\
86 & 40 & 0.0106 & 32.29 & 85 & 80 & 0.0209 & 43.32 & 85 & 30 & 0.0077 & 28.82\\
86 & 39 & 0.0103 & 31.99 & 85 & 79 & 0.0206 & 43.02 & 85 & 29 & 0.0074 & 28.53\\
86 & 38 & 0.0101 & 31.70 & 85 & 78 & 0.0203 & 42.73 & 85 & 28 & 0.0071 & 28.25\\
86 & 37 & 0.0098 & 31.40 & 85 & 77 & 0.0201 & 42.43 & 85 & 27 & 0.0069 & 27.97\\
86 & 36 & 0.0095 & 31.10 & 85 & 76 & 0.0198 & 42.14 & 85 & 26 & 0.0066 & 27.68\\
86 & 35 & 0.0093 & 30.81 & 85 & 75 & 0.0195 & 41.84 & 85 & 25 & 0.0064 & 27.40\\
86 & 34 & 0.0090 & 30.52 & 85 & 74 & 0.0192 & 41.55 & 85 & 24 & 0.0061 & 27.12\\
86 & 33 & 0.0087 & 30.22 & 85 & 73 & 0.0190 & 41.25 & 85 & 23 & 0.0059 & 26.84\\
86 & 32 & 0.0084 & 29.93 & 85 & 72 & 0.0187 & 40.96 & 85 & 22 & 0.0056 & 26.55\\
86 & 31 & 0.0082 & 29.63 & 85 & 71 & 0.0184 & 40.66 & 85 & 21 & 0.0053 & 26.27\\
86 & 30 & 0.0079 & 29.34 & 85 & 70 & 0.0182 & 40.37 & 85 & 20 & 0.0051 & 25.99\\
86 & 29 & 0.0076 & 29.05 & 85 & 69 & 0.0179 & 40.08 & 85 & 19 & 0.0048 & 25.71\\
86 & 28 & 0.0074 & 28.75 & 85 & 68 & 0.0176 & 39.78 & 85 & 18 & 0.0046 & 25.43\\
86 & 27 & 0.0071 & 28.46 & 85 & 67 & 0.0174 & 39.49 & 85 & 17 & 0.0043 & 25.15\\
86 & 26 & 0.0068 & 28.17 & 85 & 66 & 0.0171 & 39.20 & 85 & 16 & 0.0041 & 24.86\\
86 & 25 & 0.0066 & 27.87 & 85 & 65 & 0.0168 & 38.91 & 85 & 15 & 0.0038 & 24.58\\
86 & 24 & 0.0063 & 27.58 & 85 & 64 & 0.0166 & 38.61 & 85 & 14 & 0.0036 & 24.30\\
86 & 23 & 0.0060 & 27.29 & 85 & 63 & 0.0163 & 38.32 & 85 & 13 & 0.0033 & 24.02\\
86 & 22 & 0.0058 & 27.00 & 85 & 62 & 0.0160 & 38.03 & 85 & 12 & 0.0030 & 23.74\\
86 & 21 & 0.0055 & 26.71 & 85 & 61 & 0.0158 & 37.74 & 85 & 11 & 0.0028 & 23.46\\
86 & 20 & 0.0053 & 26.41 & 85 & 60 & 0.0155 & 37.45 & 84 & 100 & 0.0254 & 48.09\\
86 & 19 & 0.0050 & 26.12 & 85 & 59 & 0.0152 & 37.16 & 84 & 99 & 0.0252 & 47.80\\
86 & 18 & 0.0047 & 25.83 & 85 & 58 & 0.0150 & 36.86 & 84 & 98 & 0.0249 & 47.51\\
86 & 17 & 0.0045 & 25.54 & 85 & 57 & 0.0147 & 36.57 & 84 & 97 & 0.0246 & 47.22\\
86 & 16 & 0.0042 & 25.25 & 85 & 56 & 0.0145 & 36.28 & 84 & 96 & 0.0244 & 46.93\\
86 & 15 & 0.0039 & 24.96 & 85 & 55 & 0.0142 & 35.99 & 84 & 95 & 0.0241 & 46.64\\
86 & 14 & 0.0037 & 24.67 & 85 & 54 & 0.0139 & 35.70 & 84 & 94 & 0.0238 & 46.35\\
86 & 13 & 0.0034 & 24.38 & 85 & 53 & 0.0137 & 35.41 & 84 & 93 & 0.0236 & 46.06\\
86 & 12 & 0.0031 & 24.09 & 85 & 52 & 0.0134 & 35.12 & 84 & 92 & 0.0233 & 45.77\\
86 & 11 & 0.0029 & 23.80 & 85 & 51 & 0.0131 & 34.84 & 84 & 91 & 0.0231 & 45.48\\
85 & 100 & 0.0263 & 49.29 & 85 & 50 & 0.0129 & 34.55 & 84 & 90 & 0.0228 & 45.19\\
\bottomrule
\end{tabular}
\newpage
\begin{tabular}{llll|llll|llll}
 \toprule 
\(T_{db}\) & \(\phi\) & \(\omega\) & \(h\) & \(T_{db}\) & \(\phi\) & \(\omega\) & \(h\) & \(T_{db}\) & \(\phi\) & \(\omega\) & \(h\)  \\ \midrule 
84 & 89 & 0.0225 & 44.91 & 84 & 39 & 0.0097 & 30.79 & 83 & 68 & 0.0165 & 38.05\\
84 & 88 & 0.0223 & 44.62 & 84 & 38 & 0.0094 & 30.51 & 83 & 66 & 0.0160 & 37.50\\
84 & 87 & 0.0220 & 44.33 & 84 & 37 & 0.0092 & 30.23 & 83 & 64 & 0.0155 & 36.96\\
84 & 86 & 0.0217 & 44.04 & 84 & 36 & 0.0089 & 29.96 & 83 & 62 & 0.0150 & 36.41\\
84 & 85 & 0.0215 & 43.76 & 84 & 35 & 0.0087 & 29.68 & 83 & 60 & 0.0145 & 35.87\\
84 & 84 & 0.0212 & 43.47 & 84 & 34 & 0.0084 & 29.41 & 83 & 58 & 0.0140 & 35.32\\
84 & 83 & 0.0210 & 43.18 & 84 & 33 & 0.0082 & 29.13 & 83 & 56 & 0.0135 & 34.78\\
84 & 82 & 0.0207 & 42.90 & 84 & 32 & 0.0079 & 28.86 & 83 & 54 & 0.0130 & 34.24\\
84 & 81 & 0.0204 & 42.61 & 84 & 31 & 0.0077 & 28.58 & 83 & 52 & 0.0125 & 33.70\\
84 & 80 & 0.0202 & 42.32 & 84 & 30 & 0.0074 & 28.31 & 83 & 50 & 0.0121 & 33.16\\
84 & 79 & 0.0199 & 42.04 & 84 & 29 & 0.0072 & 28.03 & 83 & 48 & 0.0116 & 32.62\\
84 & 78 & 0.0197 & 41.75 & 84 & 27 & 0.0067 & 27.48 & 83 & 46 & 0.0111 & 32.08\\
84 & 77 & 0.0194 & 41.47 & 84 & 25 & 0.0062 & 26.93 & 83 & 44 & 0.0106 & 31.54\\
84 & 76 & 0.0191 & 41.18 & 84 & 23 & 0.0057 & 26.39 & 83 & 42 & 0.0101 & 31.01\\
84 & 75 & 0.0189 & 40.90 & 84 & 21 & 0.0052 & 25.84 & 83 & 40 & 0.0096 & 30.47\\
84 & 74 & 0.0186 & 40.61 & 84 & 19 & 0.0047 & 25.30 & 83 & 38 & 0.0091 & 29.93\\
84 & 73 & 0.0184 & 40.33 & 84 & 17 & 0.0042 & 24.75 & 83 & 36 & 0.0086 & 29.40\\
84 & 72 & 0.0181 & 40.04 & 84 & 15 & 0.0037 & 24.21 & 83 & 34 & 0.0081 & 28.87\\
84 & 71 & 0.0178 & 39.76 & 84 & 13 & 0.0032 & 23.67 & 83 & 32 & 0.0077 & 28.33\\
84 & 70 & 0.0176 & 39.47 & 84 & 11 & 0.0027 & 23.12 & 83 & 30 & 0.0072 & 27.80\\
84 & 69 & 0.0173 & 39.19 & 83 & 100 & 0.0246 & 46.92 & 83 & 28 & 0.0067 & 27.27\\
84 & 68 & 0.0171 & 38.91 & 83 & 99 & 0.0243 & 46.64 & 83 & 26 & 0.0062 & 26.74\\
84 & 67 & 0.0168 & 38.62 & 83 & 98 & 0.0241 & 46.36 & 83 & 24 & 0.0057 & 26.21\\
84 & 66 & 0.0166 & 38.34 & 83 & 97 & 0.0238 & 46.08 & 83 & 22 & 0.0052 & 25.68\\
84 & 65 & 0.0163 & 38.06 & 83 & 96 & 0.0236 & 45.80 & 83 & 20 & 0.0048 & 25.15\\
84 & 64 & 0.0160 & 37.78 & 83 & 95 & 0.0233 & 45.52 & 83 & 18 & 0.0043 & 24.63\\
84 & 63 & 0.0158 & 37.49 & 83 & 94 & 0.0231 & 45.24 & 83 & 16 & 0.0038 & 24.10\\
84 & 62 & 0.0155 & 37.21 & 83 & 93 & 0.0228 & 44.96 & 83 & 14 & 0.0033 & 23.58\\
84 & 61 & 0.0153 & 36.93 & 83 & 92 & 0.0226 & 44.68 & 83 & 12 & 0.0029 & 23.05\\
84 & 60 & 0.0150 & 36.65 & 83 & 91 & 0.0223 & 44.40 & 82 & 100 & 0.0238 & 45.78\\
84 & 59 & 0.0148 & 36.37 & 83 & 90 & 0.0220 & 44.12 & 82 & 98 & 0.0233 & 45.23\\
84 & 58 & 0.0145 & 36.09 & 83 & 89 & 0.0218 & 43.84 & 82 & 96 & 0.0228 & 44.69\\
84 & 57 & 0.0142 & 35.81 & 83 & 88 & 0.0215 & 43.57 & 82 & 94 & 0.0223 & 44.15\\
84 & 56 & 0.0140 & 35.52 & 83 & 87 & 0.0213 & 43.29 & 82 & 92 & 0.0218 & 43.61\\
84 & 55 & 0.0137 & 35.24 & 83 & 86 & 0.0210 & 43.01 & 82 & 90 & 0.0213 & 43.08\\
84 & 54 & 0.0135 & 34.96 & 83 & 85 & 0.0208 & 42.73 & 82 & 88 & 0.0208 & 42.54\\
84 & 53 & 0.0132 & 34.68 & 83 & 84 & 0.0205 & 42.46 & 82 & 86 & 0.0203 & 42.00\\
84 & 52 & 0.0130 & 34.40 & 83 & 83 & 0.0203 & 42.18 & 82 & 84 & 0.0199 & 41.47\\
84 & 51 & 0.0127 & 34.12 & 83 & 82 & 0.0200 & 41.90 & 82 & 82 & 0.0194 & 40.93\\
84 & 50 & 0.0125 & 33.85 & 83 & 81 & 0.0198 & 41.63 & 82 & 80 & 0.0189 & 40.40\\
84 & 49 & 0.0122 & 33.57 & 83 & 80 & 0.0195 & 41.35 & 82 & 78 & 0.0184 & 39.86\\
84 & 48 & 0.0120 & 33.29 & 83 & 79 & 0.0193 & 41.07 & 82 & 76 & 0.0179 & 39.33\\
84 & 47 & 0.0117 & 33.01 & 83 & 78 & 0.0190 & 40.80 & 82 & 74 & 0.0174 & 38.80\\
84 & 46 & 0.0114 & 32.73 & 83 & 77 & 0.0188 & 40.52 & 82 & 72 & 0.0169 & 38.27\\
84 & 45 & 0.0112 & 32.45 & 83 & 76 & 0.0185 & 40.25 & 82 & 70 & 0.0165 & 37.74\\
84 & 44 & 0.0109 & 32.17 & 83 & 75 & 0.0183 & 39.97 & 82 & 68 & 0.0160 & 37.21\\
84 & 43 & 0.0107 & 31.90 & 83 & 74 & 0.0180 & 39.70 & 82 & 66 & 0.0155 & 36.68\\
84 & 42 & 0.0104 & 31.62 & 83 & 73 & 0.0178 & 39.42 & 82 & 64 & 0.0150 & 36.15\\
84 & 41 & 0.0102 & 31.34 & 83 & 72 & 0.0175 & 39.15 & 82 & 62 & 0.0145 & 35.63\\
84 & 40 & 0.0099 & 31.06 & 83 & 70 & 0.0170 & 38.60 & 82 & 60 & 0.0141 & 35.10\\
\bottomrule
\end{tabular}
\newpage
\begin{tabular}{llll|llll|llll}
 \toprule 
\(T_{db}\) & \(\phi\) & \(\omega\) & \(h\) & \(T_{db}\) & \(\phi\) & \(\omega\) & \(h\) & \(T_{db}\) & \(\phi\) & \(\omega\) & \(h\)  \\ \midrule 
82 & 58 & 0.0136 & 34.58 & 81 & 48 & 0.0108 & 31.32 & 80 & 38 & 0.0083 & 28.26\\
82 & 56 & 0.0131 & 34.05 & 81 & 46 & 0.0104 & 30.81 & 80 & 36 & 0.0078 & 27.78\\
82 & 54 & 0.0126 & 33.53 & 81 & 44 & 0.0099 & 30.31 & 80 & 34 & 0.0074 & 27.30\\
82 & 52 & 0.0121 & 33.01 & 81 & 42 & 0.0095 & 29.81 & 80 & 32 & 0.0069 & 26.81\\
82 & 50 & 0.0117 & 32.48 & 81 & 40 & 0.0090 & 29.31 & 80 & 30 & 0.0065 & 26.33\\
82 & 48 & 0.0112 & 31.96 & 81 & 38 & 0.0085 & 28.81 & 80 & 28 & 0.0061 & 25.85\\
82 & 46 & 0.0107 & 31.44 & 81 & 36 & 0.0081 & 28.31 & 80 & 26 & 0.0056 & 25.37\\
82 & 44 & 0.0102 & 30.92 & 81 & 34 & 0.0076 & 27.81 & 80 & 24 & 0.0052 & 24.89\\
82 & 42 & 0.0098 & 30.40 & 81 & 32 & 0.0072 & 27.31 & 80 & 22 & 0.0048 & 24.42\\
82 & 40 & 0.0093 & 29.88 & 81 & 30 & 0.0067 & 26.82 & 80 & 20 & 0.0043 & 23.94\\
82 & 38 & 0.0088 & 29.37 & 81 & 28 & 0.0063 & 26.32 & 80 & 18 & 0.0039 & 23.46\\
82 & 36 & 0.0084 & 28.85 & 81 & 26 & 0.0058 & 25.82 & 80 & 16 & 0.0035 & 22.99\\
82 & 34 & 0.0079 & 28.33 & 81 & 24 & 0.0054 & 25.33 & 80 & 14 & 0.0030 & 22.51\\
82 & 32 & 0.0074 & 27.82 & 81 & 22 & 0.0049 & 24.83 & 80 & 12 & 0.0026 & 22.04\\
82 & 30 & 0.0069 & 27.30 & 81 & 20 & 0.0045 & 24.34 & 79 & 100 & 0.0215 & 42.51\\
82 & 28 & 0.0065 & 26.79 & 81 & 18 & 0.0040 & 23.85 & 79 & 98 & 0.0210 & 42.02\\
82 & 26 & 0.0060 & 26.28 & 81 & 16 & 0.0036 & 23.35 & 79 & 96 & 0.0206 & 41.54\\
82 & 24 & 0.0055 & 25.77 & 81 & 14 & 0.0031 & 22.86 & 79 & 94 & 0.0202 & 41.05\\
82 & 22 & 0.0051 & 25.25 & 81 & 12 & 0.0027 & 22.37 & 79 & 92 & 0.0197 & 40.57\\
82 & 20 & 0.0046 & 24.74 & 80 & 100 & 0.0222 & 43.57 & 79 & 90 & 0.0193 & 40.08\\
82 & 18 & 0.0041 & 24.23 & 80 & 98 & 0.0218 & 43.07 & 79 & 88 & 0.0188 & 39.60\\
82 & 16 & 0.0037 & 23.73 & 80 & 96 & 0.0213 & 42.56 & 79 & 86 & 0.0184 & 39.11\\
82 & 14 & 0.0032 & 23.22 & 80 & 94 & 0.0208 & 42.06 & 79 & 84 & 0.0179 & 38.63\\
82 & 12 & 0.0028 & 22.71 & 80 & 92 & 0.0204 & 41.56 & 79 & 82 & 0.0175 & 38.15\\
81 & 100 & 0.0230 & 44.66 & 80 & 90 & 0.0199 & 41.06 & 79 & 80 & 0.0171 & 37.67\\
81 & 98 & 0.0225 & 44.14 & 80 & 88 & 0.0195 & 40.56 & 79 & 78 & 0.0166 & 37.19\\
81 & 96 & 0.0220 & 43.62 & 80 & 86 & 0.0190 & 40.06 & 79 & 76 & 0.0162 & 36.71\\
81 & 94 & 0.0216 & 43.09 & 80 & 84 & 0.0186 & 39.56 & 79 & 74 & 0.0158 & 36.23\\
81 & 92 & 0.0211 & 42.57 & 80 & 82 & 0.0181 & 39.06 & 79 & 72 & 0.0153 & 35.75\\
81 & 90 & 0.0206 & 42.05 & 80 & 80 & 0.0177 & 38.56 & 79 & 70 & 0.0149 & 35.28\\
81 & 88 & 0.0201 & 41.54 & 80 & 78 & 0.0172 & 38.06 & 79 & 68 & 0.0144 & 34.80\\
81 & 86 & 0.0197 & 41.02 & 80 & 76 & 0.0167 & 37.56 & 79 & 66 & 0.0140 & 34.32\\
81 & 84 & 0.0192 & 40.50 & 80 & 74 & 0.0163 & 37.07 & 79 & 64 & 0.0136 & 33.85\\
81 & 82 & 0.0187 & 39.98 & 80 & 72 & 0.0158 & 36.57 & 79 & 62 & 0.0131 & 33.37\\
81 & 80 & 0.0183 & 39.47 & 80 & 70 & 0.0154 & 36.08 & 79 & 60 & 0.0127 & 32.90\\
81 & 78 & 0.0178 & 38.95 & 80 & 68 & 0.0149 & 35.59 & 79 & 58 & 0.0123 & 32.42\\
81 & 76 & 0.0173 & 38.44 & 80 & 66 & 0.0145 & 35.09 & 79 & 56 & 0.0119 & 31.95\\
81 & 74 & 0.0168 & 37.93 & 80 & 64 & 0.0140 & 34.60 & 79 & 54 & 0.0114 & 31.48\\
81 & 72 & 0.0164 & 37.41 & 80 & 62 & 0.0136 & 34.11 & 79 & 52 & 0.0110 & 31.01\\
81 & 70 & 0.0159 & 36.90 & 80 & 60 & 0.0131 & 33.62 & 79 & 50 & 0.0106 & 30.53\\
81 & 68 & 0.0154 & 36.39 & 80 & 58 & 0.0127 & 33.13 & 79 & 48 & 0.0101 & 30.06\\
81 & 66 & 0.0150 & 35.88 & 80 & 56 & 0.0123 & 32.64 & 79 & 46 & 0.0097 & 29.59\\
81 & 64 & 0.0145 & 35.37 & 80 & 54 & 0.0118 & 32.15 & 79 & 44 & 0.0093 & 29.12\\
81 & 62 & 0.0141 & 34.86 & 80 & 52 & 0.0114 & 31.66 & 79 & 42 & 0.0088 & 28.66\\
81 & 60 & 0.0136 & 34.35 & 80 & 50 & 0.0109 & 31.17 & 79 & 40 & 0.0084 & 28.19\\
81 & 58 & 0.0131 & 33.84 & 80 & 48 & 0.0105 & 30.68 & 79 & 38 & 0.0080 & 27.72\\
81 & 56 & 0.0127 & 33.34 & 80 & 46 & 0.0100 & 30.20 & 79 & 36 & 0.0076 & 27.25\\
81 & 54 & 0.0122 & 32.83 & 80 & 44 & 0.0096 & 29.71 & 79 & 34 & 0.0071 & 26.79\\
81 & 52 & 0.0117 & 32.33 & 80 & 42 & 0.0091 & 29.23 & 79 & 32 & 0.0067 & 26.32\\
81 & 50 & 0.0113 & 31.82 & 80 & 40 & 0.0087 & 28.74 & 79 & 30 & 0.0063 & 25.86\\
\bottomrule
\end{tabular}
\newpage
\begin{tabular}{llll|llll|llll}
 \toprule 
\(T_{db}\) & \(\phi\) & \(\omega\) & \(h\) & \(T_{db}\) & \(\phi\) & \(\omega\) & \(h\) & \(T_{db}\) & \(\phi\) & \(\omega\) & \(h\)  \\ \midrule 
79 & 28 & 0.0059 & 25.39 & 78 & 18 & 0.0036 & 22.71 & 76 & 98 & 0.0190 & 39.03\\
79 & 26 & 0.0054 & 24.93 & 78 & 16 & 0.0032 & 22.26 & 76 & 96 & 0.0186 & 38.60\\
79 & 24 & 0.0050 & 24.47 & 78 & 14 & 0.0028 & 21.82 & 76 & 94 & 0.0182 & 38.16\\
79 & 22 & 0.0046 & 24.00 & 78 & 12 & 0.0024 & 21.37 & 76 & 92 & 0.0178 & 37.72\\
79 & 20 & 0.0042 & 23.54 & 77 & 100 & 0.0201 & 40.46 & 76 & 90 & 0.0174 & 37.29\\
79 & 18 & 0.0038 & 23.08 & 77 & 98 & 0.0197 & 40.01 & 76 & 88 & 0.0170 & 36.85\\
79 & 16 & 0.0033 & 22.62 & 77 & 96 & 0.0192 & 39.55 & 76 & 86 & 0.0166 & 36.42\\
79 & 14 & 0.0029 & 22.16 & 77 & 94 & 0.0188 & 39.10 & 76 & 84 & 0.0162 & 35.99\\
79 & 12 & 0.0025 & 21.70 & 77 & 92 & 0.0184 & 38.65 & 76 & 82 & 0.0158 & 35.55\\
78 & 100 & 0.0208 & 41.47 & 77 & 90 & 0.0180 & 38.20 & 76 & 80 & 0.0154 & 35.12\\
78 & 98 & 0.0203 & 41.00 & 77 & 88 & 0.0176 & 37.75 & 76 & 78 & 0.0150 & 34.69\\
78 & 96 & 0.0199 & 40.53 & 77 & 86 & 0.0172 & 37.30 & 76 & 76 & 0.0146 & 34.26\\
78 & 94 & 0.0195 & 40.06 & 77 & 84 & 0.0168 & 36.85 & 76 & 74 & 0.0142 & 33.82\\
78 & 92 & 0.0191 & 39.60 & 77 & 82 & 0.0164 & 36.40 & 76 & 72 & 0.0138 & 33.39\\
78 & 90 & 0.0186 & 39.13 & 77 & 80 & 0.0160 & 35.95 & 76 & 70 & 0.0134 & 32.96\\
78 & 88 & 0.0182 & 38.66 & 77 & 78 & 0.0155 & 35.50 & 76 & 68 & 0.0131 & 32.53\\
78 & 86 & 0.0178 & 38.20 & 77 & 76 & 0.0151 & 35.06 & 76 & 66 & 0.0127 & 32.11\\
78 & 84 & 0.0173 & 37.73 & 77 & 74 & 0.0147 & 34.61 & 76 & 64 & 0.0123 & 31.68\\
78 & 82 & 0.0169 & 37.27 & 77 & 72 & 0.0143 & 34.16 & 76 & 62 & 0.0119 & 31.25\\
78 & 80 & 0.0165 & 36.80 & 77 & 70 & 0.0139 & 33.72 & 76 & 60 & 0.0115 & 30.82\\
78 & 78 & 0.0161 & 36.34 & 77 & 68 & 0.0135 & 33.27 & 76 & 58 & 0.0111 & 30.39\\
78 & 76 & 0.0157 & 35.87 & 77 & 66 & 0.0131 & 32.83 & 76 & 56 & 0.0107 & 29.97\\
78 & 74 & 0.0152 & 35.41 & 77 & 64 & 0.0127 & 32.39 & 76 & 54 & 0.0103 & 29.54\\
78 & 72 & 0.0148 & 34.95 & 77 & 62 & 0.0123 & 31.94 & 76 & 52 & 0.0099 & 29.12\\
78 & 70 & 0.0144 & 34.49 & 77 & 60 & 0.0119 & 31.50 & 76 & 50 & 0.0095 & 28.69\\
78 & 68 & 0.0140 & 34.03 & 77 & 58 & 0.0115 & 31.06 & 76 & 48 & 0.0092 & 28.27\\
78 & 66 & 0.0136 & 33.57 & 77 & 56 & 0.0111 & 30.62 & 76 & 46 & 0.0088 & 27.84\\
78 & 64 & 0.0131 & 33.11 & 77 & 54 & 0.0107 & 30.18 & 76 & 44 & 0.0084 & 27.42\\
78 & 62 & 0.0127 & 32.65 & 77 & 52 & 0.0103 & 29.74 & 76 & 42 & 0.0080 & 27.00\\
78 & 60 & 0.0123 & 32.19 & 77 & 50 & 0.0099 & 29.30 & 76 & 40 & 0.0076 & 26.58\\
78 & 58 & 0.0119 & 31.73 & 77 & 48 & 0.0095 & 28.86 & 76 & 38 & 0.0072 & 26.15\\
78 & 56 & 0.0115 & 31.28 & 77 & 46 & 0.0091 & 28.42 & 76 & 36 & 0.0068 & 25.73\\
78 & 54 & 0.0110 & 30.82 & 77 & 44 & 0.0087 & 27.98 & 76 & 34 & 0.0065 & 25.31\\
78 & 52 & 0.0106 & 30.36 & 77 & 42 & 0.0083 & 27.54 & 76 & 32 & 0.0061 & 24.89\\
78 & 50 & 0.0102 & 29.91 & 77 & 40 & 0.0079 & 27.10 & 76 & 30 & 0.0057 & 24.47\\
78 & 48 & 0.0098 & 29.45 & 77 & 38 & 0.0075 & 26.67 & 76 & 28 & 0.0053 & 24.05\\
78 & 46 & 0.0094 & 29.00 & 77 & 36 & 0.0071 & 26.23 & 76 & 26 & 0.0049 & 23.64\\
78 & 44 & 0.0090 & 28.55 & 77 & 34 & 0.0067 & 25.80 & 76 & 24 & 0.0045 & 23.22\\
78 & 42 & 0.0086 & 28.09 & 77 & 32 & 0.0063 & 25.36 & 76 & 22 & 0.0042 & 22.80\\
78 & 40 & 0.0081 & 27.64 & 77 & 30 & 0.0059 & 24.93 & 76 & 20 & 0.0038 & 22.38\\
78 & 38 & 0.0077 & 27.19 & 77 & 28 & 0.0055 & 24.49 & 76 & 18 & 0.0034 & 21.97\\
78 & 36 & 0.0073 & 26.74 & 77 & 26 & 0.0051 & 24.06 & 76 & 16 & 0.0030 & 21.55\\
78 & 34 & 0.0069 & 26.29 & 77 & 24 & 0.0047 & 23.63 & 76 & 14 & 0.0026 & 21.13\\
78 & 32 & 0.0065 & 25.84 & 77 & 22 & 0.0043 & 23.20 & 76 & 12 & 0.0023 & 20.72\\
78 & 30 & 0.0061 & 25.39 & 77 & 20 & 0.0039 & 22.77 & 75 & 100 & 0.0187 & 38.51\\
78 & 28 & 0.0057 & 24.94 & 77 & 18 & 0.0035 & 22.33 & 75 & 98 & 0.0184 & 38.08\\
78 & 26 & 0.0053 & 24.49 & 77 & 16 & 0.0031 & 21.90 & 75 & 96 & 0.0180 & 37.66\\
78 & 24 & 0.0049 & 24.05 & 77 & 14 & 0.0027 & 21.47 & 75 & 94 & 0.0176 & 37.24\\
78 & 22 & 0.0045 & 23.60 & 77 & 12 & 0.0023 & 21.04 & 75 & 92 & 0.0172 & 36.82\\
78 & 20 & 0.0040 & 23.15 & 76 & 100 & 0.0194 & 39.47 & 75 & 90 & 0.0168 & 36.40\\
\bottomrule
\end{tabular}
\newpage
\begin{tabular}{llll|llll|llll}
 \toprule 
\(T_{db}\) & \(\phi\) & \(\omega\) & \(h\) & \(T_{db}\) & \(\phi\) & \(\omega\) & \(h\) & \(T_{db}\) & \(\phi\) & \(\omega\) & \(h\)  \\ \midrule 
75 & 88 & 0.0164 & 35.98 & 74 & 78 & 0.0140 & 33.11 & 73 & 68 & 0.0118 & 30.41\\
75 & 86 & 0.0160 & 35.56 & 74 & 76 & 0.0137 & 32.71 & 73 & 66 & 0.0114 & 30.02\\
75 & 84 & 0.0157 & 35.14 & 74 & 74 & 0.0133 & 32.30 & 73 & 64 & 0.0111 & 29.64\\
75 & 82 & 0.0153 & 34.72 & 74 & 72 & 0.0129 & 31.90 & 73 & 62 & 0.0107 & 29.25\\
75 & 80 & 0.0149 & 34.31 & 74 & 70 & 0.0126 & 31.50 & 73 & 60 & 0.0104 & 28.87\\
75 & 78 & 0.0145 & 33.89 & 74 & 68 & 0.0122 & 31.10 & 73 & 58 & 0.0100 & 28.48\\
75 & 76 & 0.0141 & 33.47 & 74 & 66 & 0.0118 & 30.70 & 73 & 56 & 0.0097 & 28.10\\
75 & 74 & 0.0138 & 33.06 & 74 & 64 & 0.0115 & 30.30 & 73 & 54 & 0.0093 & 27.71\\
75 & 72 & 0.0134 & 32.64 & 74 & 62 & 0.0111 & 29.90 & 73 & 52 & 0.0090 & 27.33\\
75 & 70 & 0.0130 & 32.23 & 74 & 60 & 0.0107 & 29.51 & 73 & 50 & 0.0086 & 26.95\\
75 & 68 & 0.0126 & 31.81 & 74 & 58 & 0.0104 & 29.11 & 73 & 48 & 0.0083 & 26.57\\
75 & 66 & 0.0122 & 31.40 & 74 & 56 & 0.0100 & 28.71 & 73 & 46 & 0.0079 & 26.18\\
75 & 64 & 0.0119 & 30.98 & 74 & 54 & 0.0096 & 28.31 & 73 & 44 & 0.0076 & 25.80\\
75 & 62 & 0.0115 & 30.57 & 74 & 52 & 0.0093 & 27.92 & 73 & 42 & 0.0072 & 25.42\\
75 & 60 & 0.0111 & 30.16 & 74 & 50 & 0.0089 & 27.52 & 73 & 40 & 0.0069 & 25.04\\
75 & 58 & 0.0107 & 29.74 & 74 & 48 & 0.0086 & 27.12 & 73 & 38 & 0.0065 & 24.66\\
75 & 56 & 0.0104 & 29.33 & 74 & 46 & 0.0082 & 26.73 & 73 & 36 & 0.0062 & 24.28\\
75 & 54 & 0.0100 & 28.92 & 74 & 44 & 0.0078 & 26.33 & 73 & 34 & 0.0058 & 23.90\\
75 & 52 & 0.0096 & 28.51 & 74 & 42 & 0.0075 & 25.94 & 73 & 32 & 0.0055 & 23.52\\
75 & 50 & 0.0092 & 28.10 & 74 & 40 & 0.0071 & 25.55 & 73 & 30 & 0.0051 & 23.15\\
75 & 48 & 0.0089 & 27.69 & 74 & 38 & 0.0068 & 25.15 & 73 & 28 & 0.0048 & 22.77\\
75 & 46 & 0.0085 & 27.28 & 74 & 36 & 0.0064 & 24.76 & 73 & 26 & 0.0045 & 22.39\\
75 & 44 & 0.0081 & 26.87 & 74 & 34 & 0.0060 & 24.37 & 73 & 24 & 0.0041 & 22.01\\
75 & 42 & 0.0077 & 26.46 & 74 & 32 & 0.0057 & 23.97 & 73 & 22 & 0.0038 & 21.64\\
75 & 40 & 0.0074 & 26.06 & 74 & 30 & 0.0053 & 23.58 & 73 & 20 & 0.0034 & 21.26\\
75 & 38 & 0.0070 & 25.65 & 74 & 28 & 0.0050 & 23.19 & 73 & 18 & 0.0031 & 20.88\\
75 & 36 & 0.0066 & 25.24 & 74 & 26 & 0.0046 & 22.80 & 73 & 16 & 0.0027 & 20.51\\
75 & 34 & 0.0062 & 24.84 & 74 & 24 & 0.0043 & 22.41 & 73 & 14 & 0.0024 & 20.13\\
75 & 32 & 0.0059 & 24.43 & 74 & 22 & 0.0039 & 22.02 & 73 & 12 & 0.0020 & 19.76\\
75 & 30 & 0.0055 & 24.02 & 74 & 20 & 0.0035 & 21.63 & 72 & 100 & 0.0169 & 35.74\\
75 & 28 & 0.0051 & 23.62 & 74 & 18 & 0.0032 & 21.24 & 72 & 98 & 0.0165 & 35.36\\
75 & 26 & 0.0048 & 23.22 & 74 & 16 & 0.0028 & 20.85 & 72 & 96 & 0.0162 & 34.99\\
75 & 24 & 0.0044 & 22.81 & 74 & 14 & 0.0025 & 20.46 & 72 & 94 & 0.0159 & 34.61\\
75 & 22 & 0.0040 & 22.41 & 74 & 12 & 0.0021 & 20.08 & 72 & 92 & 0.0155 & 34.23\\
75 & 20 & 0.0037 & 22.00 & 73 & 100 & 0.0175 & 36.64 & 72 & 90 & 0.0152 & 33.85\\
75 & 18 & 0.0033 & 21.60 & 73 & 98 & 0.0171 & 36.25 & 72 & 88 & 0.0148 & 33.47\\
75 & 16 & 0.0029 & 21.20 & 73 & 96 & 0.0168 & 35.86 & 72 & 86 & 0.0145 & 33.10\\
75 & 14 & 0.0026 & 20.80 & 73 & 94 & 0.0164 & 35.46 & 72 & 84 & 0.0141 & 32.72\\
75 & 12 & 0.0022 & 20.40 & 73 & 92 & 0.0161 & 35.07 & 72 & 82 & 0.0138 & 32.35\\
74 & 100 & 0.0181 & 37.56 & 73 & 90 & 0.0157 & 34.68 & 72 & 80 & 0.0134 & 31.97\\
74 & 98 & 0.0177 & 37.16 & 73 & 88 & 0.0153 & 34.29 & 72 & 78 & 0.0131 & 31.60\\
74 & 96 & 0.0174 & 36.75 & 73 & 86 & 0.0150 & 33.90 & 72 & 76 & 0.0128 & 31.22\\
74 & 94 & 0.0170 & 36.34 & 73 & 84 & 0.0146 & 33.51 & 72 & 74 & 0.0124 & 30.85\\
74 & 92 & 0.0166 & 35.94 & 73 & 82 & 0.0143 & 33.12 & 72 & 72 & 0.0121 & 30.47\\
74 & 90 & 0.0162 & 35.53 & 73 & 80 & 0.0139 & 32.73 & 72 & 70 & 0.0117 & 30.10\\
74 & 88 & 0.0159 & 35.13 & 73 & 78 & 0.0136 & 32.34 & 72 & 68 & 0.0114 & 29.73\\
74 & 86 & 0.0155 & 34.72 & 73 & 76 & 0.0132 & 31.96 & 72 & 66 & 0.0110 & 29.35\\
74 & 84 & 0.0151 & 34.32 & 73 & 74 & 0.0128 & 31.57 & 72 & 64 & 0.0107 & 28.98\\
74 & 82 & 0.0148 & 33.91 & 73 & 72 & 0.0125 & 31.18 & 72 & 62 & 0.0104 & 28.61\\
74 & 80 & 0.0144 & 33.51 & 73 & 70 & 0.0121 & 30.79 & 72 & 60 & 0.0100 & 28.24\\
\bottomrule
\end{tabular}
\newpage
\begin{tabular}{llll|llll|llll}
 \toprule 
\(T_{db}\) & \(\phi\) & \(\omega\) & \(h\) & \(T_{db}\) & \(\phi\) & \(\omega\) & \(h\) & \(T_{db}\) & \(\phi\) & \(\omega\) & \(h\)  \\ \midrule 
72 & 58 & 0.0097 & 27.87 & 71 & 48 & 0.0077 & 25.48 & 70 & 38 & 0.0059 & 23.24\\
72 & 56 & 0.0093 & 27.50 & 71 & 46 & 0.0074 & 25.12 & 70 & 36 & 0.0056 & 22.90\\
72 & 54 & 0.0090 & 27.13 & 71 & 44 & 0.0071 & 24.77 & 70 & 34 & 0.0053 & 22.55\\
72 & 52 & 0.0087 & 26.76 & 71 & 42 & 0.0067 & 24.41 & 70 & 32 & 0.0050 & 22.21\\
72 & 50 & 0.0083 & 26.39 & 71 & 40 & 0.0064 & 24.06 & 70 & 30 & 0.0046 & 21.87\\
72 & 48 & 0.0080 & 26.02 & 71 & 38 & 0.0061 & 23.70 & 70 & 28 & 0.0043 & 21.53\\
72 & 46 & 0.0077 & 25.65 & 71 & 36 & 0.0058 & 23.35 & 70 & 26 & 0.0040 & 21.19\\
72 & 44 & 0.0073 & 25.28 & 71 & 34 & 0.0055 & 23.00 & 70 & 24 & 0.0037 & 20.85\\
72 & 42 & 0.0070 & 24.91 & 71 & 32 & 0.0051 & 22.64 & 70 & 22 & 0.0034 & 20.51\\
72 & 40 & 0.0066 & 24.55 & 71 & 30 & 0.0048 & 22.29 & 70 & 20 & 0.0031 & 20.17\\
72 & 38 & 0.0063 & 24.18 & 71 & 28 & 0.0045 & 21.94 & 70 & 18 & 0.0028 & 19.83\\
72 & 36 & 0.0060 & 23.81 & 71 & 26 & 0.0042 & 21.59 & 70 & 16 & 0.0025 & 19.50\\
72 & 34 & 0.0056 & 23.45 & 71 & 24 & 0.0038 & 21.23 & 70 & 14 & 0.0022 & 19.16\\
72 & 32 & 0.0053 & 23.08 & 71 & 22 & 0.0035 & 20.88 & 70 & 12 & 0.0018 & 18.82\\
72 & 30 & 0.0050 & 22.72 & 71 & 20 & 0.0032 & 20.53 & 69 & 100 & 0.0152 & 33.17\\
72 & 28 & 0.0046 & 22.35 & 71 & 18 & 0.0029 & 20.18 & 69 & 98 & 0.0149 & 32.83\\
72 & 26 & 0.0043 & 21.99 & 71 & 16 & 0.0026 & 19.83 & 69 & 96 & 0.0146 & 32.49\\
72 & 24 & 0.0040 & 21.62 & 71 & 14 & 0.0022 & 19.48 & 69 & 94 & 0.0143 & 32.15\\
72 & 22 & 0.0036 & 21.26 & 71 & 12 & 0.0019 & 19.13 & 69 & 92 & 0.0140 & 31.81\\
72 & 20 & 0.0033 & 20.89 & 70 & 100 & 0.0158 & 34.01 & 69 & 90 & 0.0137 & 31.47\\
72 & 18 & 0.0030 & 20.53 & 70 & 98 & 0.0154 & 33.65 & 69 & 88 & 0.0133 & 31.13\\
72 & 16 & 0.0026 & 20.17 & 70 & 96 & 0.0151 & 33.30 & 69 & 86 & 0.0130 & 30.79\\
72 & 14 & 0.0023 & 19.81 & 70 & 94 & 0.0148 & 32.95 & 69 & 84 & 0.0127 & 30.45\\
72 & 12 & 0.0020 & 19.44 & 70 & 92 & 0.0145 & 32.60 & 69 & 82 & 0.0124 & 30.12\\
71 & 100 & 0.0163 & 34.86 & 70 & 90 & 0.0141 & 32.25 & 69 & 80 & 0.0121 & 29.78\\
71 & 98 & 0.0160 & 34.50 & 70 & 88 & 0.0138 & 31.89 & 69 & 78 & 0.0118 & 29.44\\
71 & 96 & 0.0156 & 34.13 & 70 & 86 & 0.0135 & 31.54 & 69 & 76 & 0.0115 & 29.11\\
71 & 94 & 0.0153 & 33.77 & 70 & 84 & 0.0132 & 31.19 & 69 & 74 & 0.0112 & 28.77\\
71 & 92 & 0.0150 & 33.40 & 70 & 82 & 0.0129 & 30.84 & 69 & 72 & 0.0109 & 28.44\\
71 & 90 & 0.0146 & 33.04 & 70 & 80 & 0.0125 & 30.49 & 69 & 70 & 0.0106 & 28.10\\
71 & 88 & 0.0143 & 32.68 & 70 & 78 & 0.0122 & 30.15 & 69 & 68 & 0.0103 & 27.76\\
71 & 86 & 0.0140 & 32.31 & 70 & 76 & 0.0119 & 29.80 & 69 & 66 & 0.0100 & 27.43\\
71 & 84 & 0.0136 & 31.95 & 70 & 74 & 0.0116 & 29.45 & 69 & 64 & 0.0097 & 27.10\\
71 & 82 & 0.0133 & 31.59 & 70 & 72 & 0.0113 & 29.10 & 69 & 62 & 0.0093 & 26.76\\
71 & 80 & 0.0130 & 31.22 & 70 & 70 & 0.0109 & 28.75 & 69 & 60 & 0.0090 & 26.43\\
71 & 78 & 0.0127 & 30.86 & 70 & 68 & 0.0106 & 28.41 & 69 & 58 & 0.0087 & 26.09\\
71 & 76 & 0.0123 & 30.50 & 70 & 66 & 0.0103 & 28.06 & 69 & 56 & 0.0084 & 25.76\\
71 & 74 & 0.0120 & 30.14 & 70 & 64 & 0.0100 & 27.71 & 69 & 54 & 0.0081 & 25.43\\
71 & 72 & 0.0117 & 29.78 & 70 & 62 & 0.0097 & 27.37 & 69 & 52 & 0.0078 & 25.09\\
71 & 70 & 0.0113 & 29.42 & 70 & 60 & 0.0094 & 27.02 & 69 & 50 & 0.0075 & 24.76\\
71 & 68 & 0.0110 & 29.06 & 70 & 58 & 0.0090 & 26.67 & 69 & 48 & 0.0072 & 24.43\\
71 & 66 & 0.0107 & 28.70 & 70 & 56 & 0.0087 & 26.33 & 69 & 46 & 0.0069 & 24.10\\
71 & 64 & 0.0103 & 28.34 & 70 & 54 & 0.0084 & 25.98 & 69 & 44 & 0.0066 & 23.77\\
71 & 62 & 0.0100 & 27.98 & 70 & 52 & 0.0081 & 25.64 & 69 & 42 & 0.0063 & 23.44\\
71 & 60 & 0.0097 & 27.62 & 70 & 50 & 0.0078 & 25.30 & 69 & 40 & 0.0060 & 23.11\\
71 & 58 & 0.0094 & 27.27 & 70 & 48 & 0.0075 & 24.95 & 69 & 38 & 0.0057 & 22.78\\
71 & 56 & 0.0090 & 26.91 & 70 & 46 & 0.0071 & 24.61 & 69 & 36 & 0.0054 & 22.45\\
71 & 54 & 0.0087 & 26.55 & 70 & 44 & 0.0068 & 24.26 & 69 & 34 & 0.0051 & 22.12\\
71 & 52 & 0.0084 & 26.19 & 70 & 42 & 0.0065 & 23.92 & 69 & 32 & 0.0048 & 21.79\\
71 & 50 & 0.0081 & 25.84 & 70 & 40 & 0.0062 & 23.58 & 69 & 30 & 0.0045 & 21.46\\
\bottomrule
\end{tabular}
\newpage
\begin{tabular}{llll|llll|llll}
 \toprule 
\(T_{db}\) & \(\phi\) & \(\omega\) & \(h\) & \(T_{db}\) & \(\phi\) & \(\omega\) & \(h\) & \(T_{db}\) & \(\phi\) & \(\omega\) & \(h\)  \\ \midrule 
69 & 28 & 0.0042 & 21.13 & 68 & 18 & 0.0026 & 19.15 & 66 & 98 & 0.0134 & 30.45\\
69 & 26 & 0.0039 & 20.80 & 68 & 16 & 0.0023 & 18.83 & 66 & 96 & 0.0131 & 30.15\\
69 & 24 & 0.0036 & 20.47 & 68 & 14 & 0.0020 & 18.52 & 66 & 94 & 0.0128 & 29.85\\
69 & 22 & 0.0033 & 20.14 & 68 & 12 & 0.0017 & 18.20 & 66 & 92 & 0.0126 & 29.54\\
69 & 20 & 0.0030 & 19.82 & 67 & 100 & 0.0142 & 31.54 & 66 & 90 & 0.0123 & 29.24\\
69 & 18 & 0.0027 & 19.49 & 67 & 98 & 0.0139 & 31.23 & 66 & 88 & 0.0120 & 28.93\\
69 & 16 & 0.0024 & 19.16 & 67 & 96 & 0.0136 & 30.91 & 66 & 86 & 0.0117 & 28.63\\
69 & 14 & 0.0021 & 18.84 & 67 & 94 & 0.0133 & 30.60 & 66 & 84 & 0.0115 & 28.33\\
69 & 12 & 0.0018 & 18.51 & 67 & 92 & 0.0130 & 30.28 & 66 & 82 & 0.0112 & 28.03\\
68 & 100 & 0.0147 & 32.35 & 67 & 90 & 0.0127 & 29.97 & 66 & 80 & 0.0109 & 27.72\\
68 & 98 & 0.0144 & 32.02 & 67 & 88 & 0.0124 & 29.65 & 66 & 78 & 0.0106 & 27.42\\
68 & 96 & 0.0141 & 31.69 & 67 & 86 & 0.0122 & 29.34 & 66 & 76 & 0.0103 & 27.12\\
68 & 94 & 0.0138 & 31.36 & 67 & 84 & 0.0119 & 29.02 & 66 & 74 & 0.0101 & 26.82\\
68 & 92 & 0.0135 & 31.04 & 67 & 82 & 0.0116 & 28.71 & 66 & 72 & 0.0098 & 26.52\\
68 & 90 & 0.0132 & 30.71 & 67 & 80 & 0.0113 & 28.39 & 66 & 70 & 0.0095 & 26.22\\
68 & 88 & 0.0129 & 30.38 & 67 & 78 & 0.0110 & 28.08 & 66 & 68 & 0.0092 & 25.91\\
68 & 86 & 0.0126 & 30.06 & 67 & 76 & 0.0107 & 27.77 & 66 & 66 & 0.0090 & 25.61\\
68 & 84 & 0.0123 & 29.73 & 67 & 74 & 0.0104 & 27.46 & 66 & 64 & 0.0087 & 25.31\\
68 & 82 & 0.0120 & 29.41 & 67 & 72 & 0.0101 & 27.14 & 66 & 62 & 0.0084 & 25.01\\
68 & 80 & 0.0117 & 29.08 & 67 & 70 & 0.0099 & 26.83 & 66 & 60 & 0.0081 & 24.71\\
68 & 78 & 0.0114 & 28.76 & 67 & 68 & 0.0096 & 26.52 & 66 & 58 & 0.0079 & 24.41\\
68 & 76 & 0.0111 & 28.43 & 67 & 66 & 0.0093 & 26.21 & 66 & 56 & 0.0076 & 24.11\\
68 & 74 & 0.0108 & 28.11 & 67 & 64 & 0.0090 & 25.90 & 66 & 54 & 0.0073 & 23.82\\
68 & 72 & 0.0105 & 27.78 & 67 & 62 & 0.0087 & 25.59 & 66 & 52 & 0.0070 & 23.52\\
68 & 70 & 0.0102 & 27.46 & 67 & 60 & 0.0084 & 25.27 & 66 & 50 & 0.0068 & 23.22\\
68 & 68 & 0.0099 & 27.14 & 67 & 58 & 0.0081 & 24.96 & 66 & 48 & 0.0065 & 22.92\\
68 & 66 & 0.0096 & 26.81 & 67 & 56 & 0.0079 & 24.65 & 66 & 46 & 0.0062 & 22.62\\
68 & 64 & 0.0093 & 26.49 & 67 & 54 & 0.0076 & 24.34 & 66 & 44 & 0.0059 & 22.32\\
68 & 62 & 0.0090 & 26.17 & 67 & 52 & 0.0073 & 24.03 & 66 & 42 & 0.0057 & 22.03\\
68 & 60 & 0.0087 & 25.85 & 67 & 50 & 0.0070 & 23.72 & 66 & 40 & 0.0054 & 21.73\\
68 & 58 & 0.0084 & 25.52 & 67 & 48 & 0.0067 & 23.42 & 66 & 38 & 0.0051 & 21.43\\
68 & 56 & 0.0081 & 25.20 & 67 & 46 & 0.0064 & 23.11 & 66 & 36 & 0.0049 & 21.14\\
68 & 54 & 0.0078 & 24.88 & 67 & 44 & 0.0062 & 22.80 & 66 & 34 & 0.0046 & 20.84\\
68 & 52 & 0.0076 & 24.56 & 67 & 42 & 0.0059 & 22.49 & 66 & 32 & 0.0043 & 20.54\\
68 & 50 & 0.0073 & 24.24 & 67 & 40 & 0.0056 & 22.18 & 66 & 30 & 0.0040 & 20.25\\
68 & 48 & 0.0070 & 23.92 & 67 & 38 & 0.0053 & 21.87 & 66 & 28 & 0.0038 & 19.95\\
68 & 46 & 0.0067 & 23.60 & 67 & 36 & 0.0050 & 21.57 & 66 & 26 & 0.0035 & 19.66\\
68 & 44 & 0.0064 & 23.28 & 67 & 34 & 0.0047 & 21.26 & 66 & 24 & 0.0032 & 19.36\\
68 & 42 & 0.0061 & 22.96 & 67 & 32 & 0.0045 & 20.95 & 66 & 22 & 0.0030 & 19.07\\
68 & 40 & 0.0058 & 22.64 & 67 & 30 & 0.0042 & 20.65 & 66 & 20 & 0.0027 & 18.77\\
68 & 38 & 0.0055 & 22.32 & 67 & 28 & 0.0039 & 20.34 & 66 & 18 & 0.0024 & 18.48\\
68 & 36 & 0.0052 & 22.00 & 67 & 26 & 0.0036 & 20.03 & 66 & 16 & 0.0021 & 18.18\\
68 & 34 & 0.0049 & 21.69 & 67 & 24 & 0.0033 & 19.73 & 66 & 14 & 0.0019 & 17.89\\
68 & 32 & 0.0046 & 21.37 & 67 & 22 & 0.0031 & 19.42 & 66 & 12 & 0.0016 & 17.60\\
68 & 30 & 0.0043 & 21.05 & 67 & 20 & 0.0028 & 19.12 & 65 & 100 & 0.0132 & 29.99\\
68 & 28 & 0.0040 & 20.73 & 67 & 18 & 0.0025 & 18.81 & 65 & 98 & 0.0129 & 29.70\\
68 & 26 & 0.0038 & 20.42 & 67 & 16 & 0.0022 & 18.51 & 65 & 96 & 0.0127 & 29.41\\
68 & 24 & 0.0035 & 20.10 & 67 & 14 & 0.0019 & 18.20 & 65 & 94 & 0.0124 & 29.11\\
68 & 22 & 0.0032 & 19.78 & 67 & 12 & 0.0017 & 17.90 & 65 & 92 & 0.0121 & 28.82\\
68 & 20 & 0.0029 & 19.47 & 66 & 100 & 0.0137 & 30.76 & 65 & 90 & 0.0119 & 28.53\\
\bottomrule
\end{tabular}
\newpage
\begin{tabular}{llll|llll|llll}
 \toprule 
\(T_{db}\) & \(\phi\) & \(\omega\) & \(h\) & \(T_{db}\) & \(\phi\) & \(\omega\) & \(h\) & \(T_{db}\) & \(\phi\) & \(\omega\) & \(h\)  \\ \midrule 
65 & 88 & 0.0116 & 28.23 & 64 & 78 & 0.0099 & 26.14 & 63 & 56 & 0.0068 & 22.55\\
65 & 86 & 0.0113 & 27.94 & 64 & 76 & 0.0096 & 25.86 & 63 & 53 & 0.0065 & 22.15\\
65 & 84 & 0.0111 & 27.65 & 64 & 74 & 0.0094 & 25.58 & 63 & 50 & 0.0061 & 21.75\\
65 & 82 & 0.0108 & 27.36 & 64 & 72 & 0.0091 & 25.30 & 63 & 47 & 0.0057 & 21.35\\
65 & 80 & 0.0105 & 27.07 & 64 & 70 & 0.0089 & 25.02 & 63 & 44 & 0.0054 & 20.95\\
65 & 78 & 0.0103 & 26.77 & 64 & 68 & 0.0086 & 24.74 & 63 & 41 & 0.0050 & 20.55\\
65 & 76 & 0.0100 & 26.48 & 64 & 66 & 0.0084 & 24.46 & 63 & 38 & 0.0046 & 20.15\\
65 & 74 & 0.0097 & 26.19 & 64 & 64 & 0.0081 & 24.18 & 63 & 35 & 0.0042 & 19.75\\
65 & 72 & 0.0095 & 25.90 & 64 & 62 & 0.0078 & 23.90 & 63 & 32 & 0.0039 & 19.35\\
65 & 70 & 0.0092 & 25.61 & 64 & 60 & 0.0076 & 23.62 & 63 & 29 & 0.0035 & 18.95\\
65 & 68 & 0.0089 & 25.32 & 64 & 58 & 0.0073 & 23.34 & 63 & 26 & 0.0032 & 18.55\\
65 & 66 & 0.0087 & 25.03 & 64 & 56 & 0.0071 & 23.07 & 63 & 23 & 0.0028 & 18.15\\
65 & 64 & 0.0084 & 24.74 & 64 & 54 & 0.0068 & 22.79 & 63 & 20 & 0.0024 & 17.76\\
65 & 62 & 0.0081 & 24.45 & 64 & 52 & 0.0066 & 22.51 & 63 & 17 & 0.0021 & 17.36\\
65 & 60 & 0.0079 & 24.16 & 64 & 50 & 0.0063 & 22.23 & 63 & 14 & 0.0017 & 16.96\\
65 & 58 & 0.0076 & 23.87 & 64 & 48 & 0.0061 & 21.95 & 63 & 11 & 0.0013 & 16.57\\
65 & 56 & 0.0073 & 23.59 & 64 & 46 & 0.0058 & 21.68 & 62 & 100 & 0.0119 & 27.79\\
65 & 54 & 0.0071 & 23.30 & 64 & 44 & 0.0055 & 21.40 & 62 & 97 & 0.0115 & 27.40\\
65 & 52 & 0.0068 & 23.01 & 64 & 42 & 0.0053 & 21.12 & 62 & 94 & 0.0111 & 27.00\\
65 & 50 & 0.0065 & 22.72 & 64 & 40 & 0.0050 & 20.85 & 62 & 91 & 0.0108 & 26.61\\
65 & 48 & 0.0063 & 22.43 & 64 & 38 & 0.0048 & 20.57 & 62 & 88 & 0.0104 & 26.22\\
65 & 46 & 0.0060 & 22.15 & 64 & 36 & 0.0045 & 20.29 & 62 & 85 & 0.0101 & 25.82\\
65 & 44 & 0.0057 & 21.86 & 64 & 34 & 0.0043 & 20.02 & 62 & 82 & 0.0097 & 25.43\\
65 & 42 & 0.0055 & 21.57 & 64 & 32 & 0.0040 & 19.74 & 62 & 79 & 0.0093 & 25.04\\
65 & 40 & 0.0052 & 21.28 & 64 & 30 & 0.0038 & 19.47 & 62 & 76 & 0.0090 & 24.65\\
65 & 38 & 0.0050 & 21.00 & 64 & 28 & 0.0035 & 19.19 & 62 & 73 & 0.0086 & 24.26\\
65 & 36 & 0.0047 & 20.71 & 64 & 26 & 0.0033 & 18.92 & 62 & 70 & 0.0083 & 23.87\\
65 & 34 & 0.0044 & 20.43 & 64 & 24 & 0.0030 & 18.64 & 62 & 67 & 0.0079 & 23.48\\
65 & 32 & 0.0042 & 20.14 & 64 & 22 & 0.0028 & 18.37 & 62 & 64 & 0.0075 & 23.09\\
65 & 30 & 0.0039 & 19.85 & 64 & 20 & 0.0025 & 18.09 & 62 & 61 & 0.0072 & 22.70\\
65 & 28 & 0.0036 & 19.57 & 64 & 18 & 0.0023 & 17.82 & 62 & 58 & 0.0068 & 22.31\\
65 & 26 & 0.0034 & 19.28 & 64 & 16 & 0.0020 & 17.54 & 62 & 55 & 0.0065 & 21.92\\
65 & 24 & 0.0031 & 19.00 & 64 & 14 & 0.0018 & 17.27 & 62 & 52 & 0.0061 & 21.53\\
65 & 22 & 0.0029 & 18.71 & 64 & 12 & 0.0015 & 17.00 & 62 & 49 & 0.0058 & 21.15\\
65 & 20 & 0.0026 & 18.43 & 63 & 100 & 0.0123 & 28.51 & 62 & 46 & 0.0054 & 20.76\\
65 & 18 & 0.0023 & 18.15 & 63 & 98 & 0.0120 & 28.24 & 62 & 43 & 0.0050 & 20.37\\
65 & 16 & 0.0021 & 17.86 & 63 & 96 & 0.0118 & 27.96 & 62 & 40 & 0.0047 & 19.99\\
65 & 14 & 0.0018 & 17.58 & 63 & 94 & 0.0115 & 27.69 & 62 & 37 & 0.0043 & 19.60\\
65 & 12 & 0.0016 & 17.30 & 63 & 92 & 0.0113 & 27.42 & 62 & 34 & 0.0040 & 19.22\\
64 & 100 & 0.0127 & 29.24 & 63 & 89 & 0.0109 & 27.01 & 62 & 31 & 0.0036 & 18.83\\
64 & 98 & 0.0125 & 28.96 & 63 & 86 & 0.0105 & 26.60 & 62 & 28 & 0.0033 & 18.45\\
64 & 96 & 0.0122 & 28.68 & 63 & 83 & 0.0102 & 26.20 & 62 & 25 & 0.0029 & 18.06\\
64 & 94 & 0.0120 & 28.39 & 63 & 80 & 0.0098 & 25.79 & 62 & 22 & 0.0026 & 17.68\\
64 & 92 & 0.0117 & 28.11 & 63 & 77 & 0.0094 & 25.38 & 62 & 19 & 0.0022 & 17.30\\
64 & 90 & 0.0114 & 27.83 & 63 & 74 & 0.0091 & 24.98 & 62 & 16 & 0.0019 & 16.91\\
64 & 88 & 0.0112 & 27.55 & 63 & 71 & 0.0087 & 24.57 & 62 & 13 & 0.0015 & 16.53\\
64 & 86 & 0.0109 & 27.27 & 63 & 68 & 0.0083 & 24.17 & 61 & 100 & 0.0114 & 27.09\\
64 & 84 & 0.0107 & 26.98 & 63 & 65 & 0.0079 & 23.76 & 61 & 97 & 0.0111 & 26.71\\
64 & 82 & 0.0104 & 26.70 & 63 & 62 & 0.0076 & 23.36 & 61 & 94 & 0.0107 & 26.33\\
64 & 80 & 0.0102 & 26.42 & 63 & 59 & 0.0072 & 22.96 & 61 & 91 & 0.0104 & 25.95\\
\bottomrule
\end{tabular}
\newpage
\begin{tabular}{llll|llll|llll}
 \toprule 
\(T_{db}\) & \(\phi\) & \(\omega\) & \(h\) & \(T_{db}\) & \(\phi\) & \(\omega\) & \(h\) & \(T_{db}\) & \(\phi\) & \(\omega\) & \(h\)  \\ \midrule 
61 & 88 & 0.0100 & 25.57 & 60 & 28 & 0.0031 & 17.72 & 58 & 58 & 0.0059 & 20.34\\
61 & 85 & 0.0097 & 25.19 & 60 & 25 & 0.0027 & 17.36 & 58 & 55 & 0.0056 & 20.01\\
61 & 82 & 0.0094 & 24.81 & 60 & 22 & 0.0024 & 17.00 & 58 & 52 & 0.0053 & 19.67\\
61 & 79 & 0.0090 & 24.44 & 60 & 19 & 0.0021 & 16.65 & 58 & 49 & 0.0050 & 19.34\\
61 & 76 & 0.0087 & 24.06 & 60 & 16 & 0.0017 & 16.29 & 58 & 46 & 0.0047 & 19.00\\
61 & 73 & 0.0083 & 23.68 & 60 & 13 & 0.0014 & 15.94 & 58 & 43 & 0.0044 & 18.67\\
61 & 70 & 0.0080 & 23.31 & 59 & 100 & 0.0106 & 25.73 & 58 & 40 & 0.0041 & 18.34\\
61 & 67 & 0.0076 & 22.93 & 59 & 97 & 0.0103 & 25.38 & 58 & 37 & 0.0038 & 18.00\\
61 & 64 & 0.0073 & 22.56 & 59 & 94 & 0.0100 & 25.02 & 58 & 34 & 0.0035 & 17.67\\
61 & 61 & 0.0069 & 22.18 & 59 & 91 & 0.0097 & 24.67 & 58 & 31 & 0.0031 & 17.34\\
61 & 58 & 0.0066 & 21.81 & 59 & 88 & 0.0093 & 24.32 & 58 & 28 & 0.0028 & 17.01\\
61 & 55 & 0.0062 & 21.43 & 59 & 85 & 0.0090 & 23.97 & 58 & 25 & 0.0025 & 16.67\\
61 & 52 & 0.0059 & 21.06 & 59 & 82 & 0.0087 & 23.62 & 58 & 22 & 0.0022 & 16.34\\
61 & 49 & 0.0056 & 20.68 & 59 & 79 & 0.0084 & 23.27 & 58 & 19 & 0.0019 & 16.01\\
61 & 46 & 0.0052 & 20.31 & 59 & 76 & 0.0081 & 22.92 & 58 & 16 & 0.0016 & 15.68\\
61 & 43 & 0.0049 & 19.94 & 59 & 73 & 0.0077 & 22.57 & 58 & 13 & 0.0013 & 15.35\\
61 & 40 & 0.0045 & 19.57 & 59 & 70 & 0.0074 & 22.22 & 57 & 100 & 0.0099 & 24.43\\
61 & 37 & 0.0042 & 19.19 & 59 & 67 & 0.0071 & 21.87 & 57 & 97 & 0.0096 & 24.10\\
61 & 34 & 0.0038 & 18.82 & 59 & 64 & 0.0068 & 21.52 & 57 & 94 & 0.0093 & 23.77\\
61 & 31 & 0.0035 & 18.45 & 59 & 61 & 0.0064 & 21.17 & 57 & 91 & 0.0090 & 23.44\\
61 & 28 & 0.0032 & 18.08 & 59 & 58 & 0.0061 & 20.82 & 57 & 88 & 0.0087 & 23.12\\
61 & 25 & 0.0028 & 17.71 & 59 & 55 & 0.0058 & 20.47 & 57 & 85 & 0.0084 & 22.79\\
61 & 22 & 0.0025 & 17.34 & 59 & 52 & 0.0055 & 20.13 & 57 & 82 & 0.0081 & 22.47\\
61 & 19 & 0.0021 & 16.97 & 59 & 49 & 0.0052 & 19.78 & 57 & 79 & 0.0078 & 22.14\\
61 & 16 & 0.0018 & 16.60 & 59 & 46 & 0.0048 & 19.43 & 57 & 76 & 0.0075 & 21.82\\
61 & 13 & 0.0015 & 16.23 & 59 & 43 & 0.0045 & 19.09 & 57 & 73 & 0.0072 & 21.49\\
60 & 100 & 0.0110 & 26.40 & 59 & 40 & 0.0042 & 18.74 & 57 & 70 & 0.0069 & 21.17\\
60 & 97 & 0.0107 & 26.04 & 59 & 37 & 0.0039 & 18.39 & 57 & 67 & 0.0066 & 20.84\\
60 & 94 & 0.0104 & 25.67 & 59 & 34 & 0.0036 & 18.05 & 57 & 64 & 0.0063 & 20.52\\
60 & 91 & 0.0100 & 25.30 & 59 & 31 & 0.0033 & 17.70 & 57 & 61 & 0.0060 & 20.19\\
60 & 88 & 0.0097 & 24.94 & 59 & 28 & 0.0029 & 17.36 & 57 & 58 & 0.0057 & 19.87\\
60 & 85 & 0.0094 & 24.57 & 59 & 25 & 0.0026 & 17.02 & 57 & 55 & 0.0054 & 19.55\\
60 & 82 & 0.0090 & 24.21 & 59 & 22 & 0.0023 & 16.67 & 57 & 52 & 0.0051 & 19.23\\
60 & 79 & 0.0087 & 23.85 & 59 & 19 & 0.0020 & 16.33 & 57 & 49 & 0.0048 & 18.90\\
60 & 76 & 0.0084 & 23.48 & 59 & 16 & 0.0017 & 15.98 & 57 & 46 & 0.0045 & 18.58\\
60 & 73 & 0.0080 & 23.12 & 59 & 13 & 0.0014 & 15.64 & 57 & 43 & 0.0042 & 18.26\\
60 & 70 & 0.0077 & 22.76 & 58 & 100 & 0.0103 & 25.07 & 57 & 40 & 0.0039 & 17.94\\
60 & 67 & 0.0073 & 22.39 & 58 & 97 & 0.0099 & 24.73 & 57 & 37 & 0.0036 & 17.62\\
60 & 64 & 0.0070 & 22.03 & 58 & 94 & 0.0096 & 24.39 & 57 & 34 & 0.0033 & 17.30\\
60 & 61 & 0.0067 & 21.67 & 58 & 91 & 0.0093 & 24.05 & 57 & 31 & 0.0030 & 16.97\\
60 & 58 & 0.0064 & 21.31 & 58 & 88 & 0.0090 & 23.71 & 57 & 28 & 0.0027 & 16.65\\
60 & 55 & 0.0060 & 20.95 & 58 & 85 & 0.0087 & 23.37 & 57 & 25 & 0.0024 & 16.33\\
60 & 52 & 0.0057 & 20.59 & 58 & 82 & 0.0084 & 23.04 & 57 & 22 & 0.0021 & 16.02\\
60 & 49 & 0.0054 & 20.23 & 58 & 79 & 0.0081 & 22.70 & 57 & 19 & 0.0019 & 15.70\\
60 & 46 & 0.0050 & 19.87 & 58 & 76 & 0.0078 & 22.36 & 57 & 16 & 0.0016 & 15.38\\
60 & 43 & 0.0047 & 19.51 & 58 & 73 & 0.0075 & 22.02 & 57 & 13 & 0.0013 & 15.06\\
60 & 40 & 0.0044 & 19.15 & 58 & 70 & 0.0071 & 21.69 & 56 & 100 & 0.0095 & 23.79\\
60 & 37 & 0.0040 & 18.79 & 58 & 67 & 0.0068 & 21.35 & 56 & 97 & 0.0092 & 23.48\\
60 & 34 & 0.0037 & 18.43 & 58 & 64 & 0.0065 & 21.01 & 56 & 94 & 0.0090 & 23.16\\
60 & 31 & 0.0034 & 18.08 & 58 & 61 & 0.0062 & 20.68 & 56 & 91 & 0.0087 & 22.85\\
\bottomrule
\end{tabular}
\newpage
\begin{tabular}{llll|llll|llll}
 \toprule 
\(T_{db}\) & \(\phi\) & \(\omega\) & \(h\) & \(T_{db}\) & \(\phi\) & \(\omega\) & \(h\) & \(T_{db}\) & \(\phi\) & \(\omega\) & \(h\)  \\ \midrule 
56 & 88 & 0.0084 & 22.53 & 55 & 28 & 0.0025 & 15.96 & 53 & 58 & 0.0049 & 18.06\\
56 & 85 & 0.0081 & 22.22 & 55 & 25 & 0.0023 & 15.67 & 53 & 55 & 0.0047 & 17.78\\
56 & 82 & 0.0078 & 21.91 & 55 & 22 & 0.0020 & 15.37 & 53 & 52 & 0.0044 & 17.50\\
56 & 79 & 0.0075 & 21.59 & 55 & 19 & 0.0017 & 15.07 & 53 & 49 & 0.0042 & 17.22\\
56 & 76 & 0.0072 & 21.28 & 55 & 16 & 0.0015 & 14.78 & 53 & 46 & 0.0039 & 16.95\\
56 & 73 & 0.0069 & 20.97 & 55 & 13 & 0.0012 & 14.48 & 53 & 43 & 0.0036 & 16.67\\
56 & 70 & 0.0066 & 20.65 & 54 & 100 & 0.0089 & 22.57 & 53 & 40 & 0.0034 & 16.39\\
56 & 67 & 0.0064 & 20.34 & 54 & 97 & 0.0086 & 22.28 & 53 & 37 & 0.0031 & 16.11\\
56 & 64 & 0.0061 & 20.03 & 54 & 94 & 0.0083 & 21.98 & 53 & 34 & 0.0029 & 15.84\\
56 & 61 & 0.0058 & 19.72 & 54 & 91 & 0.0080 & 21.69 & 53 & 31 & 0.0026 & 15.56\\
56 & 58 & 0.0055 & 19.41 & 54 & 88 & 0.0078 & 21.40 & 53 & 28 & 0.0024 & 15.29\\
56 & 55 & 0.0052 & 19.10 & 54 & 85 & 0.0075 & 21.11 & 53 & 25 & 0.0021 & 15.01\\
56 & 52 & 0.0049 & 18.78 & 54 & 82 & 0.0072 & 20.82 & 53 & 22 & 0.0019 & 14.73\\
56 & 49 & 0.0046 & 18.47 & 54 & 79 & 0.0070 & 20.53 & 53 & 19 & 0.0016 & 14.46\\
56 & 46 & 0.0043 & 18.16 & 54 & 76 & 0.0067 & 20.24 & 53 & 16 & 0.0013 & 14.18\\
56 & 43 & 0.0041 & 17.85 & 54 & 73 & 0.0064 & 19.95 & 53 & 13 & 0.0011 & 13.91\\
56 & 40 & 0.0038 & 17.54 & 54 & 70 & 0.0062 & 19.66 & 52 & 100 & 0.0082 & 21.39\\
56 & 37 & 0.0035 & 17.23 & 54 & 67 & 0.0059 & 19.37 & 52 & 96 & 0.0079 & 21.03\\
56 & 34 & 0.0032 & 16.92 & 54 & 64 & 0.0056 & 19.08 & 52 & 92 & 0.0076 & 20.67\\
56 & 31 & 0.0029 & 16.62 & 54 & 61 & 0.0054 & 18.79 & 52 & 88 & 0.0072 & 20.31\\
56 & 28 & 0.0026 & 16.31 & 54 & 58 & 0.0051 & 18.50 & 52 & 84 & 0.0069 & 19.95\\
56 & 25 & 0.0024 & 16.00 & 54 & 55 & 0.0048 & 18.21 & 52 & 80 & 0.0066 & 19.59\\
56 & 22 & 0.0021 & 15.69 & 54 & 52 & 0.0046 & 17.92 & 52 & 76 & 0.0062 & 19.23\\
56 & 19 & 0.0018 & 15.38 & 54 & 49 & 0.0043 & 17.63 & 52 & 72 & 0.0059 & 18.87\\
56 & 16 & 0.0015 & 15.08 & 54 & 46 & 0.0040 & 17.35 & 52 & 68 & 0.0056 & 18.51\\
56 & 13 & 0.0012 & 14.77 & 54 & 43 & 0.0038 & 17.06 & 52 & 64 & 0.0052 & 18.16\\
55 & 100 & 0.0092 & 23.17 & 54 & 40 & 0.0035 & 16.77 & 52 & 60 & 0.0049 & 17.80\\
55 & 97 & 0.0089 & 22.87 & 54 & 37 & 0.0032 & 16.48 & 52 & 56 & 0.0046 & 17.44\\
55 & 94 & 0.0086 & 22.57 & 54 & 34 & 0.0030 & 16.20 & 52 & 52 & 0.0042 & 17.08\\
55 & 91 & 0.0084 & 22.26 & 54 & 31 & 0.0027 & 15.91 & 52 & 48 & 0.0039 & 16.73\\
55 & 88 & 0.0081 & 21.96 & 54 & 28 & 0.0025 & 15.62 & 52 & 44 & 0.0036 & 16.37\\
55 & 85 & 0.0078 & 21.66 & 54 & 25 & 0.0022 & 15.34 & 52 & 40 & 0.0033 & 16.02\\
55 & 82 & 0.0075 & 21.36 & 54 & 22 & 0.0019 & 15.05 & 52 & 36 & 0.0029 & 15.66\\
55 & 79 & 0.0072 & 21.06 & 54 & 19 & 0.0017 & 14.76 & 52 & 32 & 0.0026 & 15.31\\
55 & 76 & 0.0070 & 20.75 & 54 & 16 & 0.0014 & 14.48 & 52 & 28 & 0.0023 & 14.95\\
55 & 73 & 0.0067 & 20.45 & 54 & 13 & 0.0011 & 14.19 & 52 & 24 & 0.0020 & 14.60\\
55 & 70 & 0.0064 & 20.15 & 53 & 100 & 0.0085 & 21.97 & 52 & 20 & 0.0016 & 14.24\\
55 & 67 & 0.0061 & 19.85 & 53 & 97 & 0.0083 & 21.69 & 52 & 16 & 0.0013 & 13.89\\
55 & 64 & 0.0058 & 19.55 & 53 & 94 & 0.0080 & 21.41 & 52 & 12 & 0.0010 & 13.54\\
55 & 61 & 0.0056 & 19.25 & 53 & 91 & 0.0078 & 21.13 & 51 & 100 & 0.0079 & 20.82\\
55 & 58 & 0.0053 & 18.95 & 53 & 88 & 0.0075 & 20.85 & 51 & 96 & 0.0076 & 20.47\\
55 & 55 & 0.0050 & 18.65 & 53 & 85 & 0.0072 & 20.57 & 51 & 92 & 0.0073 & 20.13\\
55 & 52 & 0.0047 & 18.35 & 53 & 82 & 0.0070 & 20.29 & 51 & 88 & 0.0070 & 19.78\\
55 & 49 & 0.0045 & 18.05 & 53 & 79 & 0.0067 & 20.01 & 51 & 84 & 0.0066 & 19.43\\
55 & 46 & 0.0042 & 17.75 & 53 & 76 & 0.0065 & 19.73 & 51 & 80 & 0.0063 & 19.09\\
55 & 43 & 0.0039 & 17.45 & 53 & 73 & 0.0062 & 19.45 & 51 & 76 & 0.0060 & 18.74\\
55 & 40 & 0.0036 & 17.15 & 53 & 70 & 0.0059 & 19.17 & 51 & 72 & 0.0057 & 18.40\\
55 & 37 & 0.0034 & 16.86 & 53 & 67 & 0.0057 & 18.89 & 51 & 68 & 0.0054 & 18.05\\
55 & 34 & 0.0031 & 16.56 & 53 & 64 & 0.0054 & 18.61 & 51 & 64 & 0.0050 & 17.71\\
55 & 31 & 0.0028 & 16.26 & 53 & 61 & 0.0052 & 18.33 & 51 & 60 & 0.0047 & 17.36\\
\bottomrule
\end{tabular}
\newpage
\begin{tabular}{llll|llll|llll}
 \toprule 
\(T_{db}\) & \(\phi\) & \(\omega\) & \(h\) & \(T_{db}\) & \(\phi\) & \(\omega\) & \(h\) & \(T_{db}\) & \(\phi\) & \(\omega\) & \(h\)  \\ \midrule 
51 & 56 & 0.0044 & 17.02 & 49 & 40 & 0.0029 & 14.92 & 47 & 24 & 0.0016 & 13.03\\
51 & 52 & 0.0041 & 16.67 & 49 & 36 & 0.0026 & 14.60 & 47 & 20 & 0.0013 & 12.74\\
51 & 48 & 0.0038 & 16.33 & 49 & 32 & 0.0023 & 14.28 & 47 & 16 & 0.0011 & 12.45\\
51 & 44 & 0.0035 & 15.99 & 49 & 28 & 0.0020 & 13.97 & 47 & 12 & 0.0008 & 12.16\\
51 & 40 & 0.0031 & 15.65 & 49 & 24 & 0.0017 & 13.65 & 46 & 100 & 0.0066 & 18.12\\
51 & 36 & 0.0028 & 15.30 & 49 & 20 & 0.0015 & 13.34 & 46 & 96 & 0.0063 & 17.84\\
51 & 32 & 0.0025 & 14.96 & 49 & 16 & 0.0012 & 13.02 & 46 & 92 & 0.0060 & 17.55\\
51 & 28 & 0.0022 & 14.62 & 49 & 12 & 0.0009 & 12.70 & 46 & 88 & 0.0058 & 17.27\\
51 & 24 & 0.0019 & 14.28 & 48 & 100 & 0.0071 & 19.17 & 46 & 84 & 0.0055 & 16.98\\
51 & 20 & 0.0016 & 13.94 & 48 & 96 & 0.0068 & 18.86 & 46 & 80 & 0.0052 & 16.70\\
51 & 16 & 0.0013 & 13.60 & 48 & 92 & 0.0065 & 18.55 & 46 & 76 & 0.0050 & 16.41\\
51 & 12 & 0.0009 & 13.26 & 48 & 88 & 0.0062 & 18.24 & 46 & 72 & 0.0047 & 16.13\\
50 & 100 & 0.0076 & 20.26 & 48 & 84 & 0.0059 & 17.94 & 46 & 68 & 0.0044 & 15.84\\
50 & 96 & 0.0073 & 19.93 & 48 & 80 & 0.0056 & 17.63 & 46 & 64 & 0.0042 & 15.56\\
50 & 92 & 0.0070 & 19.59 & 48 & 76 & 0.0054 & 17.32 & 46 & 60 & 0.0039 & 15.27\\
50 & 88 & 0.0067 & 19.26 & 48 & 72 & 0.0051 & 17.01 & 46 & 56 & 0.0037 & 14.99\\
50 & 84 & 0.0064 & 18.92 & 48 & 68 & 0.0048 & 16.70 & 46 & 52 & 0.0034 & 14.71\\
50 & 80 & 0.0061 & 18.59 & 48 & 64 & 0.0045 & 16.40 & 46 & 48 & 0.0031 & 14.42\\
50 & 76 & 0.0058 & 18.26 & 48 & 60 & 0.0042 & 16.09 & 46 & 44 & 0.0029 & 14.14\\
50 & 72 & 0.0055 & 17.93 & 48 & 56 & 0.0039 & 15.78 & 46 & 40 & 0.0026 & 13.86\\
50 & 68 & 0.0052 & 17.59 & 48 & 52 & 0.0037 & 15.48 & 46 & 36 & 0.0023 & 13.57\\
50 & 64 & 0.0049 & 17.26 & 48 & 48 & 0.0034 & 15.17 & 46 & 32 & 0.0021 & 13.29\\
50 & 60 & 0.0046 & 16.93 & 48 & 44 & 0.0031 & 14.87 & 46 & 28 & 0.0018 & 13.01\\
50 & 56 & 0.0042 & 16.60 & 48 & 40 & 0.0028 & 14.56 & 46 & 24 & 0.0016 & 12.73\\
50 & 52 & 0.0039 & 16.27 & 48 & 36 & 0.0025 & 14.25 & 46 & 20 & 0.0013 & 12.45\\
50 & 48 & 0.0036 & 15.94 & 48 & 32 & 0.0022 & 13.95 & 46 & 16 & 0.0010 & 12.16\\
50 & 44 & 0.0033 & 15.61 & 48 & 28 & 0.0020 & 13.65 & 46 & 12 & 0.0008 & 11.88\\
50 & 40 & 0.0030 & 15.28 & 48 & 24 & 0.0017 & 13.34 & 45 & 100 & 0.0063 & 17.61\\
50 & 36 & 0.0027 & 14.95 & 48 & 20 & 0.0014 & 13.04 & 45 & 96 & 0.0060 & 17.34\\
50 & 32 & 0.0024 & 14.62 & 48 & 16 & 0.0011 & 12.73 & 45 & 92 & 0.0058 & 17.06\\
50 & 28 & 0.0021 & 14.29 & 48 & 12 & 0.0008 & 12.43 & 45 & 88 & 0.0055 & 16.79\\
50 & 24 & 0.0018 & 13.96 & 47 & 100 & 0.0068 & 18.64 & 45 & 84 & 0.0053 & 16.52\\
50 & 20 & 0.0015 & 13.64 & 47 & 96 & 0.0065 & 18.35 & 45 & 80 & 0.0050 & 16.24\\
50 & 16 & 0.0012 & 13.31 & 47 & 92 & 0.0063 & 18.05 & 45 & 76 & 0.0048 & 15.97\\
50 & 12 & 0.0009 & 12.98 & 47 & 88 & 0.0060 & 17.75 & 45 & 72 & 0.0045 & 15.69\\
49 & 100 & 0.0073 & 19.71 & 47 & 84 & 0.0057 & 17.45 & 45 & 68 & 0.0043 & 15.42\\
49 & 96 & 0.0070 & 19.39 & 47 & 80 & 0.0054 & 17.16 & 45 & 64 & 0.0040 & 15.15\\
49 & 92 & 0.0067 & 19.07 & 47 & 76 & 0.0052 & 16.86 & 45 & 60 & 0.0038 & 14.87\\
49 & 88 & 0.0065 & 18.75 & 47 & 72 & 0.0049 & 16.57 & 45 & 56 & 0.0035 & 14.60\\
49 & 84 & 0.0062 & 18.43 & 47 & 68 & 0.0046 & 16.27 & 45 & 52 & 0.0033 & 14.33\\
49 & 80 & 0.0059 & 18.11 & 47 & 64 & 0.0043 & 15.97 & 45 & 48 & 0.0030 & 14.05\\
49 & 76 & 0.0056 & 17.79 & 47 & 60 & 0.0041 & 15.68 & 45 & 44 & 0.0028 & 13.78\\
49 & 72 & 0.0053 & 17.47 & 47 & 56 & 0.0038 & 15.38 & 45 & 40 & 0.0025 & 13.51\\
49 & 68 & 0.0050 & 17.15 & 47 & 52 & 0.0035 & 15.09 & 45 & 36 & 0.0023 & 13.24\\
49 & 64 & 0.0047 & 16.83 & 47 & 48 & 0.0032 & 14.79 & 45 & 32 & 0.0020 & 12.97\\
49 & 60 & 0.0044 & 16.51 & 47 & 44 & 0.0030 & 14.50 & 45 & 28 & 0.0018 & 12.69\\
49 & 56 & 0.0041 & 16.19 & 47 & 40 & 0.0027 & 14.21 & 45 & 24 & 0.0015 & 12.42\\
49 & 52 & 0.0038 & 15.87 & 47 & 36 & 0.0024 & 13.91 & 45 & 20 & 0.0013 & 12.15\\
49 & 48 & 0.0035 & 15.55 & 47 & 32 & 0.0022 & 13.62 & 45 & 16 & 0.0010 & 11.88\\
49 & 44 & 0.0032 & 15.24 & 47 & 28 & 0.0019 & 13.33 & 45 & 12 & 0.0007 & 11.61\\
\bottomrule
\end{tabular}
\newpage
\begin{tabular}{llll|llll|llll}
 \toprule 
\(T_{db}\) & \(\phi\) & \(\omega\) & \(h\) & \(T_{db}\) & \(\phi\) & \(\omega\) & \(h\) & \(T_{db}\) & \(\phi\) & \(\omega\) & \(h\)  \\ \midrule 
44 & 100 & 0.0061 & 17.11 & 42 & 30 & 0.0017 & 11.89 & 39 & 48 & 0.0024 & 11.93\\
44 & 95 & 0.0058 & 16.78 & 42 & 25 & 0.0014 & 11.58 & 39 & 42 & 0.0021 & 11.61\\
44 & 90 & 0.0055 & 16.45 & 42 & 20 & 0.0011 & 11.28 & 39 & 36 & 0.0018 & 11.29\\
44 & 85 & 0.0051 & 16.12 & 42 & 15 & 0.0008 & 10.98 & 39 & 30 & 0.0015 & 10.97\\
44 & 80 & 0.0048 & 15.79 & 41 & 100 & 0.0054 & 15.67 & 39 & 24 & 0.0012 & 10.64\\
44 & 75 & 0.0045 & 15.46 & 41 & 95 & 0.0051 & 15.37 & 39 & 18 & 0.0009 & 10.32\\
44 & 70 & 0.0042 & 15.13 & 41 & 90 & 0.0049 & 15.08 & 39 & 12 & 0.0006 & 10.00\\
44 & 65 & 0.0039 & 14.81 & 41 & 85 & 0.0046 & 14.79 & 38 & 100 & 0.0048 & 14.29\\
44 & 60 & 0.0036 & 14.48 & 41 & 80 & 0.0043 & 14.49 & 38 & 94 & 0.0045 & 13.98\\
44 & 55 & 0.0033 & 14.15 & 41 & 75 & 0.0040 & 14.20 & 38 & 88 & 0.0042 & 13.67\\
44 & 50 & 0.0030 & 13.82 & 41 & 70 & 0.0038 & 13.91 & 38 & 82 & 0.0039 & 13.35\\
44 & 45 & 0.0027 & 13.49 & 41 & 65 & 0.0035 & 13.62 & 38 & 76 & 0.0036 & 13.04\\
44 & 40 & 0.0024 & 13.17 & 41 & 60 & 0.0032 & 13.32 & 38 & 70 & 0.0034 & 12.73\\
44 & 35 & 0.0021 & 12.84 & 41 & 55 & 0.0030 & 13.03 & 38 & 64 & 0.0031 & 12.42\\
44 & 30 & 0.0018 & 12.51 & 41 & 50 & 0.0027 & 12.74 & 38 & 58 & 0.0028 & 12.11\\
44 & 25 & 0.0015 & 12.19 & 41 & 45 & 0.0024 & 12.45 & 38 & 52 & 0.0025 & 11.80\\
44 & 20 & 0.0012 & 11.86 & 41 & 40 & 0.0021 & 12.16 & 38 & 46 & 0.0022 & 11.49\\
44 & 15 & 0.0009 & 11.54 & 41 & 35 & 0.0019 & 11.87 & 38 & 40 & 0.0019 & 11.18\\
43 & 100 & 0.0058 & 16.62 & 41 & 30 & 0.0016 & 11.58 & 38 & 34 & 0.0016 & 10.87\\
43 & 95 & 0.0055 & 16.31 & 41 & 25 & 0.0013 & 11.29 & 38 & 28 & 0.0013 & 10.56\\
43 & 90 & 0.0052 & 15.99 & 41 & 20 & 0.0011 & 11.00 & 38 & 22 & 0.0010 & 10.25\\
43 & 85 & 0.0050 & 15.67 & 41 & 15 & 0.0008 & 10.71 & 38 & 16 & 0.0008 & 9.94\\
43 & 80 & 0.0047 & 15.35 & 40 & 100 & 0.0052 & 15.20 & 37 & 100 & 0.0046 & 13.85\\
43 & 75 & 0.0044 & 15.04 & 40 & 95 & 0.0049 & 14.92 & 37 & 94 & 0.0043 & 13.55\\
43 & 70 & 0.0041 & 14.72 & 40 & 90 & 0.0047 & 14.64 & 37 & 88 & 0.0041 & 13.25\\
43 & 65 & 0.0038 & 14.40 & 40 & 85 & 0.0044 & 14.35 & 37 & 82 & 0.0038 & 12.95\\
43 & 60 & 0.0035 & 14.09 & 40 & 80 & 0.0041 & 14.07 & 37 & 76 & 0.0035 & 12.65\\
43 & 55 & 0.0032 & 13.77 & 40 & 75 & 0.0039 & 13.79 & 37 & 70 & 0.0032 & 12.35\\
43 & 50 & 0.0029 & 13.46 & 40 & 70 & 0.0036 & 13.51 & 37 & 64 & 0.0029 & 12.05\\
43 & 45 & 0.0026 & 13.14 & 40 & 65 & 0.0034 & 13.23 & 37 & 58 & 0.0027 & 11.75\\
43 & 40 & 0.0023 & 12.83 & 40 & 60 & 0.0031 & 12.95 & 37 & 52 & 0.0024 & 11.45\\
43 & 35 & 0.0020 & 12.51 & 40 & 55 & 0.0028 & 12.67 & 37 & 46 & 0.0021 & 11.16\\
43 & 30 & 0.0017 & 12.20 & 40 & 50 & 0.0026 & 12.39 & 37 & 40 & 0.0018 & 10.86\\
43 & 25 & 0.0014 & 11.88 & 40 & 45 & 0.0023 & 12.11 & 37 & 34 & 0.0016 & 10.56\\
43 & 20 & 0.0012 & 11.57 & 40 & 40 & 0.0021 & 11.83 & 37 & 28 & 0.0013 & 10.26\\
43 & 15 & 0.0009 & 11.26 & 40 & 35 & 0.0018 & 11.55 & 37 & 22 & 0.0010 & 9.97\\
42 & 100 & 0.0056 & 16.14 & 40 & 30 & 0.0015 & 11.27 & 37 & 16 & 0.0007 & 9.67\\
42 & 95 & 0.0053 & 15.84 & 40 & 25 & 0.0013 & 10.99 & 36 & 100 & 0.0044 & 13.41\\
42 & 90 & 0.0050 & 15.53 & 40 & 20 & 0.0010 & 10.71 & 36 & 94 & 0.0042 & 13.12\\
42 & 85 & 0.0048 & 15.22 & 40 & 15 & 0.0008 & 10.43 & 36 & 88 & 0.0039 & 12.84\\
42 & 80 & 0.0045 & 14.92 & 39 & 100 & 0.0050 & 14.74 & 36 & 82 & 0.0036 & 12.55\\
42 & 75 & 0.0042 & 14.62 & 39 & 95 & 0.0047 & 14.47 & 36 & 76 & 0.0034 & 12.26\\
42 & 70 & 0.0039 & 14.31 & 39 & 90 & 0.0045 & 14.20 & 36 & 70 & 0.0031 & 11.97\\
42 & 65 & 0.0036 & 14.01 & 39 & 85 & 0.0042 & 13.93 & 36 & 64 & 0.0028 & 11.69\\
42 & 60 & 0.0034 & 13.70 & 39 & 80 & 0.0040 & 13.66 & 36 & 58 & 0.0026 & 11.40\\
42 & 55 & 0.0031 & 13.40 & 39 & 75 & 0.0037 & 13.39 & 36 & 52 & 0.0023 & 11.11\\
42 & 50 & 0.0028 & 13.10 & 39 & 70 & 0.0035 & 13.12 & 36 & 46 & 0.0020 & 10.83\\
42 & 45 & 0.0025 & 12.79 & 39 & 65 & 0.0032 & 12.85 & 36 & 40 & 0.0018 & 10.54\\
42 & 40 & 0.0022 & 12.49 & 39 & 60 & 0.0030 & 12.58 & 36 & 34 & 0.0015 & 10.26\\
42 & 35 & 0.0020 & 12.19 & 39 & 54 & 0.0027 & 12.26 & 36 & 28 & 0.0012 & 9.97\\
\bottomrule
\end{tabular}
\newpage
\begin{tabular}{llll|llll|llll}
 \toprule 
\(T_{db}\) & \(\phi\) & \(\omega\) & \(h\) & \(T_{db}\) & \(\phi\) & \(\omega\) & \(h\) & \(T_{db}\) & \(\phi\) & \(\omega\) & \(h\)  \\ \midrule 
36 & 22 & 0.0010 & 9.68 & 32 & 51 & 0.0019 & 9.74 &  &  &  & \\
36 & 16 & 0.0007 & 9.40 & 32 & 44 & 0.0017 & 9.46 &  &  &  & \\
35 & 100 & 0.0043 & 12.98 & 32 & 37 & 0.0014 & 9.18 &  &  &  & \\
35 & 94 & 0.0040 & 12.71 & 32 & 30 & 0.0011 & 8.89 &  &  &  & \\
35 & 88 & 0.0037 & 12.43 & 32 & 23 & 0.0009 & 8.61 &  &  &  & \\
35 & 82 & 0.0035 & 12.15 & 32 & 16 & 0.0006 & 8.33 &  &  &  & \\
35 & 76 & 0.0032 & 11.88 &  &  &  &  &  &  &  & \\
35 & 70 & 0.0030 & 11.60 &  &  &  &  &  &  &  & \\
35 & 64 & 0.0027 & 11.33 &  &  &  &  &  &  &  & \\
35 & 58 & 0.0025 & 11.05 &  &  &  &  &  &  &  & \\
35 & 52 & 0.0022 & 10.78 &  &  &  &  &  &  &  & \\
35 & 46 & 0.0020 & 10.50 &  &  &  &  &  &  &  & \\
35 & 40 & 0.0017 & 10.23 &  &  &  &  &  &  &  & \\
35 & 34 & 0.0014 & 9.95 &  &  &  &  &  &  &  & \\
35 & 28 & 0.0012 & 9.68 &  &  &  &  &  &  &  & \\
35 & 22 & 0.0009 & 9.40 &  &  &  &  &  &  &  & \\
35 & 16 & 0.0007 & 9.13 &  &  &  &  &  &  &  & \\
34 & 100 & 0.0041 & 12.56 &  &  &  &  &  &  &  & \\
34 & 93 & 0.0038 & 12.25 &  &  &  &  &  &  &  & \\
34 & 86 & 0.0035 & 11.94 &  &  &  &  &  &  &  & \\
34 & 79 & 0.0032 & 11.63 &  &  &  &  &  &  &  & \\
34 & 72 & 0.0029 & 11.32 &  &  &  &  &  &  &  & \\
34 & 65 & 0.0027 & 11.01 &  &  &  &  &  &  &  & \\
34 & 58 & 0.0024 & 10.71 &  &  &  &  &  &  &  & \\
34 & 51 & 0.0021 & 10.40 &  &  &  &  &  &  &  & \\
34 & 44 & 0.0018 & 10.09 &  &  &  &  &  &  &  & \\
34 & 37 & 0.0015 & 9.78 &  &  &  &  &  &  &  & \\
34 & 30 & 0.0012 & 9.47 &  &  &  &  &  &  &  & \\
34 & 23 & 0.0009 & 9.17 &  &  &  &  &  &  &  & \\
34 & 16 & 0.0007 & 8.86 &  &  &  &  &  &  &  & \\
33 & 100 & 0.0039 & 12.15 &  &  &  &  &  &  &  & \\
33 & 93 & 0.0037 & 11.85 &  &  &  &  &  &  &  & \\
33 & 86 & 0.0034 & 11.55 &  &  &  &  &  &  &  & \\
33 & 79 & 0.0031 & 11.25 &  &  &  &  &  &  &  & \\
33 & 72 & 0.0028 & 10.96 &  &  &  &  &  &  &  & \\
33 & 65 & 0.0025 & 10.66 &  &  &  &  &  &  &  & \\
33 & 58 & 0.0023 & 10.36 &  &  &  &  &  &  &  & \\
33 & 51 & 0.0020 & 10.07 &  &  &  &  &  &  &  & \\
33 & 44 & 0.0017 & 9.77 &  &  &  &  &  &  &  & \\
33 & 37 & 0.0014 & 9.48 &  &  &  &  &  &  &  & \\
33 & 30 & 0.0012 & 9.18 &  &  &  &  &  &  &  & \\
33 & 23 & 0.0009 & 8.89 &  &  &  &  &  &  &  & \\
33 & 16 & 0.0006 & 8.59 &  &  &  &  &  &  &  & \\
32 & 100 & 0.0038 & 11.74 &  &  &  &  &  &  &  & \\
32 & 93 & 0.0035 & 11.45 &  &  &  &  &  &  &  & \\
32 & 86 & 0.0032 & 11.17 &  &  &  &  &  &  &  & \\
32 & 79 & 0.0030 & 10.88 &  &  &  &  &  &  &  & \\
32 & 72 & 0.0027 & 10.60 &  &  &  &  &  &  &  & \\
32 & 65 & 0.0024 & 10.31 &  &  &  &  &  &  &  & \\
32 & 58 & 0.0022 & 10.03 &  &  &  &  &  &  &  & \\
\bottomrule
\end{tabular}
\newpage


}


\chapter{  \(T_{db}\) and \(T_{dp}\) }

\newpage
{
\small

\begin{tabular}{llll|llll|llll}
 \toprule 
\(T_{db}\) & \(T_{dp}\) & \(\omega\) & \(h\) & \(T_{db}\) & \(T_{dp}\) & \(\omega\) & \(h\) & \(T_{db}\) & \(T_{dp}\) & \(\omega\) & \(h\)  \\ \midrule 
100 & 91 & 0.0321 & 59.45 & 100 & 41 & 0.0054 & 29.97 & 99 & 51 & 0.0079 & 32.51\\
100 & 90 & 0.0310 & 58.31 & 100 & 40 & 0.0052 & 29.74 & 99 & 50 & 0.0076 & 32.19\\
100 & 89 & 0.0300 & 57.19 & 100 & 39 & 0.0050 & 29.52 & 99 & 49 & 0.0073 & 31.87\\
100 & 88 & 0.0290 & 56.11 & 100 & 38 & 0.0048 & 29.30 & 99 & 48 & 0.0071 & 31.57\\
100 & 87 & 0.0281 & 55.07 & 100 & 37 & 0.0046 & 29.10 & 99 & 47 & 0.0068 & 31.28\\
100 & 86 & 0.0272 & 54.05 & 100 & 36 & 0.0044 & 28.90 & 99 & 46 & 0.0066 & 31.00\\
100 & 85 & 0.0263 & 53.07 & 100 & 35 & 0.0043 & 28.71 & 99 & 45 & 0.0063 & 30.73\\
100 & 84 & 0.0254 & 52.11 & 100 & 34 & 0.0041 & 28.52 & 99 & 44 & 0.0061 & 30.46\\
100 & 83 & 0.0246 & 51.18 & 100 & 33 & 0.0039 & 28.34 & 99 & 43 & 0.0058 & 30.21\\
100 & 82 & 0.0238 & 50.29 & 100 & 32 & 0.0038 & 28.17 & 99 & 42 & 0.0056 & 29.96\\
100 & 81 & 0.0230 & 49.41 & 99 & 91 & 0.0321 & 59.20 & 99 & 41 & 0.0054 & 29.73\\
100 & 80 & 0.0222 & 48.57 & 99 & 90 & 0.0310 & 58.05 & 99 & 40 & 0.0052 & 29.50\\
100 & 79 & 0.0215 & 47.75 & 99 & 89 & 0.0300 & 56.94 & 99 & 39 & 0.0050 & 29.28\\
100 & 78 & 0.0208 & 46.96 & 99 & 88 & 0.0290 & 55.86 & 99 & 38 & 0.0048 & 29.06\\
100 & 77 & 0.0201 & 46.18 & 99 & 87 & 0.0281 & 54.82 & 99 & 37 & 0.0046 & 28.86\\
100 & 76 & 0.0194 & 45.44 & 99 & 86 & 0.0272 & 53.80 & 99 & 36 & 0.0044 & 28.66\\
100 & 75 & 0.0187 & 44.71 & 99 & 85 & 0.0263 & 52.82 & 99 & 35 & 0.0043 & 28.46\\
100 & 74 & 0.0181 & 44.01 & 99 & 84 & 0.0254 & 51.86 & 99 & 34 & 0.0041 & 28.28\\
100 & 73 & 0.0175 & 43.33 & 99 & 83 & 0.0246 & 50.93 & 99 & 33 & 0.0039 & 28.10\\
100 & 72 & 0.0169 & 42.67 & 99 & 82 & 0.0238 & 50.04 & 99 & 32 & 0.0038 & 27.93\\
100 & 71 & 0.0163 & 42.03 & 99 & 81 & 0.0230 & 49.16 & 98 & 91 & 0.0321 & 58.95\\
100 & 70 & 0.0158 & 41.42 & 99 & 80 & 0.0222 & 48.32 & 98 & 90 & 0.0310 & 57.80\\
100 & 69 & 0.0152 & 40.82 & 99 & 79 & 0.0215 & 47.50 & 98 & 89 & 0.0300 & 56.69\\
100 & 68 & 0.0147 & 40.23 & 99 & 78 & 0.0208 & 46.71 & 98 & 88 & 0.0290 & 55.61\\
100 & 67 & 0.0142 & 39.67 & 99 & 77 & 0.0201 & 45.94 & 98 & 87 & 0.0281 & 54.56\\
100 & 66 & 0.0137 & 39.13 & 99 & 76 & 0.0194 & 45.19 & 98 & 86 & 0.0272 & 53.55\\
100 & 65 & 0.0132 & 38.60 & 99 & 75 & 0.0187 & 44.47 & 98 & 85 & 0.0263 & 52.56\\
100 & 64 & 0.0127 & 38.09 & 99 & 74 & 0.0181 & 43.76 & 98 & 84 & 0.0254 & 51.61\\
100 & 63 & 0.0123 & 37.59 & 99 & 73 & 0.0175 & 43.08 & 98 & 83 & 0.0246 & 50.68\\
100 & 62 & 0.0119 & 37.11 & 99 & 72 & 0.0169 & 42.43 & 98 & 82 & 0.0238 & 49.78\\
100 & 61 & 0.0114 & 36.65 & 99 & 71 & 0.0163 & 41.79 & 98 & 81 & 0.0230 & 48.91\\
100 & 60 & 0.0110 & 36.20 & 99 & 70 & 0.0158 & 41.17 & 98 & 80 & 0.0222 & 48.07\\
100 & 59 & 0.0106 & 35.76 & 99 & 69 & 0.0152 & 40.57 & 98 & 79 & 0.0215 & 47.25\\
100 & 58 & 0.0103 & 35.34 & 99 & 68 & 0.0147 & 39.99 & 98 & 78 & 0.0208 & 46.46\\
100 & 57 & 0.0099 & 34.93 & 99 & 67 & 0.0142 & 39.43 & 98 & 77 & 0.0201 & 45.69\\
100 & 56 & 0.0095 & 34.54 & 99 & 66 & 0.0137 & 38.88 & 98 & 76 & 0.0194 & 44.94\\
100 & 55 & 0.0092 & 34.16 & 99 & 65 & 0.0132 & 38.35 & 98 & 75 & 0.0187 & 44.22\\
100 & 54 & 0.0089 & 33.79 & 99 & 64 & 0.0127 & 37.84 & 98 & 74 & 0.0181 & 43.52\\
100 & 53 & 0.0085 & 33.43 & 99 & 63 & 0.0123 & 37.35 & 98 & 73 & 0.0175 & 42.84\\
100 & 52 & 0.0082 & 33.09 & 99 & 62 & 0.0119 & 36.87 & 98 & 72 & 0.0169 & 42.18\\
100 & 51 & 0.0079 & 32.75 & 99 & 61 & 0.0114 & 36.40 & 98 & 71 & 0.0163 & 41.54\\
100 & 50 & 0.0076 & 32.43 & 99 & 60 & 0.0110 & 35.95 & 98 & 70 & 0.0158 & 40.92\\
100 & 49 & 0.0073 & 32.12 & 99 & 59 & 0.0106 & 35.52 & 98 & 69 & 0.0152 & 40.32\\
100 & 48 & 0.0071 & 31.82 & 99 & 58 & 0.0103 & 35.10 & 98 & 68 & 0.0147 & 39.74\\
100 & 47 & 0.0068 & 31.52 & 99 & 57 & 0.0099 & 34.69 & 98 & 67 & 0.0142 & 39.18\\
100 & 46 & 0.0066 & 31.24 & 99 & 56 & 0.0095 & 34.30 & 98 & 66 & 0.0137 & 38.63\\
100 & 45 & 0.0063 & 30.97 & 99 & 55 & 0.0092 & 33.91 & 98 & 65 & 0.0132 & 38.11\\
100 & 44 & 0.0061 & 30.71 & 99 & 54 & 0.0089 & 33.55 & 98 & 64 & 0.0127 & 37.60\\
100 & 43 & 0.0058 & 30.45 & 99 & 53 & 0.0085 & 33.19 & 98 & 63 & 0.0123 & 37.10\\
100 & 42 & 0.0056 & 30.21 & 99 & 52 & 0.0082 & 32.84 & 98 & 62 & 0.0119 & 36.62\\
\bottomrule
\end{tabular}
\newpage
\begin{tabular}{llll|llll|llll}
 \toprule 
\(T_{db}\) & \(T_{dp}\) & \(\omega\) & \(h\) & \(T_{db}\) & \(T_{dp}\) & \(\omega\) & \(h\) & \(T_{db}\) & \(T_{dp}\) & \(\omega\) & \(h\)  \\ \midrule 
98 & 61 & 0.0114 & 36.16 & 97 & 72 & 0.0169 & 41.93 & 96 & 83 & 0.0246 & 50.18\\
98 & 60 & 0.0110 & 35.71 & 97 & 71 & 0.0163 & 41.29 & 96 & 82 & 0.0238 & 49.28\\
98 & 59 & 0.0106 & 35.27 & 97 & 70 & 0.0158 & 40.67 & 96 & 81 & 0.0230 & 48.41\\
98 & 58 & 0.0103 & 34.85 & 97 & 69 & 0.0152 & 40.08 & 96 & 80 & 0.0222 & 47.57\\
98 & 57 & 0.0099 & 34.45 & 97 & 68 & 0.0147 & 39.50 & 96 & 79 & 0.0215 & 46.75\\
98 & 56 & 0.0095 & 34.05 & 97 & 67 & 0.0142 & 38.93 & 96 & 78 & 0.0208 & 45.96\\
98 & 55 & 0.0092 & 33.67 & 97 & 66 & 0.0137 & 38.39 & 96 & 77 & 0.0201 & 45.19\\
98 & 54 & 0.0089 & 33.30 & 97 & 65 & 0.0132 & 37.86 & 96 & 76 & 0.0194 & 44.44\\
98 & 53 & 0.0085 & 32.94 & 97 & 64 & 0.0127 & 37.35 & 96 & 75 & 0.0187 & 43.72\\
98 & 52 & 0.0082 & 32.60 & 97 & 63 & 0.0123 & 36.86 & 96 & 74 & 0.0181 & 43.02\\
98 & 51 & 0.0079 & 32.27 & 97 & 62 & 0.0119 & 36.38 & 96 & 73 & 0.0175 & 42.34\\
98 & 50 & 0.0076 & 31.94 & 97 & 61 & 0.0114 & 35.91 & 96 & 72 & 0.0169 & 41.68\\
98 & 49 & 0.0073 & 31.63 & 97 & 60 & 0.0110 & 35.46 & 96 & 71 & 0.0163 & 41.05\\
98 & 48 & 0.0071 & 31.33 & 97 & 59 & 0.0106 & 35.03 & 96 & 70 & 0.0158 & 40.43\\
98 & 47 & 0.0068 & 31.04 & 97 & 58 & 0.0103 & 34.61 & 96 & 69 & 0.0152 & 39.83\\
98 & 46 & 0.0066 & 30.76 & 97 & 57 & 0.0099 & 34.20 & 96 & 68 & 0.0147 & 39.25\\
98 & 45 & 0.0063 & 30.48 & 97 & 56 & 0.0095 & 33.81 & 96 & 67 & 0.0142 & 38.69\\
98 & 44 & 0.0061 & 30.22 & 97 & 55 & 0.0092 & 33.43 & 96 & 66 & 0.0137 & 38.14\\
98 & 43 & 0.0058 & 29.97 & 97 & 54 & 0.0089 & 33.06 & 96 & 65 & 0.0132 & 37.62\\
98 & 42 & 0.0056 & 29.72 & 97 & 53 & 0.0085 & 32.70 & 96 & 64 & 0.0127 & 37.10\\
98 & 41 & 0.0054 & 29.48 & 97 & 52 & 0.0082 & 32.36 & 96 & 63 & 0.0123 & 36.61\\
98 & 40 & 0.0052 & 29.25 & 97 & 51 & 0.0079 & 32.02 & 96 & 62 & 0.0119 & 36.13\\
98 & 39 & 0.0050 & 29.03 & 97 & 50 & 0.0076 & 31.70 & 96 & 61 & 0.0114 & 35.67\\
98 & 38 & 0.0048 & 28.82 & 97 & 49 & 0.0073 & 31.39 & 96 & 60 & 0.0110 & 35.22\\
98 & 37 & 0.0046 & 28.61 & 97 & 48 & 0.0071 & 31.09 & 96 & 59 & 0.0106 & 34.78\\
98 & 36 & 0.0044 & 28.41 & 97 & 47 & 0.0068 & 30.79 & 96 & 58 & 0.0103 & 34.36\\
98 & 35 & 0.0043 & 28.22 & 97 & 46 & 0.0066 & 30.51 & 96 & 57 & 0.0099 & 33.96\\
98 & 34 & 0.0041 & 28.04 & 97 & 45 & 0.0063 & 30.24 & 96 & 56 & 0.0095 & 33.56\\
98 & 33 & 0.0039 & 27.86 & 97 & 44 & 0.0061 & 29.98 & 96 & 55 & 0.0092 & 33.18\\
98 & 32 & 0.0038 & 27.69 & 97 & 43 & 0.0058 & 29.72 & 96 & 54 & 0.0089 & 32.81\\
97 & 92 & 0.0331 & 59.87 & 97 & 42 & 0.0056 & 29.48 & 96 & 53 & 0.0085 & 32.46\\
97 & 91 & 0.0321 & 58.69 & 97 & 41 & 0.0054 & 29.24 & 96 & 52 & 0.0082 & 32.11\\
97 & 90 & 0.0310 & 57.55 & 97 & 40 & 0.0052 & 29.01 & 96 & 51 & 0.0079 & 31.78\\
97 & 89 & 0.0300 & 56.43 & 97 & 39 & 0.0050 & 28.79 & 96 & 50 & 0.0076 & 31.46\\
97 & 88 & 0.0290 & 55.36 & 97 & 38 & 0.0048 & 28.58 & 96 & 49 & 0.0073 & 31.14\\
97 & 87 & 0.0281 & 54.31 & 97 & 37 & 0.0046 & 28.37 & 96 & 48 & 0.0071 & 30.84\\
97 & 86 & 0.0272 & 53.30 & 97 & 36 & 0.0044 & 28.17 & 96 & 47 & 0.0068 & 30.55\\
97 & 85 & 0.0263 & 52.31 & 97 & 35 & 0.0043 & 27.98 & 96 & 46 & 0.0066 & 30.27\\
97 & 84 & 0.0254 & 51.36 & 97 & 34 & 0.0041 & 27.80 & 96 & 45 & 0.0063 & 30.00\\
97 & 83 & 0.0246 & 50.43 & 97 & 33 & 0.0039 & 27.62 & 96 & 44 & 0.0061 & 29.73\\
97 & 82 & 0.0238 & 49.53 & 97 & 32 & 0.0038 & 27.45 & 96 & 43 & 0.0058 & 29.48\\
97 & 81 & 0.0230 & 48.66 & 96 & 92 & 0.0331 & 59.62 & 96 & 42 & 0.0056 & 29.24\\
97 & 80 & 0.0222 & 47.82 & 96 & 91 & 0.0321 & 58.44 & 96 & 41 & 0.0054 & 29.00\\
97 & 79 & 0.0215 & 47.00 & 96 & 90 & 0.0310 & 57.29 & 96 & 40 & 0.0052 & 28.77\\
97 & 78 & 0.0208 & 46.21 & 96 & 89 & 0.0300 & 56.18 & 96 & 39 & 0.0050 & 28.55\\
97 & 77 & 0.0201 & 45.44 & 96 & 88 & 0.0290 & 55.10 & 96 & 38 & 0.0048 & 28.34\\
97 & 76 & 0.0194 & 44.69 & 96 & 87 & 0.0281 & 54.06 & 96 & 37 & 0.0046 & 28.13\\
97 & 75 & 0.0187 & 43.97 & 96 & 86 & 0.0272 & 53.04 & 96 & 36 & 0.0044 & 27.93\\
97 & 74 & 0.0181 & 43.27 & 96 & 85 & 0.0263 & 52.06 & 96 & 35 & 0.0043 & 27.74\\
97 & 73 & 0.0175 & 42.59 & 96 & 84 & 0.0254 & 51.11 & 96 & 34 & 0.0041 & 27.55\\
\bottomrule
\end{tabular}
\newpage
\begin{tabular}{llll|llll|llll}
 \toprule 
\(T_{db}\) & \(T_{dp}\) & \(\omega\) & \(h\) & \(T_{db}\) & \(T_{dp}\) & \(\omega\) & \(h\) & \(T_{db}\) & \(T_{dp}\) & \(\omega\) & \(h\)  \\ \midrule 
96 & 33 & 0.0039 & 27.38 & 95 & 44 & 0.0061 & 29.49 & 94 & 55 & 0.0092 & 32.69\\
96 & 32 & 0.0038 & 27.20 & 95 & 43 & 0.0058 & 29.24 & 94 & 54 & 0.0089 & 32.33\\
95 & 92 & 0.0331 & 59.36 & 95 & 42 & 0.0056 & 28.99 & 94 & 53 & 0.0085 & 31.97\\
95 & 91 & 0.0321 & 58.18 & 95 & 41 & 0.0054 & 28.76 & 94 & 52 & 0.0082 & 31.62\\
95 & 90 & 0.0310 & 57.04 & 95 & 40 & 0.0052 & 28.53 & 94 & 51 & 0.0079 & 31.29\\
95 & 89 & 0.0300 & 55.93 & 95 & 39 & 0.0050 & 28.31 & 94 & 50 & 0.0076 & 30.97\\
95 & 88 & 0.0290 & 54.85 & 95 & 38 & 0.0048 & 28.09 & 94 & 49 & 0.0073 & 30.66\\
95 & 87 & 0.0281 & 53.80 & 95 & 37 & 0.0046 & 27.89 & 94 & 48 & 0.0071 & 30.36\\
95 & 86 & 0.0272 & 52.79 & 95 & 36 & 0.0044 & 27.69 & 94 & 47 & 0.0068 & 30.07\\
95 & 85 & 0.0263 & 51.81 & 95 & 35 & 0.0043 & 27.50 & 94 & 46 & 0.0066 & 29.78\\
95 & 84 & 0.0254 & 50.85 & 95 & 34 & 0.0041 & 27.31 & 94 & 45 & 0.0063 & 29.51\\
95 & 83 & 0.0246 & 49.93 & 95 & 33 & 0.0039 & 27.13 & 94 & 44 & 0.0061 & 29.25\\
95 & 82 & 0.0238 & 49.03 & 95 & 32 & 0.0038 & 26.96 & 94 & 43 & 0.0058 & 29.00\\
95 & 81 & 0.0230 & 48.16 & 94 & 92 & 0.0331 & 59.11 & 94 & 42 & 0.0056 & 28.75\\
95 & 80 & 0.0222 & 47.32 & 94 & 91 & 0.0321 & 57.93 & 94 & 41 & 0.0054 & 28.51\\
95 & 79 & 0.0215 & 46.50 & 94 & 90 & 0.0310 & 56.78 & 94 & 40 & 0.0052 & 28.29\\
95 & 78 & 0.0208 & 45.71 & 94 & 89 & 0.0300 & 55.67 & 94 & 39 & 0.0050 & 28.06\\
95 & 77 & 0.0201 & 44.94 & 94 & 88 & 0.0290 & 54.60 & 94 & 38 & 0.0048 & 27.85\\
95 & 76 & 0.0194 & 44.19 & 94 & 87 & 0.0281 & 53.55 & 94 & 37 & 0.0046 & 27.65\\
95 & 75 & 0.0187 & 43.47 & 94 & 86 & 0.0272 & 52.54 & 94 & 36 & 0.0044 & 27.45\\
95 & 74 & 0.0181 & 42.77 & 94 & 85 & 0.0263 & 51.56 & 94 & 35 & 0.0043 & 27.26\\
95 & 73 & 0.0175 & 42.09 & 94 & 84 & 0.0254 & 50.60 & 94 & 34 & 0.0041 & 27.07\\
95 & 72 & 0.0169 & 41.44 & 94 & 83 & 0.0246 & 49.68 & 94 & 33 & 0.0039 & 26.89\\
95 & 71 & 0.0163 & 40.80 & 94 & 82 & 0.0238 & 48.78 & 94 & 32 & 0.0038 & 26.72\\
95 & 70 & 0.0158 & 40.18 & 94 & 81 & 0.0230 & 47.91 & 93 & 92 & 0.0331 & 58.85\\
95 & 69 & 0.0152 & 39.58 & 94 & 80 & 0.0222 & 47.07 & 93 & 91 & 0.0321 & 57.67\\
95 & 68 & 0.0147 & 39.00 & 94 & 79 & 0.0215 & 46.25 & 93 & 90 & 0.0310 & 56.53\\
95 & 67 & 0.0142 & 38.44 & 94 & 78 & 0.0208 & 45.46 & 93 & 89 & 0.0300 & 55.42\\
95 & 66 & 0.0137 & 37.90 & 94 & 77 & 0.0201 & 44.69 & 93 & 88 & 0.0290 & 54.34\\
95 & 65 & 0.0132 & 37.37 & 94 & 76 & 0.0194 & 43.95 & 93 & 87 & 0.0281 & 53.30\\
95 & 64 & 0.0127 & 36.86 & 94 & 75 & 0.0187 & 43.22 & 93 & 86 & 0.0272 & 52.29\\
95 & 63 & 0.0123 & 36.36 & 94 & 74 & 0.0181 & 42.52 & 93 & 85 & 0.0263 & 51.30\\
95 & 62 & 0.0119 & 35.89 & 94 & 73 & 0.0175 & 41.85 & 93 & 84 & 0.0254 & 50.35\\
95 & 61 & 0.0114 & 35.42 & 94 & 72 & 0.0169 & 41.19 & 93 & 83 & 0.0246 & 49.43\\
95 & 60 & 0.0110 & 34.97 & 94 & 71 & 0.0163 & 40.55 & 93 & 82 & 0.0238 & 48.53\\
95 & 59 & 0.0106 & 34.54 & 94 & 70 & 0.0158 & 39.93 & 93 & 81 & 0.0230 & 47.66\\
95 & 58 & 0.0103 & 34.12 & 94 & 69 & 0.0152 & 39.34 & 93 & 80 & 0.0222 & 46.82\\
95 & 57 & 0.0099 & 33.71 & 94 & 68 & 0.0147 & 38.76 & 93 & 79 & 0.0215 & 46.00\\
95 & 56 & 0.0095 & 33.32 & 94 & 67 & 0.0142 & 38.19 & 93 & 78 & 0.0208 & 45.21\\
95 & 55 & 0.0092 & 32.94 & 94 & 66 & 0.0137 & 37.65 & 93 & 77 & 0.0201 & 44.44\\
95 & 54 & 0.0089 & 32.57 & 94 & 65 & 0.0132 & 37.12 & 93 & 76 & 0.0194 & 43.70\\
95 & 53 & 0.0085 & 32.21 & 94 & 64 & 0.0127 & 36.61 & 93 & 75 & 0.0187 & 42.98\\
95 & 52 & 0.0082 & 31.87 & 94 & 63 & 0.0123 & 36.12 & 93 & 74 & 0.0181 & 42.28\\
95 & 51 & 0.0079 & 31.54 & 94 & 62 & 0.0119 & 35.64 & 93 & 73 & 0.0175 & 41.60\\
95 & 50 & 0.0076 & 31.21 & 94 & 61 & 0.0114 & 35.18 & 93 & 72 & 0.0169 & 40.94\\
95 & 49 & 0.0073 & 30.90 & 94 & 60 & 0.0110 & 34.73 & 93 & 71 & 0.0163 & 40.30\\
95 & 48 & 0.0071 & 30.60 & 94 & 59 & 0.0106 & 34.29 & 93 & 70 & 0.0158 & 39.69\\
95 & 47 & 0.0068 & 30.31 & 94 & 58 & 0.0103 & 33.87 & 93 & 69 & 0.0152 & 39.09\\
95 & 46 & 0.0066 & 30.03 & 94 & 57 & 0.0099 & 33.47 & 93 & 68 & 0.0147 & 38.51\\
95 & 45 & 0.0063 & 29.76 & 94 & 56 & 0.0095 & 33.07 & 93 & 67 & 0.0142 & 37.95\\
\bottomrule
\end{tabular}
\newpage
\begin{tabular}{llll|llll|llll}
 \toprule 
\(T_{db}\) & \(T_{dp}\) & \(\omega\) & \(h\) & \(T_{db}\) & \(T_{dp}\) & \(\omega\) & \(h\) & \(T_{db}\) & \(T_{dp}\) & \(\omega\) & \(h\)  \\ \midrule 
93 & 66 & 0.0137 & 37.40 & 92 & 77 & 0.0201 & 44.19 & 91 & 87 & 0.0281 & 52.79\\
93 & 65 & 0.0132 & 36.88 & 92 & 76 & 0.0194 & 43.45 & 91 & 86 & 0.0272 & 51.78\\
93 & 64 & 0.0127 & 36.37 & 92 & 75 & 0.0187 & 42.73 & 91 & 85 & 0.0263 & 50.80\\
93 & 63 & 0.0123 & 35.87 & 92 & 74 & 0.0181 & 42.03 & 91 & 84 & 0.0254 & 49.85\\
93 & 62 & 0.0119 & 35.40 & 92 & 73 & 0.0175 & 41.35 & 91 & 83 & 0.0246 & 48.93\\
93 & 61 & 0.0114 & 34.93 & 92 & 72 & 0.0169 & 40.69 & 91 & 82 & 0.0238 & 48.03\\
93 & 60 & 0.0110 & 34.48 & 92 & 71 & 0.0163 & 40.06 & 91 & 81 & 0.0230 & 47.16\\
93 & 59 & 0.0106 & 34.05 & 92 & 70 & 0.0158 & 39.44 & 91 & 80 & 0.0222 & 46.32\\
93 & 58 & 0.0103 & 33.63 & 92 & 69 & 0.0152 & 38.84 & 91 & 79 & 0.0215 & 45.50\\
93 & 57 & 0.0099 & 33.22 & 92 & 68 & 0.0147 & 38.26 & 91 & 78 & 0.0208 & 44.71\\
93 & 56 & 0.0095 & 32.83 & 92 & 67 & 0.0142 & 37.70 & 91 & 77 & 0.0201 & 43.94\\
93 & 55 & 0.0092 & 32.45 & 92 & 66 & 0.0137 & 37.16 & 91 & 76 & 0.0194 & 43.20\\
93 & 54 & 0.0089 & 32.08 & 92 & 65 & 0.0132 & 36.63 & 91 & 75 & 0.0187 & 42.48\\
93 & 53 & 0.0085 & 31.73 & 92 & 64 & 0.0127 & 36.12 & 91 & 74 & 0.0181 & 41.78\\
93 & 52 & 0.0082 & 31.38 & 92 & 63 & 0.0123 & 35.63 & 91 & 73 & 0.0175 & 41.10\\
93 & 51 & 0.0079 & 31.05 & 92 & 62 & 0.0119 & 35.15 & 91 & 72 & 0.0169 & 40.45\\
93 & 50 & 0.0076 & 30.73 & 92 & 61 & 0.0114 & 34.69 & 91 & 71 & 0.0163 & 39.81\\
93 & 49 & 0.0073 & 30.41 & 92 & 60 & 0.0110 & 34.24 & 91 & 70 & 0.0158 & 39.19\\
93 & 48 & 0.0071 & 30.11 & 92 & 59 & 0.0106 & 33.81 & 91 & 69 & 0.0152 & 38.59\\
93 & 47 & 0.0068 & 29.82 & 92 & 58 & 0.0103 & 33.39 & 91 & 68 & 0.0147 & 38.02\\
93 & 46 & 0.0066 & 29.54 & 92 & 57 & 0.0099 & 32.98 & 91 & 67 & 0.0142 & 37.46\\
93 & 45 & 0.0063 & 29.27 & 92 & 56 & 0.0095 & 32.59 & 91 & 66 & 0.0137 & 36.91\\
93 & 44 & 0.0061 & 29.01 & 92 & 55 & 0.0092 & 32.21 & 91 & 65 & 0.0132 & 36.39\\
93 & 43 & 0.0058 & 28.75 & 92 & 54 & 0.0089 & 31.84 & 91 & 64 & 0.0127 & 35.88\\
93 & 42 & 0.0056 & 28.51 & 92 & 53 & 0.0085 & 31.48 & 91 & 63 & 0.0123 & 35.38\\
93 & 41 & 0.0054 & 28.27 & 92 & 52 & 0.0082 & 31.14 & 91 & 62 & 0.0119 & 34.90\\
93 & 40 & 0.0052 & 28.04 & 92 & 51 & 0.0079 & 30.80 & 91 & 61 & 0.0114 & 34.44\\
93 & 39 & 0.0050 & 27.82 & 92 & 50 & 0.0076 & 30.48 & 91 & 60 & 0.0110 & 33.99\\
93 & 38 & 0.0048 & 27.61 & 92 & 49 & 0.0073 & 30.17 & 91 & 59 & 0.0106 & 33.56\\
93 & 37 & 0.0046 & 27.40 & 92 & 48 & 0.0071 & 29.87 & 91 & 58 & 0.0103 & 33.14\\
93 & 36 & 0.0044 & 27.20 & 92 & 47 & 0.0068 & 29.58 & 91 & 57 & 0.0099 & 32.73\\
93 & 35 & 0.0043 & 27.01 & 92 & 46 & 0.0066 & 29.30 & 91 & 56 & 0.0095 & 32.34\\
93 & 34 & 0.0041 & 26.83 & 92 & 45 & 0.0063 & 29.03 & 91 & 55 & 0.0092 & 31.96\\
93 & 33 & 0.0039 & 26.65 & 92 & 44 & 0.0061 & 28.76 & 91 & 54 & 0.0089 & 31.59\\
93 & 32 & 0.0038 & 26.48 & 92 & 43 & 0.0058 & 28.51 & 91 & 53 & 0.0085 & 31.24\\
92 & 92 & 0.0331 & 58.60 & 92 & 42 & 0.0056 & 28.27 & 91 & 52 & 0.0082 & 30.89\\
92 & 91 & 0.0321 & 57.42 & 92 & 41 & 0.0054 & 28.03 & 91 & 51 & 0.0079 & 30.56\\
92 & 90 & 0.0310 & 56.28 & 92 & 40 & 0.0052 & 27.80 & 91 & 50 & 0.0076 & 30.24\\
92 & 89 & 0.0300 & 55.17 & 92 & 39 & 0.0050 & 27.58 & 91 & 49 & 0.0073 & 29.93\\
92 & 88 & 0.0290 & 54.09 & 92 & 38 & 0.0048 & 27.37 & 91 & 48 & 0.0071 & 29.63\\
92 & 87 & 0.0281 & 53.05 & 92 & 37 & 0.0046 & 27.16 & 91 & 47 & 0.0068 & 29.34\\
92 & 86 & 0.0272 & 52.04 & 92 & 36 & 0.0044 & 26.96 & 91 & 46 & 0.0066 & 29.06\\
92 & 85 & 0.0263 & 51.05 & 92 & 35 & 0.0043 & 26.77 & 91 & 45 & 0.0063 & 28.78\\
92 & 84 & 0.0254 & 50.10 & 92 & 34 & 0.0041 & 26.59 & 91 & 44 & 0.0061 & 28.52\\
92 & 83 & 0.0246 & 49.18 & 92 & 33 & 0.0039 & 26.41 & 91 & 43 & 0.0058 & 28.27\\
92 & 82 & 0.0238 & 48.28 & 92 & 32 & 0.0038 & 26.24 & 91 & 42 & 0.0056 & 28.02\\
92 & 81 & 0.0230 & 47.41 & 91 & 91 & 0.0321 & 57.17 & 91 & 41 & 0.0054 & 27.79\\
92 & 80 & 0.0222 & 46.57 & 91 & 90 & 0.0310 & 56.02 & 91 & 40 & 0.0052 & 27.56\\
92 & 79 & 0.0215 & 45.75 & 91 & 89 & 0.0300 & 54.91 & 91 & 39 & 0.0050 & 27.34\\
92 & 78 & 0.0208 & 44.96 & 91 & 88 & 0.0290 & 53.84 & 91 & 38 & 0.0048 & 27.12\\
\bottomrule
\end{tabular}
\newpage
\begin{tabular}{llll|llll|llll}
 \toprule 
\(T_{db}\) & \(T_{dp}\) & \(\omega\) & \(h\) & \(T_{db}\) & \(T_{dp}\) & \(\omega\) & \(h\) & \(T_{db}\) & \(T_{dp}\) & \(\omega\) & \(h\)  \\ \midrule 
91 & 37 & 0.0046 & 26.92 & 90 & 46 & 0.0066 & 28.81 & 89 & 54 & 0.0089 & 31.11\\
91 & 36 & 0.0044 & 26.72 & 90 & 45 & 0.0063 & 28.54 & 89 & 53 & 0.0085 & 30.75\\
91 & 35 & 0.0043 & 26.53 & 90 & 44 & 0.0061 & 28.28 & 89 & 52 & 0.0082 & 30.41\\
91 & 34 & 0.0041 & 26.34 & 90 & 43 & 0.0058 & 28.03 & 89 & 51 & 0.0079 & 30.07\\
91 & 33 & 0.0039 & 26.17 & 90 & 42 & 0.0056 & 27.78 & 89 & 50 & 0.0076 & 29.75\\
91 & 32 & 0.0038 & 26.00 & 90 & 41 & 0.0054 & 27.54 & 89 & 49 & 0.0073 & 29.44\\
90 & 90 & 0.0310 & 55.77 & 90 & 40 & 0.0052 & 27.32 & 89 & 48 & 0.0071 & 29.14\\
90 & 89 & 0.0300 & 54.66 & 90 & 39 & 0.0050 & 27.10 & 89 & 47 & 0.0068 & 28.85\\
90 & 88 & 0.0290 & 53.59 & 90 & 38 & 0.0048 & 26.88 & 89 & 46 & 0.0066 & 28.57\\
90 & 87 & 0.0281 & 52.54 & 90 & 37 & 0.0046 & 26.68 & 89 & 45 & 0.0063 & 28.30\\
90 & 86 & 0.0272 & 51.53 & 90 & 36 & 0.0044 & 26.48 & 89 & 44 & 0.0061 & 28.04\\
90 & 85 & 0.0263 & 50.55 & 90 & 35 & 0.0043 & 26.29 & 89 & 43 & 0.0058 & 27.78\\
90 & 84 & 0.0254 & 49.60 & 90 & 34 & 0.0041 & 26.10 & 89 & 42 & 0.0056 & 27.54\\
90 & 83 & 0.0246 & 48.68 & 90 & 33 & 0.0039 & 25.93 & 89 & 41 & 0.0054 & 27.30\\
90 & 82 & 0.0238 & 47.78 & 90 & 32 & 0.0038 & 25.75 & 89 & 40 & 0.0052 & 27.07\\
90 & 81 & 0.0230 & 46.91 & 89 & 89 & 0.0300 & 54.41 & 89 & 39 & 0.0050 & 26.85\\
90 & 80 & 0.0222 & 46.07 & 89 & 88 & 0.0290 & 53.33 & 89 & 38 & 0.0048 & 26.64\\
90 & 79 & 0.0215 & 45.25 & 89 & 87 & 0.0281 & 52.29 & 89 & 37 & 0.0046 & 26.44\\
90 & 78 & 0.0208 & 44.46 & 89 & 86 & 0.0272 & 51.28 & 89 & 36 & 0.0044 & 26.24\\
90 & 77 & 0.0201 & 43.70 & 89 & 85 & 0.0263 & 50.30 & 89 & 35 & 0.0043 & 26.05\\
90 & 76 & 0.0194 & 42.95 & 89 & 84 & 0.0254 & 49.35 & 89 & 34 & 0.0041 & 25.86\\
90 & 75 & 0.0187 & 42.23 & 89 & 83 & 0.0246 & 48.42 & 89 & 33 & 0.0039 & 25.68\\
90 & 74 & 0.0181 & 41.53 & 89 & 82 & 0.0238 & 47.53 & 89 & 32 & 0.0038 & 25.51\\
90 & 73 & 0.0175 & 40.85 & 89 & 81 & 0.0230 & 46.66 & 88 & 88 & 0.0290 & 53.08\\
90 & 72 & 0.0169 & 40.20 & 89 & 80 & 0.0222 & 45.82 & 88 & 87 & 0.0281 & 52.04\\
90 & 71 & 0.0163 & 39.56 & 89 & 79 & 0.0215 & 45.00 & 88 & 86 & 0.0272 & 51.03\\
90 & 70 & 0.0158 & 38.95 & 89 & 78 & 0.0208 & 44.21 & 88 & 85 & 0.0263 & 50.05\\
90 & 69 & 0.0152 & 38.35 & 89 & 77 & 0.0201 & 43.45 & 88 & 84 & 0.0254 & 49.10\\
90 & 68 & 0.0147 & 37.77 & 89 & 76 & 0.0194 & 42.70 & 88 & 83 & 0.0246 & 48.17\\
90 & 67 & 0.0142 & 37.21 & 89 & 75 & 0.0187 & 41.98 & 88 & 82 & 0.0238 & 47.28\\
90 & 66 & 0.0137 & 36.67 & 89 & 74 & 0.0181 & 41.28 & 88 & 81 & 0.0230 & 46.41\\
90 & 65 & 0.0132 & 36.14 & 89 & 73 & 0.0175 & 40.61 & 88 & 80 & 0.0222 & 45.57\\
90 & 64 & 0.0127 & 35.63 & 89 & 72 & 0.0169 & 39.95 & 88 & 79 & 0.0215 & 44.76\\
90 & 63 & 0.0123 & 35.14 & 89 & 71 & 0.0163 & 39.31 & 88 & 78 & 0.0208 & 43.96\\
90 & 62 & 0.0119 & 34.66 & 89 & 70 & 0.0158 & 38.70 & 88 & 77 & 0.0201 & 43.20\\
90 & 61 & 0.0114 & 34.20 & 89 & 69 & 0.0152 & 38.10 & 88 & 76 & 0.0194 & 42.45\\
90 & 60 & 0.0110 & 33.75 & 89 & 68 & 0.0147 & 37.52 & 88 & 75 & 0.0187 & 41.73\\
90 & 59 & 0.0106 & 33.32 & 89 & 67 & 0.0142 & 36.96 & 88 & 74 & 0.0181 & 41.04\\
90 & 58 & 0.0103 & 32.90 & 89 & 66 & 0.0137 & 36.42 & 88 & 73 & 0.0175 & 40.36\\
90 & 57 & 0.0099 & 32.49 & 89 & 65 & 0.0132 & 35.89 & 88 & 72 & 0.0169 & 39.70\\
90 & 56 & 0.0095 & 32.10 & 89 & 64 & 0.0127 & 35.38 & 88 & 71 & 0.0163 & 39.07\\
90 & 55 & 0.0092 & 31.72 & 89 & 63 & 0.0123 & 34.89 & 88 & 70 & 0.0158 & 38.45\\
90 & 54 & 0.0089 & 31.35 & 89 & 62 & 0.0119 & 34.41 & 88 & 69 & 0.0152 & 37.85\\
90 & 53 & 0.0085 & 30.99 & 89 & 61 & 0.0114 & 33.95 & 88 & 68 & 0.0147 & 37.28\\
90 & 52 & 0.0082 & 30.65 & 89 & 60 & 0.0110 & 33.50 & 88 & 67 & 0.0142 & 36.72\\
90 & 51 & 0.0079 & 30.32 & 89 & 59 & 0.0106 & 33.07 & 88 & 66 & 0.0137 & 36.17\\
90 & 50 & 0.0076 & 30.00 & 89 & 58 & 0.0103 & 32.65 & 88 & 65 & 0.0132 & 35.65\\
90 & 49 & 0.0073 & 29.68 & 89 & 57 & 0.0099 & 32.25 & 88 & 64 & 0.0127 & 35.14\\
90 & 48 & 0.0071 & 29.38 & 89 & 56 & 0.0095 & 31.85 & 88 & 63 & 0.0123 & 34.65\\
90 & 47 & 0.0068 & 29.09 & 89 & 55 & 0.0092 & 31.47 & 88 & 62 & 0.0119 & 34.17\\
\bottomrule
\end{tabular}
\newpage
\begin{tabular}{llll|llll|llll}
 \toprule 
\(T_{db}\) & \(T_{dp}\) & \(\omega\) & \(h\) & \(T_{db}\) & \(T_{dp}\) & \(\omega\) & \(h\) & \(T_{db}\) & \(T_{dp}\) & \(\omega\) & \(h\)  \\ \midrule 
88 & 61 & 0.0114 & 33.71 & 87 & 67 & 0.0142 & 36.47 & 86 & 72 & 0.0169 & 39.21\\
88 & 60 & 0.0110 & 33.26 & 87 & 66 & 0.0137 & 35.93 & 86 & 71 & 0.0163 & 38.57\\
88 & 59 & 0.0106 & 32.83 & 87 & 65 & 0.0132 & 35.40 & 86 & 70 & 0.0158 & 37.96\\
88 & 58 & 0.0103 & 32.41 & 87 & 64 & 0.0127 & 34.89 & 86 & 69 & 0.0152 & 37.36\\
88 & 57 & 0.0099 & 32.00 & 87 & 63 & 0.0123 & 34.40 & 86 & 68 & 0.0147 & 36.78\\
88 & 56 & 0.0095 & 31.61 & 87 & 62 & 0.0119 & 33.92 & 86 & 67 & 0.0142 & 36.22\\
88 & 55 & 0.0092 & 31.23 & 87 & 61 & 0.0114 & 33.46 & 86 & 66 & 0.0137 & 35.68\\
88 & 54 & 0.0089 & 30.86 & 87 & 60 & 0.0110 & 33.01 & 86 & 65 & 0.0132 & 35.16\\
88 & 53 & 0.0085 & 30.51 & 87 & 59 & 0.0106 & 32.58 & 86 & 64 & 0.0127 & 34.65\\
88 & 52 & 0.0082 & 30.16 & 87 & 58 & 0.0103 & 32.16 & 86 & 63 & 0.0123 & 34.16\\
88 & 51 & 0.0079 & 29.83 & 87 & 57 & 0.0099 & 31.76 & 86 & 62 & 0.0119 & 33.68\\
88 & 50 & 0.0076 & 29.51 & 87 & 56 & 0.0095 & 31.37 & 86 & 61 & 0.0114 & 33.22\\
88 & 49 & 0.0073 & 29.20 & 87 & 55 & 0.0092 & 30.99 & 86 & 60 & 0.0110 & 32.77\\
88 & 48 & 0.0071 & 28.90 & 87 & 54 & 0.0089 & 30.62 & 86 & 59 & 0.0106 & 32.34\\
88 & 47 & 0.0068 & 28.61 & 87 & 53 & 0.0085 & 30.26 & 86 & 58 & 0.0103 & 31.92\\
88 & 46 & 0.0066 & 28.33 & 87 & 52 & 0.0082 & 29.92 & 86 & 57 & 0.0099 & 31.51\\
88 & 45 & 0.0063 & 28.06 & 87 & 51 & 0.0079 & 29.59 & 86 & 56 & 0.0095 & 31.12\\
88 & 44 & 0.0061 & 27.79 & 87 & 50 & 0.0076 & 29.27 & 86 & 55 & 0.0092 & 30.74\\
88 & 43 & 0.0058 & 27.54 & 87 & 49 & 0.0073 & 28.96 & 86 & 54 & 0.0089 & 30.37\\
88 & 42 & 0.0056 & 27.30 & 87 & 48 & 0.0071 & 28.65 & 86 & 53 & 0.0085 & 30.02\\
88 & 41 & 0.0054 & 27.06 & 87 & 47 & 0.0068 & 28.36 & 86 & 52 & 0.0082 & 29.68\\
88 & 40 & 0.0052 & 26.83 & 87 & 46 & 0.0066 & 28.08 & 86 & 51 & 0.0079 & 29.34\\
88 & 39 & 0.0050 & 26.61 & 87 & 45 & 0.0063 & 27.81 & 86 & 50 & 0.0076 & 29.02\\
88 & 38 & 0.0048 & 26.40 & 87 & 44 & 0.0061 & 27.55 & 86 & 49 & 0.0073 & 28.71\\
88 & 37 & 0.0046 & 26.19 & 87 & 43 & 0.0058 & 27.30 & 86 & 48 & 0.0071 & 28.41\\
88 & 36 & 0.0044 & 26.00 & 87 & 42 & 0.0056 & 27.05 & 86 & 47 & 0.0068 & 28.12\\
88 & 35 & 0.0043 & 25.80 & 87 & 41 & 0.0054 & 26.82 & 86 & 46 & 0.0066 & 27.84\\
88 & 34 & 0.0041 & 25.62 & 87 & 40 & 0.0052 & 26.59 & 86 & 45 & 0.0063 & 27.57\\
88 & 33 & 0.0039 & 25.44 & 87 & 39 & 0.0050 & 26.37 & 86 & 44 & 0.0061 & 27.31\\
88 & 32 & 0.0038 & 25.27 & 87 & 38 & 0.0048 & 26.16 & 86 & 43 & 0.0058 & 27.05\\
87 & 87 & 0.0281 & 51.78 & 87 & 37 & 0.0046 & 25.95 & 86 & 42 & 0.0056 & 26.81\\
87 & 86 & 0.0272 & 50.77 & 87 & 36 & 0.0044 & 25.75 & 86 & 41 & 0.0054 & 26.57\\
87 & 85 & 0.0263 & 49.79 & 87 & 35 & 0.0043 & 25.56 & 86 & 40 & 0.0052 & 26.35\\
87 & 84 & 0.0254 & 48.84 & 87 & 34 & 0.0041 & 25.38 & 86 & 39 & 0.0050 & 26.13\\
87 & 83 & 0.0246 & 47.92 & 87 & 33 & 0.0039 & 25.20 & 86 & 38 & 0.0048 & 25.91\\
87 & 82 & 0.0238 & 47.03 & 87 & 32 & 0.0038 & 25.03 & 86 & 37 & 0.0046 & 25.71\\
87 & 81 & 0.0230 & 46.16 & 86 & 86 & 0.0272 & 50.52 & 86 & 36 & 0.0044 & 25.51\\
87 & 80 & 0.0222 & 45.32 & 86 & 85 & 0.0263 & 49.54 & 86 & 35 & 0.0043 & 25.32\\
87 & 79 & 0.0215 & 44.51 & 86 & 84 & 0.0254 & 48.59 & 86 & 34 & 0.0041 & 25.14\\
87 & 78 & 0.0208 & 43.72 & 86 & 83 & 0.0246 & 47.67 & 86 & 33 & 0.0039 & 24.96\\
87 & 77 & 0.0201 & 42.95 & 86 & 82 & 0.0238 & 46.78 & 86 & 32 & 0.0038 & 24.79\\
87 & 76 & 0.0194 & 42.21 & 86 & 81 & 0.0230 & 45.91 & 85 & 85 & 0.0263 & 49.29\\
87 & 75 & 0.0187 & 41.49 & 86 & 80 & 0.0222 & 45.07 & 85 & 84 & 0.0254 & 48.34\\
87 & 74 & 0.0181 & 40.79 & 86 & 79 & 0.0215 & 44.26 & 85 & 83 & 0.0246 & 47.42\\
87 & 73 & 0.0175 & 40.11 & 86 & 78 & 0.0208 & 43.47 & 85 & 82 & 0.0238 & 46.53\\
87 & 72 & 0.0169 & 39.46 & 86 & 77 & 0.0201 & 42.70 & 85 & 81 & 0.0230 & 45.66\\
87 & 71 & 0.0163 & 38.82 & 86 & 76 & 0.0194 & 41.96 & 85 & 80 & 0.0222 & 44.82\\
87 & 70 & 0.0158 & 38.20 & 86 & 75 & 0.0187 & 41.24 & 85 & 79 & 0.0215 & 44.01\\
87 & 69 & 0.0152 & 37.61 & 86 & 74 & 0.0181 & 40.54 & 85 & 78 & 0.0208 & 43.22\\
87 & 68 & 0.0147 & 37.03 & 86 & 73 & 0.0175 & 39.86 & 85 & 77 & 0.0201 & 42.45\\
\bottomrule
\end{tabular}
\newpage
\begin{tabular}{llll|llll|llll}
 \toprule 
\(T_{db}\) & \(T_{dp}\) & \(\omega\) & \(h\) & \(T_{db}\) & \(T_{dp}\) & \(\omega\) & \(h\) & \(T_{db}\) & \(T_{dp}\) & \(\omega\) & \(h\)  \\ \midrule 
85 & 76 & 0.0194 & 41.71 & 84 & 79 & 0.0215 & 43.76 & 83 & 81 & 0.0230 & 45.16\\
85 & 75 & 0.0187 & 40.99 & 84 & 78 & 0.0208 & 42.97 & 83 & 80 & 0.0222 & 44.32\\
85 & 74 & 0.0181 & 40.29 & 84 & 77 & 0.0201 & 42.20 & 83 & 79 & 0.0215 & 43.51\\
85 & 73 & 0.0175 & 39.62 & 84 & 76 & 0.0194 & 41.46 & 83 & 78 & 0.0208 & 42.72\\
85 & 72 & 0.0169 & 38.96 & 84 & 75 & 0.0187 & 40.74 & 83 & 77 & 0.0201 & 41.95\\
85 & 71 & 0.0163 & 38.33 & 84 & 74 & 0.0181 & 40.04 & 83 & 76 & 0.0194 & 41.21\\
85 & 70 & 0.0158 & 37.71 & 84 & 73 & 0.0175 & 39.37 & 83 & 75 & 0.0187 & 40.49\\
85 & 69 & 0.0152 & 37.11 & 84 & 72 & 0.0169 & 38.71 & 83 & 74 & 0.0181 & 39.80\\
85 & 68 & 0.0147 & 36.54 & 84 & 71 & 0.0163 & 38.08 & 83 & 73 & 0.0175 & 39.12\\
85 & 67 & 0.0142 & 35.98 & 84 & 70 & 0.0158 & 37.46 & 83 & 72 & 0.0169 & 38.47\\
85 & 66 & 0.0137 & 35.44 & 84 & 69 & 0.0152 & 36.87 & 83 & 71 & 0.0163 & 37.83\\
85 & 65 & 0.0132 & 34.91 & 84 & 68 & 0.0147 & 36.29 & 83 & 70 & 0.0158 & 37.22\\
85 & 64 & 0.0127 & 34.40 & 84 & 67 & 0.0142 & 35.73 & 83 & 69 & 0.0152 & 36.62\\
85 & 63 & 0.0123 & 33.91 & 84 & 66 & 0.0137 & 35.19 & 83 & 68 & 0.0147 & 36.04\\
85 & 62 & 0.0119 & 33.43 & 84 & 65 & 0.0132 & 34.66 & 83 & 67 & 0.0142 & 35.48\\
85 & 61 & 0.0114 & 32.97 & 84 & 64 & 0.0127 & 34.16 & 83 & 66 & 0.0137 & 34.94\\
85 & 60 & 0.0110 & 32.52 & 84 & 63 & 0.0123 & 33.66 & 83 & 65 & 0.0132 & 34.42\\
85 & 59 & 0.0106 & 32.09 & 84 & 62 & 0.0119 & 33.19 & 83 & 64 & 0.0127 & 33.91\\
85 & 58 & 0.0103 & 31.67 & 84 & 61 & 0.0114 & 32.73 & 83 & 63 & 0.0123 & 33.42\\
85 & 57 & 0.0099 & 31.27 & 84 & 60 & 0.0110 & 32.28 & 83 & 62 & 0.0119 & 32.94\\
85 & 56 & 0.0095 & 30.88 & 84 & 59 & 0.0106 & 31.85 & 83 & 61 & 0.0114 & 32.48\\
85 & 55 & 0.0092 & 30.50 & 84 & 58 & 0.0103 & 31.43 & 83 & 60 & 0.0110 & 32.03\\
85 & 54 & 0.0089 & 30.13 & 84 & 57 & 0.0099 & 31.02 & 83 & 59 & 0.0106 & 31.60\\
85 & 53 & 0.0085 & 29.78 & 84 & 56 & 0.0095 & 30.63 & 83 & 58 & 0.0103 & 31.18\\
85 & 52 & 0.0082 & 29.43 & 84 & 55 & 0.0092 & 30.25 & 83 & 57 & 0.0099 & 30.78\\
85 & 51 & 0.0079 & 29.10 & 84 & 54 & 0.0089 & 29.89 & 83 & 56 & 0.0095 & 30.39\\
85 & 50 & 0.0076 & 28.78 & 84 & 53 & 0.0085 & 29.53 & 83 & 55 & 0.0092 & 30.01\\
85 & 49 & 0.0073 & 28.47 & 84 & 52 & 0.0082 & 29.19 & 83 & 54 & 0.0089 & 29.64\\
85 & 48 & 0.0071 & 28.17 & 84 & 51 & 0.0079 & 28.86 & 83 & 53 & 0.0085 & 29.29\\
85 & 47 & 0.0068 & 27.88 & 84 & 50 & 0.0076 & 28.54 & 83 & 52 & 0.0082 & 28.94\\
85 & 46 & 0.0066 & 27.60 & 84 & 49 & 0.0073 & 28.23 & 83 & 51 & 0.0079 & 28.61\\
85 & 45 & 0.0063 & 27.33 & 84 & 48 & 0.0071 & 27.93 & 83 & 50 & 0.0076 & 28.29\\
85 & 44 & 0.0061 & 27.07 & 84 & 47 & 0.0068 & 27.64 & 83 & 49 & 0.0073 & 27.98\\
85 & 43 & 0.0058 & 26.81 & 84 & 46 & 0.0066 & 27.36 & 83 & 48 & 0.0071 & 27.68\\
85 & 42 & 0.0056 & 26.57 & 84 & 45 & 0.0063 & 27.08 & 83 & 47 & 0.0068 & 27.39\\
85 & 41 & 0.0054 & 26.33 & 84 & 44 & 0.0061 & 26.82 & 83 & 46 & 0.0066 & 27.11\\
85 & 40 & 0.0052 & 26.10 & 84 & 43 & 0.0058 & 26.57 & 83 & 45 & 0.0063 & 26.84\\
85 & 39 & 0.0050 & 25.88 & 84 & 42 & 0.0056 & 26.33 & 83 & 44 & 0.0061 & 26.58\\
85 & 38 & 0.0048 & 25.67 & 84 & 41 & 0.0054 & 26.09 & 83 & 43 & 0.0058 & 26.33\\
85 & 37 & 0.0046 & 25.47 & 84 & 40 & 0.0052 & 25.86 & 83 & 42 & 0.0056 & 26.08\\
85 & 36 & 0.0044 & 25.27 & 84 & 39 & 0.0050 & 25.64 & 83 & 41 & 0.0054 & 25.85\\
85 & 35 & 0.0043 & 25.08 & 84 & 38 & 0.0048 & 25.43 & 83 & 40 & 0.0052 & 25.62\\
85 & 34 & 0.0041 & 24.89 & 84 & 37 & 0.0046 & 25.22 & 83 & 39 & 0.0050 & 25.40\\
85 & 33 & 0.0039 & 24.72 & 84 & 36 & 0.0044 & 25.03 & 83 & 38 & 0.0048 & 25.19\\
85 & 32 & 0.0038 & 24.55 & 84 & 35 & 0.0043 & 24.84 & 83 & 37 & 0.0046 & 24.98\\
84 & 84 & 0.0254 & 48.09 & 84 & 34 & 0.0041 & 24.65 & 83 & 36 & 0.0044 & 24.79\\
84 & 83 & 0.0246 & 47.17 & 84 & 33 & 0.0039 & 24.47 & 83 & 35 & 0.0043 & 24.59\\
84 & 82 & 0.0238 & 46.28 & 84 & 32 & 0.0038 & 24.30 & 83 & 34 & 0.0041 & 24.41\\
84 & 81 & 0.0230 & 45.41 & 83 & 83 & 0.0246 & 46.92 & 83 & 33 & 0.0039 & 24.23\\
84 & 80 & 0.0222 & 44.57 & 83 & 82 & 0.0238 & 46.03 & 83 & 32 & 0.0038 & 24.06\\
\bottomrule
\end{tabular}
\newpage
\begin{tabular}{llll|llll|llll}
 \toprule 
\(T_{db}\) & \(T_{dp}\) & \(\omega\) & \(h\) & \(T_{db}\) & \(T_{dp}\) & \(\omega\) & \(h\) & \(T_{db}\) & \(T_{dp}\) & \(\omega\) & \(h\)  \\ \midrule 
82 & 82 & 0.0238 & 45.78 & 82 & 32 & 0.0038 & 23.82 & 81 & 32 & 0.0038 & 23.58\\
82 & 81 & 0.0230 & 44.91 & 81 & 81 & 0.0230 & 44.66 & 80 & 80 & 0.0222 & 43.57\\
82 & 80 & 0.0222 & 44.07 & 81 & 80 & 0.0222 & 43.82 & 80 & 79 & 0.0215 & 42.76\\
82 & 79 & 0.0215 & 43.26 & 81 & 79 & 0.0215 & 43.01 & 80 & 78 & 0.0208 & 41.97\\
82 & 78 & 0.0208 & 42.47 & 81 & 78 & 0.0208 & 42.22 & 80 & 77 & 0.0201 & 41.21\\
82 & 77 & 0.0201 & 41.70 & 81 & 77 & 0.0201 & 41.46 & 80 & 76 & 0.0194 & 40.47\\
82 & 76 & 0.0194 & 40.96 & 81 & 76 & 0.0194 & 40.71 & 80 & 75 & 0.0187 & 39.75\\
82 & 75 & 0.0187 & 40.24 & 81 & 75 & 0.0187 & 40.00 & 80 & 74 & 0.0181 & 39.05\\
82 & 74 & 0.0181 & 39.55 & 81 & 74 & 0.0181 & 39.30 & 80 & 73 & 0.0175 & 38.38\\
82 & 73 & 0.0175 & 38.87 & 81 & 73 & 0.0175 & 38.62 & 80 & 72 & 0.0169 & 37.72\\
82 & 72 & 0.0169 & 38.22 & 81 & 72 & 0.0169 & 37.97 & 80 & 71 & 0.0163 & 37.09\\
82 & 71 & 0.0163 & 37.58 & 81 & 71 & 0.0163 & 37.34 & 80 & 70 & 0.0158 & 36.48\\
82 & 70 & 0.0158 & 36.97 & 81 & 70 & 0.0158 & 36.72 & 80 & 69 & 0.0152 & 35.88\\
82 & 69 & 0.0152 & 36.37 & 81 & 69 & 0.0152 & 36.13 & 80 & 68 & 0.0147 & 35.30\\
82 & 68 & 0.0147 & 35.80 & 81 & 68 & 0.0147 & 35.55 & 80 & 67 & 0.0142 & 34.75\\
82 & 67 & 0.0142 & 35.24 & 81 & 67 & 0.0142 & 34.99 & 80 & 66 & 0.0137 & 34.20\\
82 & 66 & 0.0137 & 34.70 & 81 & 66 & 0.0137 & 34.45 & 80 & 65 & 0.0132 & 33.68\\
82 & 65 & 0.0132 & 34.17 & 81 & 65 & 0.0132 & 33.93 & 80 & 64 & 0.0127 & 33.17\\
82 & 64 & 0.0127 & 33.67 & 81 & 64 & 0.0127 & 33.42 & 80 & 63 & 0.0123 & 32.68\\
82 & 63 & 0.0123 & 33.17 & 81 & 63 & 0.0123 & 32.93 & 80 & 62 & 0.0119 & 32.21\\
82 & 62 & 0.0119 & 32.70 & 81 & 62 & 0.0119 & 32.45 & 80 & 61 & 0.0114 & 31.75\\
82 & 61 & 0.0114 & 32.24 & 81 & 61 & 0.0114 & 31.99 & 80 & 60 & 0.0110 & 31.30\\
82 & 60 & 0.0110 & 31.79 & 81 & 60 & 0.0110 & 31.55 & 80 & 59 & 0.0106 & 30.87\\
82 & 59 & 0.0106 & 31.36 & 81 & 59 & 0.0106 & 31.11 & 80 & 58 & 0.0103 & 30.45\\
82 & 58 & 0.0103 & 30.94 & 81 & 58 & 0.0103 & 30.70 & 80 & 57 & 0.0099 & 30.05\\
82 & 57 & 0.0099 & 30.54 & 81 & 57 & 0.0099 & 30.29 & 80 & 56 & 0.0095 & 29.66\\
82 & 56 & 0.0095 & 30.14 & 81 & 56 & 0.0095 & 29.90 & 80 & 55 & 0.0092 & 29.28\\
82 & 55 & 0.0092 & 29.76 & 81 & 55 & 0.0092 & 29.52 & 80 & 54 & 0.0089 & 28.91\\
82 & 54 & 0.0089 & 29.40 & 81 & 54 & 0.0089 & 29.15 & 80 & 53 & 0.0085 & 28.56\\
82 & 53 & 0.0085 & 29.04 & 81 & 53 & 0.0085 & 28.80 & 80 & 52 & 0.0082 & 28.21\\
82 & 52 & 0.0082 & 28.70 & 81 & 52 & 0.0082 & 28.46 & 80 & 51 & 0.0079 & 27.88\\
82 & 51 & 0.0079 & 28.37 & 81 & 51 & 0.0079 & 28.13 & 80 & 50 & 0.0076 & 27.56\\
82 & 50 & 0.0076 & 28.05 & 81 & 50 & 0.0076 & 27.81 & 80 & 49 & 0.0073 & 27.25\\
82 & 49 & 0.0073 & 27.74 & 81 & 49 & 0.0073 & 27.50 & 80 & 48 & 0.0071 & 26.95\\
82 & 48 & 0.0071 & 27.44 & 81 & 48 & 0.0071 & 27.20 & 80 & 47 & 0.0068 & 26.66\\
82 & 47 & 0.0068 & 27.15 & 81 & 47 & 0.0068 & 26.91 & 80 & 46 & 0.0066 & 26.38\\
82 & 46 & 0.0066 & 26.87 & 81 & 46 & 0.0066 & 26.63 & 80 & 45 & 0.0063 & 26.11\\
82 & 45 & 0.0063 & 26.60 & 81 & 45 & 0.0063 & 26.36 & 80 & 44 & 0.0061 & 25.85\\
82 & 44 & 0.0061 & 26.34 & 81 & 44 & 0.0061 & 26.09 & 80 & 43 & 0.0058 & 25.60\\
82 & 43 & 0.0058 & 26.08 & 81 & 43 & 0.0058 & 25.84 & 80 & 42 & 0.0056 & 25.36\\
82 & 42 & 0.0056 & 25.84 & 81 & 42 & 0.0056 & 25.60 & 80 & 41 & 0.0054 & 25.12\\
82 & 41 & 0.0054 & 25.60 & 81 & 41 & 0.0054 & 25.36 & 80 & 40 & 0.0052 & 24.89\\
82 & 40 & 0.0052 & 25.38 & 81 & 40 & 0.0052 & 25.14 & 80 & 39 & 0.0050 & 24.67\\
82 & 39 & 0.0050 & 25.16 & 81 & 39 & 0.0050 & 24.92 & 80 & 38 & 0.0048 & 24.46\\
82 & 38 & 0.0048 & 24.95 & 81 & 38 & 0.0048 & 24.70 & 80 & 37 & 0.0046 & 24.26\\
82 & 37 & 0.0046 & 24.74 & 81 & 37 & 0.0046 & 24.50 & 80 & 36 & 0.0044 & 24.06\\
82 & 36 & 0.0044 & 24.54 & 81 & 36 & 0.0044 & 24.30 & 80 & 35 & 0.0043 & 23.87\\
82 & 35 & 0.0043 & 24.35 & 81 & 35 & 0.0043 & 24.11 & 80 & 34 & 0.0041 & 23.68\\
82 & 34 & 0.0041 & 24.17 & 81 & 34 & 0.0041 & 23.93 & 80 & 33 & 0.0039 & 23.51\\
82 & 33 & 0.0039 & 23.99 & 81 & 33 & 0.0039 & 23.75 & 80 & 32 & 0.0038 & 23.34\\
\bottomrule
\end{tabular}
\newpage
\begin{tabular}{llll|llll|llll}
 \toprule 
\(T_{db}\) & \(T_{dp}\) & \(\omega\) & \(h\) & \(T_{db}\) & \(T_{dp}\) & \(\omega\) & \(h\) & \(T_{db}\) & \(T_{dp}\) & \(\omega\) & \(h\)  \\ \midrule 
79 & 79 & 0.0215 & 42.51 & 78 & 76 & 0.0194 & 39.97 & 77 & 72 & 0.0169 & 36.98\\
79 & 78 & 0.0208 & 41.72 & 78 & 75 & 0.0187 & 39.25 & 77 & 71 & 0.0163 & 36.35\\
79 & 77 & 0.0201 & 40.96 & 78 & 74 & 0.0181 & 38.56 & 77 & 70 & 0.0158 & 35.73\\
79 & 76 & 0.0194 & 40.22 & 78 & 73 & 0.0175 & 37.88 & 77 & 69 & 0.0152 & 35.14\\
79 & 75 & 0.0187 & 39.50 & 78 & 72 & 0.0169 & 37.23 & 77 & 68 & 0.0147 & 34.56\\
79 & 74 & 0.0181 & 38.80 & 78 & 71 & 0.0163 & 36.59 & 77 & 67 & 0.0142 & 34.01\\
79 & 73 & 0.0175 & 38.13 & 78 & 70 & 0.0158 & 35.98 & 77 & 66 & 0.0137 & 33.47\\
79 & 72 & 0.0169 & 37.48 & 78 & 69 & 0.0152 & 35.39 & 77 & 65 & 0.0132 & 32.94\\
79 & 71 & 0.0163 & 36.84 & 78 & 68 & 0.0147 & 34.81 & 77 & 64 & 0.0127 & 32.44\\
79 & 70 & 0.0158 & 36.23 & 78 & 67 & 0.0142 & 34.25 & 77 & 63 & 0.0123 & 31.95\\
79 & 69 & 0.0152 & 35.63 & 78 & 66 & 0.0137 & 33.71 & 77 & 62 & 0.0119 & 31.47\\
79 & 68 & 0.0147 & 35.06 & 78 & 65 & 0.0132 & 33.19 & 77 & 61 & 0.0114 & 31.01\\
79 & 67 & 0.0142 & 34.50 & 78 & 64 & 0.0127 & 32.68 & 77 & 60 & 0.0110 & 30.57\\
79 & 66 & 0.0137 & 33.96 & 78 & 63 & 0.0123 & 32.19 & 77 & 59 & 0.0106 & 30.13\\
79 & 65 & 0.0132 & 33.44 & 78 & 62 & 0.0119 & 31.72 & 77 & 58 & 0.0103 & 29.72\\
79 & 64 & 0.0127 & 32.93 & 78 & 61 & 0.0114 & 31.26 & 77 & 57 & 0.0099 & 29.31\\
79 & 63 & 0.0123 & 32.44 & 78 & 60 & 0.0110 & 30.81 & 77 & 56 & 0.0095 & 28.92\\
79 & 62 & 0.0119 & 31.96 & 78 & 59 & 0.0106 & 30.38 & 77 & 55 & 0.0092 & 28.54\\
79 & 61 & 0.0114 & 31.50 & 78 & 58 & 0.0103 & 29.96 & 77 & 54 & 0.0089 & 28.18\\
79 & 60 & 0.0110 & 31.06 & 78 & 57 & 0.0099 & 29.56 & 77 & 53 & 0.0085 & 27.82\\
79 & 59 & 0.0106 & 30.62 & 78 & 56 & 0.0095 & 29.17 & 77 & 52 & 0.0082 & 27.48\\
79 & 58 & 0.0103 & 30.21 & 78 & 55 & 0.0092 & 28.79 & 77 & 51 & 0.0079 & 27.15\\
79 & 57 & 0.0099 & 29.80 & 78 & 54 & 0.0089 & 28.42 & 77 & 50 & 0.0076 & 26.83\\
79 & 56 & 0.0095 & 29.41 & 78 & 53 & 0.0085 & 28.07 & 77 & 49 & 0.0073 & 26.52\\
79 & 55 & 0.0092 & 29.03 & 78 & 52 & 0.0082 & 27.73 & 77 & 48 & 0.0071 & 26.22\\
79 & 54 & 0.0089 & 28.67 & 78 & 51 & 0.0079 & 27.40 & 77 & 47 & 0.0068 & 25.93\\
79 & 53 & 0.0085 & 28.31 & 78 & 50 & 0.0076 & 27.08 & 77 & 46 & 0.0066 & 25.65\\
79 & 52 & 0.0082 & 27.97 & 78 & 49 & 0.0073 & 26.77 & 77 & 45 & 0.0063 & 25.38\\
79 & 51 & 0.0079 & 27.64 & 78 & 48 & 0.0071 & 26.47 & 77 & 44 & 0.0061 & 25.12\\
79 & 50 & 0.0076 & 27.32 & 78 & 47 & 0.0068 & 26.18 & 77 & 43 & 0.0058 & 24.87\\
79 & 49 & 0.0073 & 27.01 & 78 & 46 & 0.0066 & 25.90 & 77 & 42 & 0.0056 & 24.63\\
79 & 48 & 0.0071 & 26.71 & 78 & 45 & 0.0063 & 25.63 & 77 & 41 & 0.0054 & 24.39\\
79 & 47 & 0.0068 & 26.42 & 78 & 44 & 0.0061 & 25.37 & 77 & 40 & 0.0052 & 24.17\\
79 & 46 & 0.0066 & 26.14 & 78 & 43 & 0.0058 & 25.11 & 77 & 39 & 0.0050 & 23.95\\
79 & 45 & 0.0063 & 25.87 & 78 & 42 & 0.0056 & 24.87 & 77 & 38 & 0.0048 & 23.73\\
79 & 44 & 0.0061 & 25.61 & 78 & 41 & 0.0054 & 24.64 & 77 & 37 & 0.0046 & 23.53\\
79 & 43 & 0.0058 & 25.36 & 78 & 40 & 0.0052 & 24.41 & 77 & 36 & 0.0044 & 23.33\\
79 & 42 & 0.0056 & 25.11 & 78 & 39 & 0.0050 & 24.19 & 77 & 35 & 0.0043 & 23.14\\
79 & 41 & 0.0054 & 24.88 & 78 & 38 & 0.0048 & 23.98 & 77 & 34 & 0.0041 & 22.96\\
79 & 40 & 0.0052 & 24.65 & 78 & 37 & 0.0046 & 23.77 & 77 & 33 & 0.0039 & 22.78\\
79 & 39 & 0.0050 & 24.43 & 78 & 36 & 0.0044 & 23.58 & 77 & 32 & 0.0038 & 22.61\\
79 & 38 & 0.0048 & 24.22 & 78 & 35 & 0.0043 & 23.39 & 76 & 76 & 0.0194 & 39.47\\
79 & 37 & 0.0046 & 24.01 & 78 & 34 & 0.0041 & 23.20 & 76 & 75 & 0.0187 & 38.75\\
79 & 36 & 0.0044 & 23.82 & 78 & 33 & 0.0039 & 23.02 & 76 & 74 & 0.0181 & 38.06\\
79 & 35 & 0.0043 & 23.63 & 78 & 32 & 0.0038 & 22.85 & 76 & 73 & 0.0175 & 37.39\\
79 & 34 & 0.0041 & 23.44 & 77 & 77 & 0.0201 & 40.46 & 76 & 72 & 0.0169 & 36.73\\
79 & 33 & 0.0039 & 23.27 & 77 & 76 & 0.0194 & 39.72 & 76 & 71 & 0.0163 & 36.10\\
79 & 32 & 0.0038 & 23.09 & 77 & 75 & 0.0187 & 39.00 & 76 & 70 & 0.0158 & 35.49\\
78 & 78 & 0.0208 & 41.47 & 77 & 74 & 0.0181 & 38.31 & 76 & 69 & 0.0152 & 34.89\\
78 & 77 & 0.0201 & 40.71 & 77 & 73 & 0.0175 & 37.63 & 76 & 68 & 0.0147 & 34.32\\
\bottomrule
\end{tabular}
\newpage
\begin{tabular}{llll|llll|llll}
 \toprule 
\(T_{db}\) & \(T_{dp}\) & \(\omega\) & \(h\) & \(T_{db}\) & \(T_{dp}\) & \(\omega\) & \(h\) & \(T_{db}\) & \(T_{dp}\) & \(\omega\) & \(h\)  \\ \midrule 
76 & 67 & 0.0142 & 33.76 & 75 & 61 & 0.0114 & 30.52 & 74 & 54 & 0.0089 & 27.45\\
76 & 66 & 0.0137 & 33.22 & 75 & 60 & 0.0110 & 30.08 & 74 & 53 & 0.0085 & 27.09\\
76 & 65 & 0.0132 & 32.70 & 75 & 59 & 0.0106 & 29.64 & 74 & 52 & 0.0082 & 26.75\\
76 & 64 & 0.0127 & 32.19 & 75 & 58 & 0.0103 & 29.23 & 74 & 51 & 0.0079 & 26.42\\
76 & 63 & 0.0123 & 31.70 & 75 & 57 & 0.0099 & 28.82 & 74 & 50 & 0.0076 & 26.10\\
76 & 62 & 0.0119 & 31.23 & 75 & 56 & 0.0095 & 28.43 & 74 & 49 & 0.0073 & 25.79\\
76 & 61 & 0.0114 & 30.77 & 75 & 55 & 0.0092 & 28.06 & 74 & 48 & 0.0071 & 25.49\\
76 & 60 & 0.0110 & 30.32 & 75 & 54 & 0.0089 & 27.69 & 74 & 47 & 0.0068 & 25.20\\
76 & 59 & 0.0106 & 29.89 & 75 & 53 & 0.0085 & 27.34 & 74 & 46 & 0.0066 & 24.93\\
76 & 58 & 0.0103 & 29.47 & 75 & 52 & 0.0082 & 27.00 & 74 & 45 & 0.0063 & 24.66\\
76 & 57 & 0.0099 & 29.07 & 75 & 51 & 0.0079 & 26.66 & 74 & 44 & 0.0061 & 24.40\\
76 & 56 & 0.0095 & 28.68 & 75 & 50 & 0.0076 & 26.34 & 74 & 43 & 0.0058 & 24.14\\
76 & 55 & 0.0092 & 28.30 & 75 & 49 & 0.0073 & 26.04 & 74 & 42 & 0.0056 & 23.90\\
76 & 54 & 0.0089 & 27.93 & 75 & 48 & 0.0071 & 25.74 & 74 & 41 & 0.0054 & 23.67\\
76 & 53 & 0.0085 & 27.58 & 75 & 47 & 0.0068 & 25.45 & 74 & 40 & 0.0052 & 23.44\\
76 & 52 & 0.0082 & 27.24 & 75 & 46 & 0.0066 & 25.17 & 74 & 39 & 0.0050 & 23.22\\
76 & 51 & 0.0079 & 26.91 & 75 & 45 & 0.0063 & 24.90 & 74 & 38 & 0.0048 & 23.01\\
76 & 50 & 0.0076 & 26.59 & 75 & 44 & 0.0061 & 24.64 & 74 & 37 & 0.0046 & 22.80\\
76 & 49 & 0.0073 & 26.28 & 75 & 43 & 0.0058 & 24.39 & 74 & 36 & 0.0044 & 22.61\\
76 & 48 & 0.0071 & 25.98 & 75 & 42 & 0.0056 & 24.14 & 74 & 35 & 0.0043 & 22.42\\
76 & 47 & 0.0068 & 25.69 & 75 & 41 & 0.0054 & 23.91 & 74 & 34 & 0.0041 & 22.23\\
76 & 46 & 0.0066 & 25.41 & 75 & 40 & 0.0052 & 23.68 & 74 & 33 & 0.0039 & 22.06\\
76 & 45 & 0.0063 & 25.14 & 75 & 39 & 0.0050 & 23.46 & 74 & 32 & 0.0038 & 21.89\\
76 & 44 & 0.0061 & 24.88 & 75 & 38 & 0.0048 & 23.25 & 73 & 73 & 0.0175 & 36.64\\
76 & 43 & 0.0058 & 24.63 & 75 & 37 & 0.0046 & 23.05 & 73 & 72 & 0.0169 & 35.99\\
76 & 42 & 0.0056 & 24.39 & 75 & 36 & 0.0044 & 22.85 & 73 & 71 & 0.0163 & 35.36\\
76 & 41 & 0.0054 & 24.15 & 75 & 35 & 0.0043 & 22.66 & 73 & 70 & 0.0158 & 34.75\\
76 & 40 & 0.0052 & 23.92 & 75 & 34 & 0.0041 & 22.48 & 73 & 69 & 0.0152 & 34.15\\
76 & 39 & 0.0050 & 23.70 & 75 & 33 & 0.0039 & 22.30 & 73 & 68 & 0.0147 & 33.58\\
76 & 38 & 0.0048 & 23.49 & 75 & 32 & 0.0038 & 22.13 & 73 & 67 & 0.0142 & 33.02\\
76 & 37 & 0.0046 & 23.29 & 74 & 74 & 0.0181 & 37.56 & 73 & 66 & 0.0137 & 32.48\\
76 & 36 & 0.0044 & 23.09 & 74 & 73 & 0.0175 & 36.89 & 73 & 65 & 0.0132 & 31.96\\
76 & 35 & 0.0043 & 22.90 & 74 & 72 & 0.0169 & 36.24 & 73 & 64 & 0.0127 & 31.45\\
76 & 34 & 0.0041 & 22.72 & 74 & 71 & 0.0163 & 35.61 & 73 & 63 & 0.0123 & 30.96\\
76 & 33 & 0.0039 & 22.54 & 74 & 70 & 0.0158 & 34.99 & 73 & 62 & 0.0119 & 30.49\\
76 & 32 & 0.0038 & 22.37 & 74 & 69 & 0.0152 & 34.40 & 73 & 61 & 0.0114 & 30.03\\
75 & 75 & 0.0187 & 38.51 & 74 & 68 & 0.0147 & 33.82 & 73 & 60 & 0.0110 & 29.59\\
75 & 74 & 0.0181 & 37.81 & 74 & 67 & 0.0142 & 33.27 & 73 & 59 & 0.0106 & 29.16\\
75 & 73 & 0.0175 & 37.14 & 74 & 66 & 0.0137 & 32.73 & 73 & 58 & 0.0103 & 28.74\\
75 & 72 & 0.0169 & 36.49 & 74 & 65 & 0.0132 & 32.21 & 73 & 57 & 0.0099 & 28.34\\
75 & 71 & 0.0163 & 35.85 & 74 & 64 & 0.0127 & 31.70 & 73 & 56 & 0.0095 & 27.95\\
75 & 70 & 0.0158 & 35.24 & 74 & 63 & 0.0123 & 31.21 & 73 & 55 & 0.0092 & 27.57\\
75 & 69 & 0.0152 & 34.65 & 74 & 62 & 0.0119 & 30.74 & 73 & 54 & 0.0089 & 27.20\\
75 & 68 & 0.0147 & 34.07 & 74 & 61 & 0.0114 & 30.28 & 73 & 53 & 0.0085 & 26.85\\
75 & 67 & 0.0142 & 33.51 & 74 & 60 & 0.0110 & 29.83 & 73 & 52 & 0.0082 & 26.51\\
75 & 66 & 0.0137 & 32.97 & 74 & 59 & 0.0106 & 29.40 & 73 & 51 & 0.0079 & 26.18\\
75 & 65 & 0.0132 & 32.45 & 74 & 58 & 0.0103 & 28.98 & 73 & 50 & 0.0076 & 25.86\\
75 & 64 & 0.0127 & 31.95 & 74 & 57 & 0.0099 & 28.58 & 73 & 49 & 0.0073 & 25.55\\
75 & 63 & 0.0123 & 31.46 & 74 & 56 & 0.0095 & 28.19 & 73 & 48 & 0.0071 & 25.25\\
75 & 62 & 0.0119 & 30.98 & 74 & 55 & 0.0092 & 27.81 & 73 & 47 & 0.0068 & 24.96\\
\bottomrule
\end{tabular}
\newpage
\begin{tabular}{llll|llll|llll}
 \toprule 
\(T_{db}\) & \(T_{dp}\) & \(\omega\) & \(h\) & \(T_{db}\) & \(T_{dp}\) & \(\omega\) & \(h\) & \(T_{db}\) & \(T_{dp}\) & \(\omega\) & \(h\)  \\ \midrule 
73 & 46 & 0.0066 & 24.68 & 72 & 37 & 0.0046 & 22.32 & 70 & 66 & 0.0137 & 31.74\\
73 & 45 & 0.0063 & 24.41 & 72 & 36 & 0.0044 & 22.12 & 70 & 65 & 0.0132 & 31.22\\
73 & 44 & 0.0061 & 24.15 & 72 & 35 & 0.0043 & 21.93 & 70 & 64 & 0.0127 & 30.72\\
73 & 43 & 0.0058 & 23.90 & 72 & 34 & 0.0041 & 21.75 & 70 & 63 & 0.0123 & 30.23\\
73 & 42 & 0.0056 & 23.66 & 72 & 33 & 0.0039 & 21.57 & 70 & 62 & 0.0119 & 29.75\\
73 & 41 & 0.0054 & 23.42 & 72 & 32 & 0.0038 & 21.40 & 70 & 61 & 0.0114 & 29.30\\
73 & 40 & 0.0052 & 23.20 & 71 & 71 & 0.0163 & 34.86 & 70 & 60 & 0.0110 & 28.85\\
73 & 39 & 0.0050 & 22.98 & 71 & 70 & 0.0158 & 34.25 & 70 & 59 & 0.0106 & 28.42\\
73 & 38 & 0.0048 & 22.77 & 71 & 69 & 0.0152 & 33.66 & 70 & 58 & 0.0103 & 28.01\\
73 & 37 & 0.0046 & 22.56 & 71 & 68 & 0.0147 & 33.09 & 70 & 57 & 0.0099 & 27.60\\
73 & 36 & 0.0044 & 22.37 & 71 & 67 & 0.0142 & 32.53 & 70 & 56 & 0.0095 & 27.21\\
73 & 35 & 0.0043 & 22.18 & 71 & 66 & 0.0137 & 31.99 & 70 & 55 & 0.0092 & 26.84\\
73 & 34 & 0.0041 & 21.99 & 71 & 65 & 0.0132 & 31.47 & 70 & 54 & 0.0089 & 26.47\\
73 & 33 & 0.0039 & 21.82 & 71 & 64 & 0.0127 & 30.96 & 70 & 53 & 0.0085 & 26.12\\
73 & 32 & 0.0038 & 21.64 & 71 & 63 & 0.0123 & 30.47 & 70 & 52 & 0.0082 & 25.78\\
72 & 72 & 0.0169 & 35.74 & 71 & 62 & 0.0119 & 30.00 & 70 & 51 & 0.0079 & 25.45\\
72 & 71 & 0.0163 & 35.11 & 71 & 61 & 0.0114 & 29.54 & 70 & 50 & 0.0076 & 25.13\\
72 & 70 & 0.0158 & 34.50 & 71 & 60 & 0.0110 & 29.10 & 70 & 49 & 0.0073 & 24.82\\
72 & 69 & 0.0152 & 33.91 & 71 & 59 & 0.0106 & 28.67 & 70 & 48 & 0.0071 & 24.52\\
72 & 68 & 0.0147 & 33.33 & 71 & 58 & 0.0103 & 28.25 & 70 & 47 & 0.0068 & 24.23\\
72 & 67 & 0.0142 & 32.78 & 71 & 57 & 0.0099 & 27.85 & 70 & 46 & 0.0066 & 23.95\\
72 & 66 & 0.0137 & 32.24 & 71 & 56 & 0.0095 & 27.46 & 70 & 45 & 0.0063 & 23.68\\
72 & 65 & 0.0132 & 31.71 & 71 & 55 & 0.0092 & 27.08 & 70 & 44 & 0.0061 & 23.42\\
72 & 64 & 0.0127 & 31.21 & 71 & 54 & 0.0089 & 26.72 & 70 & 43 & 0.0058 & 23.17\\
72 & 63 & 0.0123 & 30.72 & 71 & 53 & 0.0085 & 26.36 & 70 & 42 & 0.0056 & 22.93\\
72 & 62 & 0.0119 & 30.24 & 71 & 52 & 0.0082 & 26.02 & 70 & 41 & 0.0054 & 22.70\\
72 & 61 & 0.0114 & 29.79 & 71 & 51 & 0.0079 & 25.69 & 70 & 40 & 0.0052 & 22.47\\
72 & 60 & 0.0110 & 29.34 & 71 & 50 & 0.0076 & 25.37 & 70 & 39 & 0.0050 & 22.25\\
72 & 59 & 0.0106 & 28.91 & 71 & 49 & 0.0073 & 25.06 & 70 & 38 & 0.0048 & 22.04\\
72 & 58 & 0.0103 & 28.49 & 71 & 48 & 0.0071 & 24.76 & 70 & 37 & 0.0046 & 21.84\\
72 & 57 & 0.0099 & 28.09 & 71 & 47 & 0.0068 & 24.48 & 70 & 36 & 0.0044 & 21.64\\
72 & 56 & 0.0095 & 27.70 & 71 & 46 & 0.0066 & 24.20 & 70 & 35 & 0.0043 & 21.45\\
72 & 55 & 0.0092 & 27.32 & 71 & 45 & 0.0063 & 23.93 & 70 & 34 & 0.0041 & 21.27\\
72 & 54 & 0.0089 & 26.96 & 71 & 44 & 0.0061 & 23.67 & 70 & 33 & 0.0039 & 21.09\\
72 & 53 & 0.0085 & 26.61 & 71 & 43 & 0.0058 & 23.42 & 70 & 32 & 0.0038 & 20.92\\
72 & 52 & 0.0082 & 26.26 & 71 & 42 & 0.0056 & 23.17 & 69 & 69 & 0.0152 & 33.17\\
72 & 51 & 0.0079 & 25.93 & 71 & 41 & 0.0054 & 22.94 & 69 & 68 & 0.0147 & 32.59\\
72 & 50 & 0.0076 & 25.61 & 71 & 40 & 0.0052 & 22.71 & 69 & 67 & 0.0142 & 32.04\\
72 & 49 & 0.0073 & 25.31 & 71 & 39 & 0.0050 & 22.49 & 69 & 66 & 0.0137 & 31.50\\
72 & 48 & 0.0071 & 25.01 & 71 & 38 & 0.0048 & 22.28 & 69 & 65 & 0.0132 & 30.98\\
72 & 47 & 0.0068 & 24.72 & 71 & 37 & 0.0046 & 22.08 & 69 & 64 & 0.0127 & 30.47\\
72 & 46 & 0.0066 & 24.44 & 71 & 36 & 0.0044 & 21.88 & 69 & 63 & 0.0123 & 29.98\\
72 & 45 & 0.0063 & 24.17 & 71 & 35 & 0.0043 & 21.69 & 69 & 62 & 0.0119 & 29.51\\
72 & 44 & 0.0061 & 23.91 & 71 & 34 & 0.0041 & 21.51 & 69 & 61 & 0.0114 & 29.05\\
72 & 43 & 0.0058 & 23.66 & 71 & 33 & 0.0039 & 21.33 & 69 & 60 & 0.0110 & 28.61\\
72 & 42 & 0.0056 & 23.42 & 71 & 32 & 0.0038 & 21.16 & 69 & 59 & 0.0106 & 28.18\\
72 & 41 & 0.0054 & 23.18 & 70 & 70 & 0.0158 & 34.01 & 69 & 58 & 0.0103 & 27.76\\
72 & 40 & 0.0052 & 22.95 & 70 & 69 & 0.0152 & 33.41 & 69 & 57 & 0.0099 & 27.36\\
72 & 39 & 0.0050 & 22.74 & 70 & 68 & 0.0147 & 32.84 & 69 & 56 & 0.0095 & 26.97\\
72 & 38 & 0.0048 & 22.52 & 70 & 67 & 0.0142 & 32.28 & 69 & 55 & 0.0092 & 26.59\\
\bottomrule
\end{tabular}
\newpage
\begin{tabular}{llll|llll|llll}
 \toprule 
\(T_{db}\) & \(T_{dp}\) & \(\omega\) & \(h\) & \(T_{db}\) & \(T_{dp}\) & \(\omega\) & \(h\) & \(T_{db}\) & \(T_{dp}\) & \(\omega\) & \(h\)  \\ \midrule 
69 & 54 & 0.0089 & 26.23 & 68 & 41 & 0.0054 & 22.21 & 66 & 62 & 0.0119 & 28.77\\
69 & 53 & 0.0085 & 25.87 & 68 & 40 & 0.0052 & 21.99 & 66 & 61 & 0.0114 & 28.31\\
69 & 52 & 0.0082 & 25.53 & 68 & 39 & 0.0050 & 21.77 & 66 & 60 & 0.0110 & 27.87\\
69 & 51 & 0.0079 & 25.20 & 68 & 38 & 0.0048 & 21.56 & 66 & 59 & 0.0106 & 27.44\\
69 & 50 & 0.0076 & 24.88 & 68 & 37 & 0.0046 & 21.35 & 66 & 58 & 0.0103 & 27.03\\
69 & 49 & 0.0073 & 24.58 & 68 & 36 & 0.0044 & 21.16 & 66 & 57 & 0.0099 & 26.62\\
69 & 48 & 0.0071 & 24.28 & 68 & 35 & 0.0043 & 20.97 & 66 & 56 & 0.0095 & 26.24\\
69 & 47 & 0.0068 & 23.99 & 68 & 34 & 0.0041 & 20.78 & 66 & 55 & 0.0092 & 25.86\\
69 & 46 & 0.0066 & 23.71 & 68 & 33 & 0.0039 & 20.61 & 66 & 54 & 0.0089 & 25.50\\
69 & 45 & 0.0063 & 23.44 & 68 & 32 & 0.0038 & 20.44 & 66 & 53 & 0.0085 & 25.14\\
69 & 44 & 0.0061 & 23.18 & 67 & 67 & 0.0142 & 31.54 & 66 & 52 & 0.0082 & 24.80\\
69 & 43 & 0.0058 & 22.93 & 67 & 66 & 0.0137 & 31.01 & 66 & 51 & 0.0079 & 24.47\\
69 & 42 & 0.0056 & 22.69 & 67 & 65 & 0.0132 & 30.48 & 66 & 50 & 0.0076 & 24.15\\
69 & 41 & 0.0054 & 22.45 & 67 & 64 & 0.0127 & 29.98 & 66 & 49 & 0.0073 & 23.85\\
69 & 40 & 0.0052 & 22.23 & 67 & 63 & 0.0123 & 29.49 & 66 & 48 & 0.0071 & 23.55\\
69 & 39 & 0.0050 & 22.01 & 67 & 62 & 0.0119 & 29.02 & 66 & 47 & 0.0068 & 23.26\\
69 & 38 & 0.0048 & 21.80 & 67 & 61 & 0.0114 & 28.56 & 66 & 46 & 0.0066 & 22.98\\
69 & 37 & 0.0046 & 21.59 & 67 & 60 & 0.0110 & 28.12 & 66 & 45 & 0.0063 & 22.71\\
69 & 36 & 0.0044 & 21.40 & 67 & 59 & 0.0106 & 27.69 & 66 & 44 & 0.0061 & 22.45\\
69 & 35 & 0.0043 & 21.21 & 67 & 58 & 0.0103 & 27.27 & 66 & 43 & 0.0058 & 22.20\\
69 & 34 & 0.0041 & 21.02 & 67 & 57 & 0.0099 & 26.87 & 66 & 42 & 0.0056 & 21.96\\
69 & 33 & 0.0039 & 20.85 & 67 & 56 & 0.0095 & 26.48 & 66 & 41 & 0.0054 & 21.73\\
69 & 32 & 0.0038 & 20.68 & 67 & 55 & 0.0092 & 26.10 & 66 & 40 & 0.0052 & 21.50\\
68 & 68 & 0.0147 & 32.35 & 67 & 54 & 0.0089 & 25.74 & 66 & 39 & 0.0050 & 21.28\\
68 & 67 & 0.0142 & 31.79 & 67 & 53 & 0.0085 & 25.39 & 66 & 38 & 0.0048 & 21.07\\
68 & 66 & 0.0137 & 31.25 & 67 & 52 & 0.0082 & 25.05 & 66 & 37 & 0.0046 & 20.87\\
68 & 65 & 0.0132 & 30.73 & 67 & 51 & 0.0079 & 24.72 & 66 & 36 & 0.0044 & 20.67\\
68 & 64 & 0.0127 & 30.23 & 67 & 50 & 0.0076 & 24.40 & 66 & 35 & 0.0043 & 20.48\\
68 & 63 & 0.0123 & 29.74 & 67 & 49 & 0.0073 & 24.09 & 66 & 34 & 0.0041 & 20.30\\
68 & 62 & 0.0119 & 29.26 & 67 & 48 & 0.0071 & 23.79 & 66 & 33 & 0.0039 & 20.12\\
68 & 61 & 0.0114 & 28.81 & 67 & 47 & 0.0068 & 23.50 & 66 & 32 & 0.0038 & 19.95\\
68 & 60 & 0.0110 & 28.36 & 67 & 46 & 0.0066 & 23.23 & 65 & 65 & 0.0132 & 29.99\\
68 & 59 & 0.0106 & 27.93 & 67 & 45 & 0.0063 & 22.96 & 65 & 64 & 0.0127 & 29.49\\
68 & 58 & 0.0103 & 27.52 & 67 & 44 & 0.0061 & 22.70 & 65 & 63 & 0.0123 & 29.00\\
68 & 57 & 0.0099 & 27.11 & 67 & 43 & 0.0058 & 22.45 & 65 & 62 & 0.0119 & 28.53\\
68 & 56 & 0.0095 & 26.72 & 67 & 42 & 0.0056 & 22.20 & 65 & 61 & 0.0114 & 28.07\\
68 & 55 & 0.0092 & 26.35 & 67 & 41 & 0.0054 & 21.97 & 65 & 60 & 0.0110 & 27.63\\
68 & 54 & 0.0089 & 25.98 & 67 & 40 & 0.0052 & 21.74 & 65 & 59 & 0.0106 & 27.20\\
68 & 53 & 0.0085 & 25.63 & 67 & 39 & 0.0050 & 21.52 & 65 & 58 & 0.0103 & 26.78\\
68 & 52 & 0.0082 & 25.29 & 67 & 38 & 0.0048 & 21.31 & 65 & 57 & 0.0099 & 26.38\\
68 & 51 & 0.0079 & 24.96 & 67 & 37 & 0.0046 & 21.11 & 65 & 56 & 0.0095 & 25.99\\
68 & 50 & 0.0076 & 24.64 & 67 & 36 & 0.0044 & 20.91 & 65 & 55 & 0.0092 & 25.62\\
68 & 49 & 0.0073 & 24.33 & 67 & 35 & 0.0043 & 20.72 & 65 & 54 & 0.0089 & 25.25\\
68 & 48 & 0.0071 & 24.03 & 67 & 34 & 0.0041 & 20.54 & 65 & 53 & 0.0085 & 24.90\\
68 & 47 & 0.0068 & 23.75 & 67 & 33 & 0.0039 & 20.36 & 65 & 52 & 0.0082 & 24.56\\
68 & 46 & 0.0066 & 23.47 & 67 & 32 & 0.0038 & 20.19 & 65 & 51 & 0.0079 & 24.23\\
68 & 45 & 0.0063 & 23.20 & 66 & 66 & 0.0137 & 30.76 & 65 & 50 & 0.0076 & 23.91\\
68 & 44 & 0.0061 & 22.94 & 66 & 65 & 0.0132 & 30.24 & 65 & 49 & 0.0073 & 23.60\\
68 & 43 & 0.0058 & 22.69 & 66 & 64 & 0.0127 & 29.73 & 65 & 48 & 0.0071 & 23.31\\
68 & 42 & 0.0056 & 22.45 & 66 & 63 & 0.0123 & 29.25 & 65 & 47 & 0.0068 & 23.02\\
\bottomrule
\end{tabular}
\newpage
\begin{tabular}{llll|llll|llll}
 \toprule 
\(T_{db}\) & \(T_{dp}\) & \(\omega\) & \(h\) & \(T_{db}\) & \(T_{dp}\) & \(\omega\) & \(h\) & \(T_{db}\) & \(T_{dp}\) & \(\omega\) & \(h\)  \\ \midrule 
65 & 46 & 0.0066 & 22.74 & 63 & 61 & 0.0114 & 27.58 & 62 & 42 & 0.0056 & 20.99\\
65 & 45 & 0.0063 & 22.47 & 63 & 60 & 0.0110 & 27.14 & 62 & 41 & 0.0054 & 20.76\\
65 & 44 & 0.0061 & 22.21 & 63 & 59 & 0.0106 & 26.71 & 62 & 40 & 0.0052 & 20.53\\
65 & 43 & 0.0058 & 21.96 & 63 & 58 & 0.0103 & 26.29 & 62 & 39 & 0.0050 & 20.31\\
65 & 42 & 0.0056 & 21.72 & 63 & 57 & 0.0099 & 25.89 & 62 & 38 & 0.0048 & 20.10\\
65 & 41 & 0.0054 & 21.48 & 63 & 56 & 0.0095 & 25.50 & 62 & 37 & 0.0046 & 19.90\\
65 & 40 & 0.0052 & 21.26 & 63 & 55 & 0.0092 & 25.13 & 62 & 36 & 0.0044 & 19.70\\
65 & 39 & 0.0050 & 21.04 & 63 & 54 & 0.0089 & 24.76 & 62 & 35 & 0.0043 & 19.51\\
65 & 38 & 0.0048 & 20.83 & 63 & 53 & 0.0085 & 24.41 & 62 & 34 & 0.0041 & 19.33\\
65 & 37 & 0.0046 & 20.63 & 63 & 52 & 0.0082 & 24.07 & 62 & 33 & 0.0039 & 19.16\\
65 & 36 & 0.0044 & 20.43 & 63 & 51 & 0.0079 & 23.74 & 62 & 32 & 0.0038 & 18.99\\
65 & 35 & 0.0043 & 20.24 & 63 & 50 & 0.0076 & 23.42 & 61 & 61 & 0.0114 & 27.09\\
65 & 34 & 0.0041 & 20.06 & 63 & 49 & 0.0073 & 23.12 & 61 & 60 & 0.0110 & 26.65\\
65 & 33 & 0.0039 & 19.88 & 63 & 48 & 0.0071 & 22.82 & 61 & 59 & 0.0106 & 26.22\\
65 & 32 & 0.0038 & 19.71 & 63 & 47 & 0.0068 & 22.53 & 61 & 58 & 0.0103 & 25.80\\
64 & 64 & 0.0127 & 29.24 & 63 & 46 & 0.0066 & 22.25 & 61 & 57 & 0.0099 & 25.40\\
64 & 63 & 0.0123 & 28.75 & 63 & 45 & 0.0063 & 21.99 & 61 & 56 & 0.0095 & 25.01\\
64 & 62 & 0.0119 & 28.28 & 63 & 44 & 0.0061 & 21.73 & 61 & 55 & 0.0092 & 24.64\\
64 & 61 & 0.0114 & 27.82 & 63 & 43 & 0.0058 & 21.48 & 61 & 54 & 0.0089 & 24.28\\
64 & 60 & 0.0110 & 27.38 & 63 & 42 & 0.0056 & 21.23 & 61 & 53 & 0.0085 & 23.92\\
64 & 59 & 0.0106 & 26.95 & 63 & 41 & 0.0054 & 21.00 & 61 & 52 & 0.0082 & 23.58\\
64 & 58 & 0.0103 & 26.54 & 63 & 40 & 0.0052 & 20.77 & 61 & 51 & 0.0079 & 23.26\\
64 & 57 & 0.0099 & 26.14 & 63 & 39 & 0.0050 & 20.56 & 61 & 50 & 0.0076 & 22.94\\
64 & 56 & 0.0095 & 25.75 & 63 & 38 & 0.0048 & 20.35 & 61 & 49 & 0.0073 & 22.63\\
64 & 55 & 0.0092 & 25.37 & 63 & 37 & 0.0046 & 20.14 & 61 & 48 & 0.0071 & 22.33\\
64 & 54 & 0.0089 & 25.01 & 63 & 36 & 0.0044 & 19.95 & 61 & 47 & 0.0068 & 22.05\\
64 & 53 & 0.0085 & 24.66 & 63 & 35 & 0.0043 & 19.76 & 61 & 46 & 0.0066 & 21.77\\
64 & 52 & 0.0082 & 24.32 & 63 & 34 & 0.0041 & 19.57 & 61 & 45 & 0.0063 & 21.50\\
64 & 51 & 0.0079 & 23.99 & 63 & 33 & 0.0039 & 19.40 & 61 & 44 & 0.0061 & 21.24\\
64 & 50 & 0.0076 & 23.67 & 63 & 32 & 0.0038 & 19.23 & 61 & 43 & 0.0058 & 20.99\\
64 & 49 & 0.0073 & 23.36 & 62 & 62 & 0.0119 & 27.79 & 61 & 42 & 0.0056 & 20.75\\
64 & 48 & 0.0071 & 23.06 & 62 & 61 & 0.0114 & 27.33 & 61 & 41 & 0.0054 & 20.51\\
64 & 47 & 0.0068 & 22.77 & 62 & 60 & 0.0110 & 26.89 & 61 & 40 & 0.0052 & 20.29\\
64 & 46 & 0.0066 & 22.50 & 62 & 59 & 0.0106 & 26.46 & 61 & 39 & 0.0050 & 20.07\\
64 & 45 & 0.0063 & 22.23 & 62 & 58 & 0.0103 & 26.05 & 61 & 38 & 0.0048 & 19.86\\
64 & 44 & 0.0061 & 21.97 & 62 & 57 & 0.0099 & 25.65 & 61 & 37 & 0.0046 & 19.66\\
64 & 43 & 0.0058 & 21.72 & 62 & 56 & 0.0095 & 25.26 & 61 & 36 & 0.0044 & 19.46\\
64 & 42 & 0.0056 & 21.48 & 62 & 55 & 0.0092 & 24.88 & 61 & 35 & 0.0043 & 19.27\\
64 & 41 & 0.0054 & 21.24 & 62 & 54 & 0.0089 & 24.52 & 61 & 34 & 0.0041 & 19.09\\
64 & 40 & 0.0052 & 21.02 & 62 & 53 & 0.0085 & 24.17 & 61 & 33 & 0.0039 & 18.91\\
64 & 39 & 0.0050 & 20.80 & 62 & 52 & 0.0082 & 23.83 & 61 & 32 & 0.0038 & 18.74\\
64 & 38 & 0.0048 & 20.59 & 62 & 51 & 0.0079 & 23.50 & 60 & 60 & 0.0110 & 26.40\\
64 & 37 & 0.0046 & 20.38 & 62 & 50 & 0.0076 & 23.18 & 60 & 59 & 0.0106 & 25.97\\
64 & 36 & 0.0044 & 20.19 & 62 & 49 & 0.0073 & 22.87 & 60 & 58 & 0.0103 & 25.56\\
64 & 35 & 0.0043 & 20.00 & 62 & 48 & 0.0071 & 22.58 & 60 & 57 & 0.0099 & 25.16\\
64 & 34 & 0.0041 & 19.82 & 62 & 47 & 0.0068 & 22.29 & 60 & 56 & 0.0095 & 24.77\\
64 & 33 & 0.0039 & 19.64 & 62 & 46 & 0.0066 & 22.01 & 60 & 55 & 0.0092 & 24.39\\
64 & 32 & 0.0038 & 19.47 & 62 & 45 & 0.0063 & 21.74 & 60 & 54 & 0.0089 & 24.03\\
63 & 63 & 0.0123 & 28.51 & 62 & 44 & 0.0061 & 21.48 & 60 & 53 & 0.0085 & 23.68\\
63 & 62 & 0.0119 & 28.04 & 62 & 43 & 0.0058 & 21.23 & 60 & 52 & 0.0082 & 23.34\\
\bottomrule
\end{tabular}
\newpage
\begin{tabular}{llll|llll|llll}
 \toprule 
\(T_{db}\) & \(T_{dp}\) & \(\omega\) & \(h\) & \(T_{db}\) & \(T_{dp}\) & \(\omega\) & \(h\) & \(T_{db}\) & \(T_{dp}\) & \(\omega\) & \(h\)  \\ \midrule 
60 & 51 & 0.0079 & 23.01 & 58 & 56 & 0.0095 & 24.28 & 57 & 32 & 0.0038 & 17.78\\
60 & 50 & 0.0076 & 22.69 & 58 & 55 & 0.0092 & 23.91 & 56 & 56 & 0.0095 & 23.79\\
60 & 49 & 0.0073 & 22.39 & 58 & 54 & 0.0089 & 23.54 & 56 & 55 & 0.0092 & 23.42\\
60 & 48 & 0.0071 & 22.09 & 58 & 53 & 0.0085 & 23.19 & 56 & 54 & 0.0089 & 23.06\\
60 & 47 & 0.0068 & 21.80 & 58 & 52 & 0.0082 & 22.85 & 56 & 53 & 0.0085 & 22.71\\
60 & 46 & 0.0066 & 21.53 & 58 & 51 & 0.0079 & 22.52 & 56 & 52 & 0.0082 & 22.37\\
60 & 45 & 0.0063 & 21.26 & 58 & 50 & 0.0076 & 22.21 & 56 & 51 & 0.0079 & 22.04\\
60 & 44 & 0.0061 & 21.00 & 58 & 49 & 0.0073 & 21.90 & 56 & 50 & 0.0076 & 21.72\\
60 & 43 & 0.0058 & 20.75 & 58 & 48 & 0.0071 & 21.60 & 56 & 49 & 0.0073 & 21.41\\
60 & 42 & 0.0056 & 20.51 & 58 & 47 & 0.0068 & 21.32 & 56 & 48 & 0.0071 & 21.12\\
60 & 41 & 0.0054 & 20.27 & 58 & 46 & 0.0066 & 21.04 & 56 & 47 & 0.0068 & 20.83\\
60 & 40 & 0.0052 & 20.05 & 58 & 45 & 0.0063 & 20.77 & 56 & 46 & 0.0066 & 20.55\\
60 & 39 & 0.0050 & 19.83 & 58 & 44 & 0.0061 & 20.51 & 56 & 45 & 0.0063 & 20.29\\
60 & 38 & 0.0048 & 19.62 & 58 & 43 & 0.0058 & 20.26 & 56 & 44 & 0.0061 & 20.03\\
60 & 37 & 0.0046 & 19.42 & 58 & 42 & 0.0056 & 20.02 & 56 & 43 & 0.0058 & 19.78\\
60 & 36 & 0.0044 & 19.22 & 58 & 41 & 0.0054 & 19.79 & 56 & 42 & 0.0056 & 19.54\\
60 & 35 & 0.0043 & 19.03 & 58 & 40 & 0.0052 & 19.56 & 56 & 41 & 0.0054 & 19.30\\
60 & 34 & 0.0041 & 18.85 & 58 & 39 & 0.0050 & 19.34 & 56 & 40 & 0.0052 & 19.08\\
60 & 33 & 0.0039 & 18.67 & 58 & 38 & 0.0048 & 19.13 & 56 & 39 & 0.0050 & 18.86\\
60 & 32 & 0.0038 & 18.50 & 58 & 37 & 0.0046 & 18.93 & 56 & 38 & 0.0048 & 18.65\\
59 & 59 & 0.0106 & 25.73 & 58 & 36 & 0.0044 & 18.74 & 56 & 37 & 0.0046 & 18.45\\
59 & 58 & 0.0103 & 25.31 & 58 & 35 & 0.0043 & 18.55 & 56 & 36 & 0.0044 & 18.25\\
59 & 57 & 0.0099 & 24.91 & 58 & 34 & 0.0041 & 18.36 & 56 & 35 & 0.0043 & 18.06\\
59 & 56 & 0.0095 & 24.53 & 58 & 33 & 0.0039 & 18.19 & 56 & 34 & 0.0041 & 17.88\\
59 & 55 & 0.0092 & 24.15 & 58 & 32 & 0.0038 & 18.02 & 56 & 33 & 0.0039 & 17.71\\
59 & 54 & 0.0089 & 23.79 & 57 & 57 & 0.0099 & 24.43 & 56 & 32 & 0.0038 & 17.54\\
59 & 53 & 0.0085 & 23.44 & 57 & 56 & 0.0095 & 24.04 & 55 & 55 & 0.0092 & 23.17\\
59 & 52 & 0.0082 & 23.10 & 57 & 55 & 0.0092 & 23.66 & 55 & 54 & 0.0089 & 22.81\\
59 & 51 & 0.0079 & 22.77 & 57 & 54 & 0.0089 & 23.30 & 55 & 53 & 0.0085 & 22.46\\
59 & 50 & 0.0076 & 22.45 & 57 & 53 & 0.0085 & 22.95 & 55 & 52 & 0.0082 & 22.12\\
59 & 49 & 0.0073 & 22.14 & 57 & 52 & 0.0082 & 22.61 & 55 & 51 & 0.0079 & 21.79\\
59 & 48 & 0.0071 & 21.85 & 57 & 51 & 0.0079 & 22.28 & 55 & 50 & 0.0076 & 21.48\\
59 & 47 & 0.0068 & 21.56 & 57 & 50 & 0.0076 & 21.96 & 55 & 49 & 0.0073 & 21.17\\
59 & 46 & 0.0066 & 21.28 & 57 & 49 & 0.0073 & 21.66 & 55 & 48 & 0.0071 & 20.87\\
59 & 45 & 0.0063 & 21.01 & 57 & 48 & 0.0071 & 21.36 & 55 & 47 & 0.0068 & 20.59\\
59 & 44 & 0.0061 & 20.76 & 57 & 47 & 0.0068 & 21.07 & 55 & 46 & 0.0066 & 20.31\\
59 & 43 & 0.0058 & 20.50 & 57 & 46 & 0.0066 & 20.80 & 55 & 45 & 0.0063 & 20.04\\
59 & 42 & 0.0056 & 20.26 & 57 & 45 & 0.0063 & 20.53 & 55 & 44 & 0.0061 & 19.78\\
59 & 41 & 0.0054 & 20.03 & 57 & 44 & 0.0061 & 20.27 & 55 & 43 & 0.0058 & 19.53\\
59 & 40 & 0.0052 & 19.80 & 57 & 43 & 0.0058 & 20.02 & 55 & 42 & 0.0056 & 19.29\\
59 & 39 & 0.0050 & 19.59 & 57 & 42 & 0.0056 & 19.78 & 55 & 41 & 0.0054 & 19.06\\
59 & 38 & 0.0048 & 19.38 & 57 & 41 & 0.0054 & 19.54 & 55 & 40 & 0.0052 & 18.84\\
59 & 37 & 0.0046 & 19.17 & 57 & 40 & 0.0052 & 19.32 & 55 & 39 & 0.0050 & 18.62\\
59 & 36 & 0.0044 & 18.98 & 57 & 39 & 0.0050 & 19.10 & 55 & 38 & 0.0048 & 18.41\\
59 & 35 & 0.0043 & 18.79 & 57 & 38 & 0.0048 & 18.89 & 55 & 37 & 0.0046 & 18.21\\
59 & 34 & 0.0041 & 18.61 & 57 & 37 & 0.0046 & 18.69 & 55 & 36 & 0.0044 & 18.01\\
59 & 33 & 0.0039 & 18.43 & 57 & 36 & 0.0044 & 18.49 & 55 & 35 & 0.0043 & 17.82\\
59 & 32 & 0.0038 & 18.26 & 57 & 35 & 0.0043 & 18.31 & 55 & 34 & 0.0041 & 17.64\\
58 & 58 & 0.0103 & 25.07 & 57 & 34 & 0.0041 & 18.12 & 55 & 33 & 0.0039 & 17.46\\
58 & 57 & 0.0099 & 24.67 & 57 & 33 & 0.0039 & 17.95 & 55 & 32 & 0.0038 & 17.29\\
\bottomrule
\end{tabular}
\newpage
\begin{tabular}{llll|llll|llll}
 \toprule 
\(T_{db}\) & \(T_{dp}\) & \(\omega\) & \(h\) & \(T_{db}\) & \(T_{dp}\) & \(\omega\) & \(h\) & \(T_{db}\) & \(T_{dp}\) & \(\omega\) & \(h\)  \\ \midrule 
54 & 54 & 0.0089 & 22.57 & 52 & 47 & 0.0068 & 19.86 & 50 & 36 & 0.0044 & 16.80\\
54 & 53 & 0.0085 & 22.22 & 52 & 46 & 0.0066 & 19.58 & 50 & 35 & 0.0043 & 16.61\\
54 & 52 & 0.0082 & 21.88 & 52 & 45 & 0.0063 & 19.31 & 50 & 34 & 0.0041 & 16.43\\
54 & 51 & 0.0079 & 21.55 & 52 & 44 & 0.0061 & 19.06 & 50 & 33 & 0.0039 & 16.26\\
54 & 50 & 0.0076 & 21.23 & 52 & 43 & 0.0058 & 18.81 & 50 & 32 & 0.0038 & 16.09\\
54 & 49 & 0.0073 & 20.93 & 52 & 42 & 0.0056 & 18.57 & 49 & 49 & 0.0073 & 19.71\\
54 & 48 & 0.0071 & 20.63 & 52 & 41 & 0.0054 & 18.33 & 49 & 48 & 0.0071 & 19.42\\
54 & 47 & 0.0068 & 20.34 & 52 & 40 & 0.0052 & 18.11 & 49 & 47 & 0.0068 & 19.13\\
54 & 46 & 0.0066 & 20.07 & 52 & 39 & 0.0050 & 17.89 & 49 & 46 & 0.0066 & 18.85\\
54 & 45 & 0.0063 & 19.80 & 52 & 38 & 0.0048 & 17.68 & 49 & 45 & 0.0063 & 18.59\\
54 & 44 & 0.0061 & 19.54 & 52 & 37 & 0.0046 & 17.48 & 49 & 44 & 0.0061 & 18.33\\
54 & 43 & 0.0058 & 19.29 & 52 & 36 & 0.0044 & 17.28 & 49 & 43 & 0.0058 & 18.08\\
54 & 42 & 0.0056 & 19.05 & 52 & 35 & 0.0043 & 17.10 & 49 & 42 & 0.0056 & 17.84\\
54 & 41 & 0.0054 & 18.82 & 52 & 34 & 0.0041 & 16.91 & 49 & 41 & 0.0054 & 17.61\\
54 & 40 & 0.0052 & 18.59 & 52 & 33 & 0.0039 & 16.74 & 49 & 40 & 0.0052 & 17.38\\
54 & 39 & 0.0050 & 18.38 & 52 & 32 & 0.0038 & 16.57 & 49 & 39 & 0.0050 & 17.16\\
54 & 38 & 0.0048 & 18.17 & 51 & 51 & 0.0079 & 20.82 & 49 & 38 & 0.0048 & 16.96\\
54 & 37 & 0.0046 & 17.96 & 51 & 50 & 0.0076 & 20.50 & 49 & 37 & 0.0046 & 16.75\\
54 & 36 & 0.0044 & 17.77 & 51 & 49 & 0.0073 & 20.20 & 49 & 36 & 0.0044 & 16.56\\
54 & 35 & 0.0043 & 17.58 & 51 & 48 & 0.0071 & 19.90 & 49 & 35 & 0.0043 & 16.37\\
54 & 34 & 0.0041 & 17.40 & 51 & 47 & 0.0068 & 19.62 & 49 & 34 & 0.0041 & 16.19\\
54 & 33 & 0.0039 & 17.22 & 51 & 46 & 0.0066 & 19.34 & 49 & 33 & 0.0039 & 16.01\\
54 & 32 & 0.0038 & 17.05 & 51 & 45 & 0.0063 & 19.07 & 49 & 32 & 0.0038 & 15.84\\
53 & 53 & 0.0085 & 21.97 & 51 & 44 & 0.0061 & 18.81 & 48 & 48 & 0.0071 & 19.17\\
53 & 52 & 0.0082 & 21.63 & 51 & 43 & 0.0058 & 18.56 & 48 & 47 & 0.0068 & 18.89\\
53 & 51 & 0.0079 & 21.31 & 51 & 42 & 0.0056 & 18.32 & 48 & 46 & 0.0066 & 18.61\\
53 & 50 & 0.0076 & 20.99 & 51 & 41 & 0.0054 & 18.09 & 48 & 45 & 0.0063 & 18.34\\
53 & 49 & 0.0073 & 20.68 & 51 & 40 & 0.0052 & 17.87 & 48 & 44 & 0.0061 & 18.09\\
53 & 48 & 0.0071 & 20.39 & 51 & 39 & 0.0050 & 17.65 & 48 & 43 & 0.0058 & 17.84\\
53 & 47 & 0.0068 & 20.10 & 51 & 38 & 0.0048 & 17.44 & 48 & 42 & 0.0056 & 17.60\\
53 & 46 & 0.0066 & 19.82 & 51 & 37 & 0.0046 & 17.24 & 48 & 41 & 0.0054 & 17.36\\
53 & 45 & 0.0063 & 19.56 & 51 & 36 & 0.0044 & 17.04 & 48 & 40 & 0.0052 & 17.14\\
53 & 44 & 0.0061 & 19.30 & 51 & 35 & 0.0043 & 16.85 & 48 & 39 & 0.0050 & 16.92\\
53 & 43 & 0.0058 & 19.05 & 51 & 34 & 0.0041 & 16.67 & 48 & 38 & 0.0048 & 16.71\\
53 & 42 & 0.0056 & 18.81 & 51 & 33 & 0.0039 & 16.50 & 48 & 37 & 0.0046 & 16.51\\
53 & 41 & 0.0054 & 18.58 & 51 & 32 & 0.0038 & 16.33 & 48 & 36 & 0.0044 & 16.32\\
53 & 40 & 0.0052 & 18.35 & 50 & 50 & 0.0076 & 20.26 & 48 & 35 & 0.0043 & 16.13\\
53 & 39 & 0.0050 & 18.13 & 50 & 49 & 0.0073 & 19.95 & 48 & 34 & 0.0041 & 15.95\\
53 & 38 & 0.0048 & 17.92 & 50 & 48 & 0.0071 & 19.66 & 48 & 33 & 0.0039 & 15.77\\
53 & 37 & 0.0046 & 17.72 & 50 & 47 & 0.0068 & 19.37 & 48 & 32 & 0.0038 & 15.60\\
53 & 36 & 0.0044 & 17.53 & 50 & 46 & 0.0066 & 19.10 & 47 & 47 & 0.0068 & 18.64\\
53 & 35 & 0.0043 & 17.34 & 50 & 45 & 0.0063 & 18.83 & 47 & 46 & 0.0066 & 18.37\\
53 & 34 & 0.0041 & 17.16 & 50 & 44 & 0.0061 & 18.57 & 47 & 45 & 0.0063 & 18.10\\
53 & 33 & 0.0039 & 16.98 & 50 & 43 & 0.0058 & 18.32 & 47 & 44 & 0.0061 & 17.84\\
53 & 32 & 0.0038 & 16.81 & 50 & 42 & 0.0056 & 18.08 & 47 & 43 & 0.0058 & 17.59\\
52 & 52 & 0.0082 & 21.39 & 50 & 41 & 0.0054 & 17.85 & 47 & 42 & 0.0056 & 17.35\\
52 & 51 & 0.0079 & 21.06 & 50 & 40 & 0.0052 & 17.62 & 47 & 41 & 0.0054 & 17.12\\
52 & 50 & 0.0076 & 20.75 & 50 & 39 & 0.0050 & 17.41 & 47 & 40 & 0.0052 & 16.90\\
52 & 49 & 0.0073 & 20.44 & 50 & 38 & 0.0048 & 17.20 & 47 & 39 & 0.0050 & 16.68\\
52 & 48 & 0.0071 & 20.14 & 50 & 37 & 0.0046 & 17.00 & 47 & 38 & 0.0048 & 16.47\\
\bottomrule
\end{tabular}
\newpage
\begin{tabular}{llll|llll|llll}
 \toprule 
\(T_{db}\) & \(T_{dp}\) & \(\omega\) & \(h\) & \(T_{db}\) & \(T_{dp}\) & \(\omega\) & \(h\) & \(T_{db}\) & \(T_{dp}\) & \(\omega\) & \(h\)  \\ \midrule 
47 & 37 & 0.0046 & 16.27 & 43 & 41 & 0.0054 & 16.15 & 38 & 36 & 0.0044 & 13.90\\
47 & 36 & 0.0044 & 16.07 & 43 & 40 & 0.0052 & 15.93 & 38 & 35 & 0.0043 & 13.71\\
47 & 35 & 0.0043 & 15.89 & 43 & 39 & 0.0050 & 15.71 & 38 & 34 & 0.0041 & 13.53\\
47 & 34 & 0.0041 & 15.70 & 43 & 38 & 0.0048 & 15.50 & 38 & 33 & 0.0039 & 13.35\\
47 & 33 & 0.0039 & 15.53 & 43 & 37 & 0.0046 & 15.30 & 38 & 32 & 0.0038 & 13.19\\
47 & 32 & 0.0038 & 15.36 & 43 & 36 & 0.0044 & 15.11 & 37 & 37 & 0.0046 & 13.85\\
46 & 46 & 0.0066 & 18.12 & 43 & 35 & 0.0043 & 14.92 & 37 & 36 & 0.0044 & 13.65\\
46 & 45 & 0.0063 & 17.86 & 43 & 34 & 0.0041 & 14.74 & 37 & 35 & 0.0043 & 13.47\\
46 & 44 & 0.0061 & 17.60 & 43 & 33 & 0.0039 & 14.56 & 37 & 34 & 0.0041 & 13.29\\
46 & 43 & 0.0058 & 17.35 & 43 & 32 & 0.0038 & 14.39 & 37 & 33 & 0.0039 & 13.11\\
46 & 42 & 0.0056 & 17.11 & 42 & 42 & 0.0056 & 16.14 & 37 & 32 & 0.0038 & 12.94\\
46 & 41 & 0.0054 & 16.88 & 42 & 41 & 0.0054 & 15.91 & 36 & 36 & 0.0044 & 13.41\\
46 & 40 & 0.0052 & 16.65 & 42 & 40 & 0.0052 & 15.69 & 36 & 35 & 0.0043 & 13.23\\
46 & 39 & 0.0050 & 16.44 & 42 & 39 & 0.0050 & 15.47 & 36 & 34 & 0.0041 & 13.04\\
46 & 38 & 0.0048 & 16.23 & 42 & 38 & 0.0048 & 15.26 & 36 & 33 & 0.0039 & 12.87\\
46 & 37 & 0.0046 & 16.03 & 42 & 37 & 0.0046 & 15.06 & 36 & 32 & 0.0038 & 12.70\\
46 & 36 & 0.0044 & 15.83 & 42 & 36 & 0.0044 & 14.86 & 35 & 35 & 0.0043 & 12.98\\
46 & 35 & 0.0043 & 15.64 & 42 & 35 & 0.0043 & 14.68 & 35 & 34 & 0.0041 & 12.80\\
46 & 34 & 0.0041 & 15.46 & 42 & 34 & 0.0041 & 14.50 & 35 & 33 & 0.0039 & 12.63\\
46 & 33 & 0.0039 & 15.29 & 42 & 33 & 0.0039 & 14.32 & 35 & 32 & 0.0038 & 12.46\\
46 & 32 & 0.0038 & 15.12 & 42 & 32 & 0.0038 & 14.15 & 34 & 34 & 0.0041 & 12.56\\
45 & 45 & 0.0063 & 17.61 & 41 & 41 & 0.0054 & 15.67 & 34 & 33 & 0.0039 & 12.39\\
45 & 44 & 0.0061 & 17.36 & 41 & 40 & 0.0052 & 15.44 & 34 & 32 & 0.0038 & 12.22\\
45 & 43 & 0.0058 & 17.11 & 41 & 39 & 0.0050 & 15.23 & 33 & 33 & 0.0039 & 12.15\\
45 & 42 & 0.0056 & 16.87 & 41 & 38 & 0.0048 & 15.02 & 33 & 32 & 0.0038 & 11.98\\
45 & 41 & 0.0054 & 16.64 & 41 & 37 & 0.0046 & 14.82 & 32 & 32 & 0.0038 & 11.74\\
45 & 40 & 0.0052 & 16.41 & 41 & 36 & 0.0044 & 14.62 &  &  &  & \\
45 & 39 & 0.0050 & 16.20 & 41 & 35 & 0.0043 & 14.43 &  &  &  & \\
45 & 38 & 0.0048 & 15.99 & 41 & 34 & 0.0041 & 14.25 &  &  &  & \\
45 & 37 & 0.0046 & 15.78 & 41 & 33 & 0.0039 & 14.08 &  &  &  & \\
45 & 36 & 0.0044 & 15.59 & 41 & 32 & 0.0038 & 13.91 &  &  &  & \\
45 & 35 & 0.0043 & 15.40 & 40 & 40 & 0.0052 & 15.20 &  &  &  & \\
45 & 34 & 0.0041 & 15.22 & 40 & 39 & 0.0050 & 14.98 &  &  &  & \\
45 & 33 & 0.0039 & 15.05 & 40 & 38 & 0.0048 & 14.78 &  &  &  & \\
45 & 32 & 0.0038 & 14.88 & 40 & 37 & 0.0046 & 14.57 &  &  &  & \\
44 & 44 & 0.0061 & 17.11 & 40 & 36 & 0.0044 & 14.38 &  &  &  & \\
44 & 43 & 0.0058 & 16.87 & 40 & 35 & 0.0043 & 14.19 &  &  &  & \\
44 & 42 & 0.0056 & 16.63 & 40 & 34 & 0.0041 & 14.01 &  &  &  & \\
44 & 41 & 0.0054 & 16.39 & 40 & 33 & 0.0039 & 13.84 &  &  &  & \\
44 & 40 & 0.0052 & 16.17 & 40 & 32 & 0.0038 & 13.67 &  &  &  & \\
44 & 39 & 0.0050 & 15.95 & 39 & 39 & 0.0050 & 14.74 &  &  &  & \\
44 & 38 & 0.0048 & 15.74 & 39 & 38 & 0.0048 & 14.53 &  &  &  & \\
44 & 37 & 0.0046 & 15.54 & 39 & 37 & 0.0046 & 14.33 &  &  &  & \\
44 & 36 & 0.0044 & 15.35 & 39 & 36 & 0.0044 & 14.14 &  &  &  & \\
44 & 35 & 0.0043 & 15.16 & 39 & 35 & 0.0043 & 13.95 &  &  &  & \\
44 & 34 & 0.0041 & 14.98 & 39 & 34 & 0.0041 & 13.77 &  &  &  & \\
44 & 33 & 0.0039 & 14.80 & 39 & 33 & 0.0039 & 13.60 &  &  &  & \\
44 & 32 & 0.0038 & 14.64 & 39 & 32 & 0.0038 & 13.43 &  &  &  & \\
43 & 43 & 0.0058 & 16.62 & 38 & 38 & 0.0048 & 14.29 &  &  &  & \\
43 & 42 & 0.0056 & 16.38 & 38 & 37 & 0.0046 & 14.09 &  &  &  & \\
\bottomrule
\end{tabular}
\newpage



}


%{
%\small
%\begin{tabular}{llll|llll|llll}

    %\input{../psychrometrics/table.tex}

%\end{tabular}

%}

%\newpage


\end{document}
